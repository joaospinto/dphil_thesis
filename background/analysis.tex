\subsection{Laurent polynomials}

A multivariate \textbf{Laurent polynomial} is a polynomial in positive
and negative powers of variables $z_1,\ldots,z_s$ with complex
coefficients.  We are interested in Laurent polynomials of the special form
\[ g = \sum_{j=1}^k \left( c_j {z_1}^{n_{1,j}}\ldots {z_s}^{n_{s,j}} +
    \overline{c_j} {z_1}^{-n_{1,j}}\ldots {z_s}^{-n_{s,j}} \right) \,
  ,\] where $c_1,\ldots,c_k \in \mathbb{C}$ and
$n_{1,1},\ldots,n_{s,k} \in \Integers$.  We call such $g$
\textbf{self-conjugate Laurent polynomials}.  Notice that if
$a_1,\ldots,a_s \in \mathbb{T}$ then $g(a_1,\ldots,a_s)$ is a real
number, so we may regard $g$ as a function from $\mathbb{T}^s$ to
$\mathbb{R}$.

We say that $g$ is \textbf{simple} if $g$ has no constant term and
each monomial in $g$ mentions only a single variable.

\begin{proposition}
\label{first}
Let $g \in \mathbb{C}[z^{\pm 1}_1,\ldots,z^{\pm 1}_s]$ be a
self-conjugate Laurent polynomial that has no constant term.
Given real numbers $\theta_1,\ldots,\theta_s \in \mathbb{R}$ such that
$\theta_1,\ldots,\theta_s,1$ are linearly independent over
$\mathbb{Q}$, define a function
$f:\mathbb{R}_{\geq 0}\rightarrow \mathbb{R}$ by
\[ f(t) = g(\phi(t\theta_1),\ldots,\phi(t\theta_s)) \, .\]
Then either $f$ is identically zero, or
\begin{align*}
\liminf\limits_{n\rightarrow\infty} f(n) < 0 \, ,
\end{align*}
where $n$ ranges over the nonnegative integers.
\end{proposition}
\begin{proof}
  Recall that we may regard $g$ as a function from
$\mathbb{T}^s$ to $\mathbb{R}$.  Now we consider the
  function $g \circ \phi^s : \mathbb{R}^s \rightarrow \mathbb{R}$,
  \[ (x_1,\ldots,x_s) \mapsto g(\phi(x_1),\ldots,\phi(x_s)) \, . \] We
  use an averaging argument to establish that either $g\circ\phi^s$ is
  identically zero on $\mathbb{R}^s$ or there exist
  $x_1^*,\ldots,x_s^* \in [0,2\pi]$ such that
  $g(\phi(x_1^*),\ldots,\phi(x_s^*))<0$.

  Since $\int_0^{2\pi} \exp(2\pi i n x) dx=0$ for all non-zero
  integers $n$, it holds that
\begin{equation*}
\int_0^{2\pi} \ldots \int_0^{2\pi}
g(\phi(x_1),\ldots,\phi(x_s))
dx_1 \ldots dx_s =0 \, .
\end{equation*}
%
Suppose that
$g\circ \phi^s$ is not identically zero over
$\mathbb{R}^s$ and hence not identically zero over $[0,2\pi]^{s}$.
Then $g\circ \phi^s$ cannot be nonnegative on $[0,2\pi]^{s}$, since
the integral over a set of positive measure of a continuous
nonnegative function that is not identically zero must be strictly
positive.  We conclude that there must exist
$(x_1^*,\ldots,x_s^*) \in [0,2\pi]^{s}$ such that
$g(\phi(x_1^*),\ldots,\phi(x_s^*))<0$.

By assumption, $\theta_1,\ldots,\theta_s,1$ are linearly
independent over $\mathbb{Q}$.  By \cref{corl:kronecker} it
follows that
\[ \{ (\phi(n\theta_1),\ldots,\phi(n\theta_s)) : n \in
  \Naturals \} \] is dense in $\mathbb{T}^s$ and hence has
$(\phi(x_1^*),\ldots,\phi(x_s^*))$ as a limit point.
Since $g\circ\phi^s$ is continuous, there are
arbitrarily large $n\in\Naturals$ for which
\[
f(n) = g(\phi(n\theta_1),\ldots,\phi(n\theta_s))
    \leq \textstyle\frac{1}{2} g(\phi(x_1^*),\ldots,\phi(x_s^*)) < 0 \, , \]
which proves the result.
\end{proof}

Note that this proof could be made constructive by using an effective
version of Kronecker's Theorem, as studied in~\cite{ConstructiveKronecker1} and~\cite{ConstructiveKronecker2},
although we do not make use of this fact in the present paper.

The following consequence of \cref{first} will be key to
proving decidability of the problem at hand. It is a continuous-time
analogue of Lemma 4 in~\cite{Bra06}.

\begin{proposition}
\label{prop:liminf}
Let $g\in\mathbb{C}[z_1^{\pm 1},\ldots,z_s^{\pm 1}]$ be a simple
self-conjugate Laurent polynomial and $\theta_1,\ldots,\theta_s$
non-zero real numbers.  Then either
\begin{align*}
& g(\phi(t\theta_1),\ldots,\phi(t\theta_s)) = 0 \mbox{ for all $t\in\mathbb{R}$}
\intertext{or}
&
\liminf\limits_{n\rightarrow\infty}  g(\phi(n\theta_1),\ldots,\phi(n\theta_s)) < 0 \, ,
\end{align*}
where $n$ ranges over the nonnegative integers.
\end{proposition}

\begin{proof}
Note that if $\theta_1,\ldots,\theta_s,1$ are linearly
  independent over $\mathbb{Q}$ the the result follows
  from \cref{first}.

More generally, let $\{\theta_1,\ldots,\theta_k\}$ be a maximal
  subset of $\{\theta_1,\ldots,\theta_s\}$ such that
$\theta_1,\ldots,\theta_k,1$ are linearly independent over $\mathbb{Q}$.
Note that $k\geq 1$ since the $\theta_i$ are non-zero.

Then, for some $N\in\Naturals$ and each $j\geq k+1$, one can write
\begin{equation*}
N\theta_{j}= \left( m  +\sum\limits_{i=1}^{k} n_{i}\theta_{i}\right) \, ,
\end{equation*}
where $m,n_{1},\ldots,n_{k}$ are integers that depend on $j$, whilst
$N$ does not depend on $j$.  It follows that for all $j\geq k+1$ and
$t \in \mathbb{R}$,
\begin{align*}
\phi(N\theta_{j}t) &= \phi(m t) \cdot \prod\limits_{i=1}^{k} \phi( n_i \theta_{i} t) \\
&= \phi(t)^m \cdot \prod\limits_{i=1}^{k} \phi( \theta_{i} t)^{n_i}  \, .
\end{align*}
In other words, for $j=1,\ldots,s$, $\phi(N\theta_j t)$ can be written
as a product of positive and negative powers of the terms
$\phi(t), \phi(\theta_1 t),\ldots,\phi(\theta_k t)$.
It follows that there exists a self-conjugate Laurent polynomial
$h\in\mathbb{C}[z_1^{\pm 1},\ldots,z_k^{\pm 1}]$, not necessarily
simple, but with zero constant term, such that for all
$t\in \mathbb{R}$,
\[ g(\phi(N\theta_1t),\ldots,\phi(N\theta_s t)) =
  h(\phi(\theta_1t),\ldots,\phi(\theta_k t)) \, .\] Since
$\theta_1,\ldots,\theta_k,1$ are linearly independent over
$\mathbb{Q}$, the result follows by applying \cref{first}
to $h$.
\end{proof}

%%%%%%%%%%%%%%%%%%%%%%%%%%(removed something from here)%%%%%%%%%%%%%%

In order to compare the asymptotic growth of expressions of the form
$t^{n}\exp(\lambda t)$, for $\lambda\in\mathbb{R}$ and
$n\in\Naturals_0$, we define $\prec$ to be the lexicographic order on
$\mathbb{R}\times\Naturals_{0}$, that is,
\begin{equation*}
(\eta,j)\prec (\rho,m) \quad \mbox{iff} \quad
\eta<\rho \mbox{ or } (\eta = \rho \mbox{ and } j< m) \, .
\end{equation*}
Clearly $\exp(\eta t)t^{j}=o(\exp(\rho t)t^{m})$ as $t\rightarrow \infty$ if and only if $(\eta,j)\prec (\rho,m)$.

\begin{definition}
If $\boldsymbol{b}^{T}\exp(At)\boldsymbol{v}$ is not identically zero,
the maximal $(\rho,m)\in\mathbb{R}\times\Naturals_{0}$ with respect
to $\prec$ for which there is a term $t^{m}\exp (\lambda t)$ with
$\Re(\lambda)=\rho$ in the closed-form expression for
$\boldsymbol{b}^{T}\exp(At)\boldsymbol{v}$ is called \emph{dominant} for
$\boldsymbol{b}^{T}\exp(At)\boldsymbol{v}$.
\end{definition}

Before we can proceed, we shall need the following auxiliary result:

\begin{proposition}
\label{conj-relation}
Suppose that $\boldsymbol{v}\in\mathbb{R}^{d}$ and that $\boldsymbol{v}=\sum\limits_{\lambda\in\sigma(A)} \boldsymbol{v}_{\lambda}$, where $\boldsymbol{v}_{\lambda} \in\mathcal{V}_{\lambda}$. Then $\boldsymbol{v}_{\overline{\lambda}}$ and $\boldsymbol{v}_{\lambda}$ are component-wise complex conjugates.
\end{proposition}

\begin{proof}
  Note that $\boldsymbol{v}_{\lambda}\in \ker(A-\lambda I)^{k}$
  implies that
  $\overline{\boldsymbol{v}_{\lambda}} \in \ker(A-\overline{\lambda}
  I)^{k}$
  and hence that
  $\overline{ \boldsymbol{v}_{ \overline{\lambda}}} \in
  \mathcal{V}_{\lambda}$.  The result follows from the fact that
\begin{align*}
\boldsymbol{0}=\boldsymbol{v}-\overline{\boldsymbol{v}}=\sum\limits_{\lambda\in \sigma(A)}(\boldsymbol{v}_{\lambda}-\overline{ \boldsymbol{v}_{ \overline{\lambda}}})
\end{align*}
and from uniqueness of the above decomposition.
%
%For each $\lambda\in\sigma(A)$, let $f_{\lambda}$ be the function mapping each $\boldsymbol{v}$ to the corresponding $\boldsymbol{v}_{\lambda}$. Fix a basis $\mathcal{B}$ of generalised eigenvectors of $A$, with the property
%\begin{align*}
%\boldsymbol{x}\in\mathcal{B}\Rightarrow \overline{\boldsymbol{x}}\in\mathcal{B}
%\end{align*}
%For each $\boldsymbol{x}\in\mathcal{B}$, we define a linear functional $g_{\boldsymbol{x}}$ so that
%\begin{align*}
%g_{\boldsymbol{x}}(\sum\limits_{\boldsymbol{y}\in\mathcal{B}} \alpha_{\boldsymbol{y}}\boldsymbol{y})=\alpha_{\boldsymbol{x}}
%\end{align*}
%Then $\forall \boldsymbol{u}\in\mathbb{C}^{d}, \overline{g_{\overline{\boldsymbol{x}}}(\overline{\boldsymbol{u}})}=g_{\boldsymbol{x}}(\boldsymbol{u})$, as this equality certainly holds when $\boldsymbol{u}\in\mathcal{B}$, and by linearity of both sides. The result follows from the fact that
%\begin{align*}
%\forall \boldsymbol{u}\in\mathbb{R}^{d}, f_{\lambda}(\boldsymbol{u})= \sum\limits_{\boldsymbol{x}\in\mathcal{B}:\boldsymbol{x}\in\mathcal{V}_{\lambda}} g_{\boldsymbol{x}}(\boldsymbol{u})\boldsymbol{x}
%\end{align*}
%
\end{proof}

\begin{proposition}
\label{liminfprop}
Consider a function of the form $h(t)=\boldsymbol{b}^{T}\exp(At) \boldsymbol{v}^{c}$, where $\boldsymbol{v}^{c}\in\mathcal{V}^{c}$, with $(\rho,m)\in\mathbb{R}\times \Naturals_{0}$ dominant. If $h(t)\not\equiv 0$, then we have
\begin{equation*}
-\infty<\liminf\limits_{t\rightarrow\infty} \frac{h(t)}{\exp(\rho
  t)t^{m}}<0 \, .
\end{equation*}
\end{proposition}

\begin{proof}
  Let
  $\Re(\sigma(A))=\lbrace \eta\in\mathbb{R}:
  \eta+i\theta\in\sigma(A),\mbox{ for some }\theta\in\mathbb{R}
  \rbrace$. Moreover, for $\eta\in\Re(\sigma(A))$, we define
  $\boldsymbol{\theta}_{\eta}=\lbrace \theta\in\mathbb{R}_{>0}:
  \eta+i\theta \in\sigma(A) \rbrace$. By abuse of notation, we also
  use $\boldsymbol{\theta}_{\eta}$ to refer to the vector whose
  coordinates are exactly the members of this set, ordered in an
  increasing way. We note that, due to \cref{conj-relation}
  and \cref{prop:linear}, the
  following holds:

\begin{align*}
\boldsymbol{b}^{T}\exp(At)\boldsymbol{v}^{c} &= \boldsymbol{b}^{T} \exp(At) \sum\limits_{\eta\in\Re(\sigma(A))} \sum\limits_{\theta\in\boldsymbol{\theta}_{\eta}} \boldsymbol{v}_{\eta+i\theta}+\boldsymbol{v}_{\eta-i\theta} \\
&= \sum\limits_{\eta\in\Re(\sigma(A))} \sum\limits_{\theta\in\boldsymbol{\theta}_{\eta}} \boldsymbol{b}^{T} \exp(At) \boldsymbol{v}_{\eta+i\theta} \\
& \qquad \qquad \qquad \qquad \qquad + \overline{\boldsymbol{b}^{T} \exp(At) \boldsymbol{v}_{\eta+i\theta}} \\
& = \sum\limits_{\eta\in\Re(\sigma(A))} \sum\limits_{j=0}^{\nu(A)-1} t^{j}\exp(\eta t)  g_{(\eta,j)}( \exp(i\boldsymbol{\theta}_{\eta}t) )
\end{align*}
for some simple self-conjugate Laurent polynomials
$g_{(\eta,j)}$.
Note that
\begin{equation*}
(\rho,m)=\max\limits_{\prec} \lbrace (\eta,j)\in\mathbb{R}\times
\Naturals_{0}: g_{(\eta,j)}(\exp(i\boldsymbol{\theta}_{\eta} t)) \not
\equiv 0 \rbrace \, .
\end{equation*}

The result then follows from \cref{prop:liminf} and the fact
that
\begin{align*}
\liminf\limits_{t\rightarrow\infty} \frac{h(t)}{\exp(\rho t)t^{m}}=
  \liminf\limits_{t\rightarrow\infty}
  g_{(\rho,m)}(\exp(i\boldsymbol{\theta}_{\rho} t)) \, .
\end{align*}
\end{proof}
