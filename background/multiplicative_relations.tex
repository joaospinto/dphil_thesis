%BEGIN SODA
\section{Groups of Multiplicative Relations}
\label{sec:mult}

This section introduces some concepts concerning groups of
multiplicative relations among algebraic numbers.  Here we will assume
some basic notions from algebraic number theory and the first-order
theory of reals.  We assume also a natural first-order interpretation of
the field of complex numbers in the ordered field of real numbers (in
which each complex number is encoded as a pair comprising its real and
imaginary parts).  Under this interpretation we refer to sets of
complex numbers as being semi-algebraic and first-order definable.
Details of the relevant notions can be found in the Appendix.

Let $\mathbb{T}=\lbrace z\in\mathbb{C}: \lvert z\rvert =1\rbrace$.  We
define the \emph{$s$-dimensional torus} to be $\mathbb{T}^s$,
considered as a group under componentwise multiplication.

Given a tuple of algebraic numbers
$\boldsymbol\lambda=(\lambda_1,\ldots,\lambda_s)$, in this section we
consider how to effectively represent the \emph{orbit} $\{
\boldsymbol\lambda^n : n \in \mathbb{N}\}$.  More
precisely, we will give an algebraic representation of the topological
closure of that orbit in $\mathbb{T}^s$.

The \emph{group of multiplicative relations} of
$\boldsymbol{\lambda}$, which is an additive subgroup of
$\mathbb{Z}^s$, is defined as
\begin{equation*}
L(\boldsymbol{\lambda})=\lbrace \boldsymbol{v}\in \mathbb{Z}^s : \boldsymbol\lambda^{\boldsymbol v}=1 \rbrace \, ,
\end{equation*}
where $\boldsymbol\lambda^{\boldsymbol v}$ is defined to be
$\lambda_1^{v_1}\cdots\lambda_s^{v_s}$ for $\boldsymbol{v}\in \mathbb{Z}^s$, that is, exponentiation acts
coordinatewise.

Since $\mathbb{Z}^s$ is a free abelian group, its subgroups are also
free.  In particular, $L(\boldsymbol\lambda)$ has a finite basis. The
following powerful theorem of Masser~\cite{Mas88} gives bounds on the
magnitude of the components of such a basis.

\begin{theorem}[Masser]
\label{masser}
The free abelian group $L(\boldsymbol{\lambda})$ has a basis $\boldsymbol{v}_1,\cdots,\boldsymbol{v}_l\in\mathbb{Z}^s$ for which
\[ \max\limits_{1\leq i\leq l,1\leq j\leq s} \lvert v_{i,j} \rvert \leq (D\log H)^{O(s^2)} \]
where $H$ and $D$ bound respectively the heights and degrees of all the $\lambda_i$.
\end{theorem}
Membership of a tuple $\boldsymbol{v}\in \mathbb{Z}^s$ in
$L(\boldsymbol{\lambda})$ can be computed in polynomial space, using a
decision procedure for the existential theory of the reals.  In
combination with Theorem~\ref{masser}, it follows that we can compute
a basis for $L(\boldsymbol{\lambda})$ in polynomial space by
brute-force search.

Corresponding to $L(\boldsymbol{\lambda})$, we consider the following
multiplicative subgroup of $\mathbb{T}^s$:
\begin{equation*}
T(\boldsymbol{\lambda})=\lbrace \boldsymbol \mu\in\mathbb{T}^s : \forall \boldsymbol v\in L(\boldsymbol\lambda),\,\boldsymbol\mu^{\boldsymbol v}=1\rbrace \, .
\end{equation*}
If $V$ is a basis of $L(\boldsymbol{\lambda})$ then we can
equivalently characterise $T(\boldsymbol{\lambda})$ as $\{
\boldsymbol{\mu} \in \mathbb{T}^s: \forall \boldsymbol{v}\in
V,\,\,\boldsymbol\mu^{\boldsymbol v}=1\}$.  Crucially, this finitary
characterisation allows us to represent $T(\boldsymbol\lambda)$ as a
semi-algebraic set.

We will use the following classical lemma of Kronecker on simultaneous
Diophantine approximation, in order to show that the orbit $\lbrace
\boldsymbol\lambda^n : n\in\mathbb{N} \rbrace$ is a dense subset of
$T(\boldsymbol{\lambda})$.

\begin{lemma}
  Let $\boldsymbol \theta,\boldsymbol \psi\in\mathbb{R}^s$. Suppose that for all $\boldsymbol v\in\mathbb{Z}^s$, if
  $\boldsymbol v^T\boldsymbol \theta\in\mathbb{Z}$ then also
  $\boldsymbol v^T\boldsymbol\psi\in\mathbb{Z}$, i.e., all integer
  relations among the coordinates of $\boldsymbol \theta$ also hold
  among those of $\boldsymbol\psi$ (modulo $\mathbb{Z}$). Then, for
  each $\varepsilon>0$, there exist $\boldsymbol p\in\mathbb{Z}^s$ and
  a non-negative integer $n$ such that
\[ \| n\boldsymbol\theta - \boldsymbol p - \boldsymbol\psi \|_\infty \leq\varepsilon \, .\]
\end{lemma}

We now arrive at the main result of the section:

\begin{theorem}
\label{dense}
Let $\boldsymbol{\lambda}\in\mathbb{T}^s$. Then the orbit $\lbrace \boldsymbol\lambda^n : n\in\mathbb{N} \rbrace$ is a dense subset of $T(\boldsymbol{\lambda})$.
\end{theorem}

\begin{proof}
  Let $\boldsymbol \theta\in\mathbb{R}^s$ be such that
  $\boldsymbol\lambda=e^{2\pi i\boldsymbol\theta}$ (with
  exponentiation operating coordinatewise). Notice that
  $\boldsymbol\lambda^{\boldsymbol v}=1$ if and only if $\boldsymbol v^T
  \boldsymbol\theta\in\mathbb{Z}$. If $\boldsymbol\mu\in
  T(\boldsymbol{\lambda})$, we can likewise define
  $\boldsymbol\psi\in\mathbb{R}^s$ to be such that
  $\boldsymbol\mu=e^{2\pi i \boldsymbol\psi}$. Then the premisses of
  Kronecker's lemma apply to $\boldsymbol \theta$ and $\boldsymbol
  \psi$. Thus, given $\varepsilon>0$, there exist a non-negative
  integer $n$ and $\boldsymbol p\in\mathbb{Z}^s$ such that $\|
  n\boldsymbol \theta -\boldsymbol p-\boldsymbol \psi \|_\infty
  \leq\varepsilon$. Whence
\[ \| \boldsymbol\lambda^n-\boldsymbol\mu \|_\infty = \| e^{2\pi i(n\boldsymbol\theta-\boldsymbol p)}-e^{2\pi i \boldsymbol\psi} \|_\infty \leq \]
\[ \| 2\pi (n\boldsymbol \theta -\boldsymbol p - \boldsymbol \psi) \|_\infty \leq 2\pi\varepsilon \, .\]
\end{proof}
%END SODA
