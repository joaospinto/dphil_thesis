%BEGIN SODA
\section{Groups of Multiplicative Relations}
\label{sec:mult}

This section introduces some concepts concerning groups of
multiplicative relations among algebraic numbers.  Here we will assume
some basic notions from algebraic number theory and the first-order
theory of reals.  We assume also a natural first-order interpretation of
the field of complex numbers in the ordered field of real numbers (in
which each complex number is encoded as a pair comprising its real and
imaginary parts).  Under this interpretation we refer to sets of
complex numbers as being semi-algebraic and first-order definable.
Details of the relevant notions can be found in the Appendix.

Let $\mathbb{T}=\lbrace z\in\Complex: \lvert z\rvert =1\rbrace$.  We
define the \emph{$s$-dimensional torus} to be $\mathbb{T}^s$,
considered as a group under componentwise multiplication.

Given a tuple of algebraic numbers
$\myvector\lambda=(\lambda_1,\ldots,\lambda_s)$, in this section we
consider how to effectively represent the \emph{orbit} $\{
\myvector\lambda^n : n \in \mathbb{N}\}$.  More
precisely, we will give an algebraic representation of the topological
closure of that orbit in $\mathbb{T}^s$.

The \emph{group of multiplicative relations} of
$\myvector{\lambda}$, which is an additive subgroup of
$\mathbb{Z}^s$, is defined as
\begin{equation*}
L(\myvector{\lambda})=\lbrace \myvector{v}\in \mathbb{Z}^s : \myvector\lambda^{\myvector v}=1 \rbrace \, ,
\end{equation*}
where $\myvector\lambda^{\myvector v}$ is defined to be
$\lambda_1^{v_1}\cdots\lambda_s^{v_s}$ for $\myvector{v}\in \mathbb{Z}^s$, that is, exponentiation acts
coordinatewise.

Since $\mathbb{Z}^s$ is a free abelian group, its subgroups are also
free.  In particular, $L(\myvector\lambda)$ has a finite basis. The
following powerful theorem of Masser~\cite{Mas88} gives bounds on the
magnitude of the components of such a basis.

\begin{theorem}[Masser]
\label{masser}
The free abelian group $L(\myvector{\lambda})$ has a basis $\myvector{v}_1,\cdots,\myvector{v}_l\in\mathbb{Z}^s$ for which
\[ \max\limits_{1\leq i\leq l,1\leq j\leq s} \lvert v_{i,j} \rvert \leq (D\log H)^{O(s^2)} \]
where $H$ and $D$ bound respectively the heights and degrees of all the $\lambda_i$.
\end{theorem}
Membership of a tuple $\myvector{v}\in \mathbb{Z}^s$ in
$L(\myvector{\lambda})$ can be computed in polynomial space, using a
decision procedure for the existential theory of the reals.  In
combination with \cref{masser}, it follows that we can compute
a basis for $L(\myvector{\lambda})$ in polynomial space by
brute-force search.

Corresponding to $L(\myvector{\lambda})$, we consider the following
multiplicative subgroup of $\mathbb{T}^s$:
\begin{equation*}
T(\myvector{\lambda})=\lbrace \myvector \mu\in\mathbb{T}^s : \forall \myvector v\in L(\myvector\lambda),\,\myvector\mu^{\myvector v}=1\rbrace \, .
\end{equation*}
If $V$ is a basis of $L(\myvector{\lambda})$ then we can
equivalently characterise $T(\myvector{\lambda})$ as $\{
\myvector{\mu} \in \mathbb{T}^s: \forall \myvector{v}\in
V,\,\,\myvector\mu^{\myvector v}=1\}$.  Crucially, this finitary
characterisation allows us to represent $T(\myvector\lambda)$ as a
semi-algebraic set.

We will use the following classical lemma of Kronecker on simultaneous
Diophantine approximation, in order to show that the orbit $\lbrace
\myvector\lambda^n : n\in\mathbb{N} \rbrace$ is a dense subset of
$T(\myvector{\lambda})$.

\begin{lemma}
  Let $\myvector \theta,\myvector \psi\in\Reals^s$. Suppose that for all $\myvector v\in\mathbb{Z}^s$, if
  $\myvector v^T\myvector \theta\in\mathbb{Z}$ then also
  $\myvector v^T\myvector\psi\in\mathbb{Z}$, i.e., all integer
  relations among the coordinates of $\myvector \theta$ also hold
  among those of $\myvector\psi$ (modulo $\mathbb{Z}$). Then, for
  each $\varepsilon>0$, there exist $\myvector p\in\mathbb{Z}^s$ and
  a non-negative integer $n$ such that
\[ \| n\myvector\theta - \myvector p - \myvector\psi \|_\infty \leq\varepsilon \, .\]
\end{lemma}

We now arrive at the main result of the section:

\begin{theorem}
\label{dense}
Let $\myvector{\lambda}\in\mathbb{T}^s$. Then the orbit $\lbrace \myvector\lambda^n : n\in\mathbb{N} \rbrace$ is a dense subset of $T(\myvector{\lambda})$.
\end{theorem}

\begin{proof}
  Let $\myvector \theta\in\Reals^s$ be such that
  $\myvector\lambda=e^{2\pi i\myvector\theta}$ (with
  exponentiation operating coordinatewise). Notice that
  $\myvector\lambda^{\myvector v}=1$ if and only if $\myvector v^T
  \myvector\theta\in\mathbb{Z}$. If $\myvector\mu\in
  T(\myvector{\lambda})$, we can likewise define
  $\myvector\psi\in\Reals^s$ to be such that
  $\myvector\mu=e^{2\pi i \myvector\psi}$. Then the premisses of
  Kronecker's lemma apply to $\myvector \theta$ and $\myvector
  \psi$. Thus, given $\varepsilon>0$, there exist a non-negative
  integer $n$ and $\myvector p\in\mathbb{Z}^s$ such that $\|
  n\myvector \theta -\myvector p-\myvector \psi \|_\infty
  \leq\varepsilon$. Whence
\[ \| \myvector\lambda^n-\myvector\mu \|_\infty = \| e^{2\pi i(n\myvector\theta-\myvector p)}-e^{2\pi i \myvector\psi} \|_\infty \leq \]
\[ \| 2\pi (n\myvector \theta -\myvector p - \myvector \psi) \|_\infty \leq 2\pi\varepsilon \, .\]
\end{proof}
%END SODA

%BEGIN HSCC
Given a vector $\myvector{\lambda} \in \Algebraics^{m}$, its \emph{group of multiplicative relations} is defined as
\begin{align*}
L(\myvector{\lambda}) = \lbrace \myvector{v} \in \mathbb{Z}^{m} : \myvector{\lambda}^{\myvector{v}} = 1 \rbrace .
\end{align*}

Moreover, letting $\log$ represent a fixed branch of the complex logarithm function, note that $\log(\alpha_{1}), \ldots, \log(\alpha_{m})$ are linearly independent over $\Rationals$ if and only if
\begin{align*}
L(\alpha_{1}, \ldots, \alpha_{m}) = \lbrace \myvector{0} \rbrace .
\end{align*}

Being a subgroup of the free finitely generated abelian group $\mathbb{Z}^{m}$, the group $L(\myvector{\lambda})$ is also free and admits a finite basis.

The following theorem, due to David Masser, allows us to effectively determine $L(\myvector{\lambda})$, and in particular decide whether it is equal to $\lbrace \myvector{0} \rbrace$. This result can be found in \cite{Masser}.

\begin{theorem}[Masser]
The free abelian group $L(\myvector{\lambda})$ has a basis $\myvector{v}_{1}, \ldots, \myvector{v}_{l} \in \mathbb{Z}^{m}$ for which
\begin{align*}
\max\limits_{1 \leq i \leq l, 1 \leq j \leq m} \lvert v_{i,j} \rvert \leq (D \log H)^{O(m^{2})}
\end{align*}
where $H$ and $D$ bound respectively the heights and degrees of all the $\lambda_{i}$.
\end{theorem}
%END HSCC
