%BEGIN SODA
\subsection{Groups of Multiplicative Relations}
\label{sec:mult}

This section introduces some concepts concerning groups of multiplicative relations among algebraic numbers. We assume a natural first-order interpretation of the field of complex numbers in the ordered field of real numbers (in which each complex number is encoded as a pair comprising its real and imaginary parts). Under this interpretation we refer to sets of complex numbers as being semi-algebraic and first-order definable.

Let $\mathbb{T}=\lbrace z \in \Complex: \lvert z \rvert = 1 \rbrace$. We define the \emph{$s$-dimensional torus} to be $\mathbb{T}^s$, considered as a group under componentwise multiplication. Then the function $x \mapsto \exp(2 \pi i x)$ is a homomorphism from the additive group of real numbers to $\mathbb{T}$, with kernel the subgroup of integers. By abuse of notation, we will also allow $\exp$ to be applied component-wise to a vector of reals.

Given a tuple of algebraic numbers $\myvector{\lambda} = (\lambda_{1}, \ldots, \lambda_{s}) \in \mathbb{T}^{s}$, we consider how to effectively represent the \emph{orbit}
\begin{equation*}
\lbrace \myvector{\lambda}^{n} : n \in \Naturals \rbrace.
\end{equation*}
More precisely, we will give an algebraic representation of the topological closure of that orbit in $\mathbb{T}^s$.

The \emph{group of multiplicative relations} of $\myvector{\lambda}$, which is an additive subgroup of $\Integers^s$, is defined as
\begin{equation*}
L(\myvector{\lambda})=\lbrace \myvector{v}\in \Integers^s : \myvector\lambda^{\myvector v}=1 \rbrace \, ,
\end{equation*}
where $\myvector\lambda^{\myvector v}$ is defined to be $\lambda_1^{v_1}\cdots\lambda_s^{v_s}$ for $\myvector{v}\in \Integers^s$, that is, exponentiation acts coordinatewise.

Since $\Integers^s$ is a free abelian group, its subgroups are also
free.  In particular, $L(\myvector\lambda)$ has a finite basis. The
following powerful theorem of Masser~\cite{Mas88} gives bounds on the
magnitude of the components of such a basis.

Note that $\log(\alpha_{1}), \ldots, \log(\alpha_{m})$ are linearly independent over $\Rationals$ if and only if
\begin{align*}
L(\alpha_{1}, \ldots, \alpha_{m}) = \lbrace \myvector{0} \rbrace .
\end{align*}
Together with \cref{thm:Baker}, Masser's theorem allows us to eliminate all algebraic relations in the description of linear forms in logarithms of algebraic numbers, and therefore also to test whether a linear form in logarithms of algebraic numbers is zero.

\begin{theorem}[Masser]
\label{thm:masser}
The free abelian group $L(\myvector{\lambda})$ has a basis $\myvector{v}_1, \ldots, \myvector{v}_{l} \in \Integers^{s}$ for which
\[ \max\limits_{1\leq i\leq l,1\leq j\leq s} \lvert v_{i,j} \rvert \leq (D\log H)^{O(s^2)} \]
where $H$ and $D$ bound respectively the heights and degrees of all the $\lambda_{i}$.
\end{theorem}
Membership of a tuple $\myvector{v}\in \Integers^{s}$ in $L(\myvector{\lambda})$ can be computed in polynomial space, using a decision procedure for the existential theory of the reals. In combination with \cref{thm:masser}, it follows that we can compute a basis for $L(\myvector{\lambda})$ in polynomial space by brute-force search (due to Savitch's theorem).

Corresponding to $L(\myvector{\lambda})$, we consider the following
multiplicative subgroup of $\mathbb{T}^{s}$:
\begin{equation*}
T(\myvector{\lambda})=\lbrace \myvector{\mu} \in \mathbb{T}^{s} : \forall \myvector{v} \in L(\myvector{\lambda}), \, \myvector{\mu}^{\myvector{v}} = 1 \rbrace \, .
\end{equation*}
If $\mathcal{B}$ is a basis of $L(\myvector{\lambda})$ then we can
equivalently characterise $T(\myvector{\lambda})$ as
\begin{equation*}
\lbrace \myvector{\mu} \in \mathbb{T}^{s}: \forall \myvector{v} \in \mathcal{B}, \, \, \myvector{\mu}^{\myvector{v}} = 1 \rbrace \, .
\end{equation*}
Crucially, this finitary characterisation allows us to represent $T(\myvector\lambda)$ as a semi-algebraic set.

We will use the \cref{thm:Kronecker} in order to show that the orbit $\lbrace
\myvector\lambda^n : n\in\Naturals \rbrace$ is a dense subset of
$T(\myvector{\lambda})$.

\begin{theorem}
\label{dense}
Let $\myvector{\lambda}\in\mathbb{T}^s$. Then the orbit $\lbrace \myvector\lambda^n : n\in\Naturals \rbrace$ is a dense subset of $T(\myvector{\lambda})$.
\end{theorem}

\begin{proof}
Let $\myvector{\theta} \in \Reals^{s}$ be such that $\myvector{\lambda} = \exp(2 \pi i \myvector{\theta})$ (with exponentiation operating coordinatewise). Notice that $\myvector{\lambda}^{\myvector{v}} = 1$ if and only if $\myvector{v}^{T} \myvector{\theta} \in \Integers$.
If $\myvector{\mu} \in T(\myvector{\lambda})$, we can likewise define $\myvector{\psi} \in \Reals^{s}$ to be such that $\myvector{\mu}=\exp(2 \pi i \myvector{\psi})$. Then the premisses of \cref{thm:Kronecker} apply to $\myvector{\theta}$ and $\myvector{\psi}$.
Thus, given $\varepsilon>0$, there exist a non-negative integer $n$ and a vector $\myvector{p} \in \Integers^{s}$ such that $\dist(n \myvector{\theta} - \myvector{\psi}, \myvector{p}) \leq \varepsilon$.
Whence
\begin{equation*}
  | \myvector{\lambda}^{n}-\myvector{\mu} \|_{\infty} = \| \exp( 2 \pi i (n\myvector{\theta} - \myvector{p})) - \exp(2 \pi i \myvector{\psi}) \|_{\infty} \leq \| 2 \pi (n\myvector{\theta} -\myvector{p} - \myvector{\psi}) \|_\infty \leq 2 \pi \varepsilon \, .
\end{equation*}
Given that $\varepsilon$ was arbitrary, it follows that $\lbrace \myvector{\lambda}^{n} : n \in \Naturals \rbrace$ is dense in $T(\myvector{\lambda})$.
\end{proof}

We will also need the following simple corollary:
\begin{corollary}
  Let $\myvector{\theta} \in \Reals^{s}$ be such that $A(\myvector{\theta}) = \lbrace \myvector{0} \rbrace$. Then $\lbrace \exp(2 \pi i n \myvector{\theta}) : n \in \Naturals \rbrace$ is a dense subset of $\mathbb{T}^s$.
\label{corl:kronecker}
\end{corollary}
\begin{proof}
This result follows from the fact that
\begin{equation*}
  L(\exp(2 \pi i \myvector{\theta})) = A(\myvector{\theta}) = \lbrace \myvector{0} \rbrace \, .
\end{equation*}
Therefore, $T(\myvector{\lambda}) = \mathbb{T}^{s}$, which establishes the result.
\end{proof}
