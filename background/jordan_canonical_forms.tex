\subsection{Jordan Canonical Forms}

Let $A \in \Rationals^{d \times d}$ be a square matrix with rational
entries.  The \textbf{minimal polynomial} of $A$ is the unique monic
polynomial $m(x) \in \Rationals[x]$ of least degree such that
$m(A)=0$.  By the Cayley-Hamilton Theorem the degree of $m$ is at most
the dimension of $A$. The set $\sigma(A)$ of eigenvalues is the set of
roots of $m$.  The \textbf{index} of an eigenvalue $\lambda$, denoted
by $\nu(\lambda)$, is defined as its multiplicity as a root of $m$. We
use $\nu(A)$ to denote $\max_{\lambda\in\sigma(A)} \nu(\lambda)$: the
maximum index over all eigenvalues of $A$.  Given an eigenvalue
$\lambda \in \sigma(A)$, we say that $\myvector{v} \in \Complex^{d}$
is a \textbf{generalised eigenvector} of $A$ if
$\myvector{v}\in \ker(A-\lambda I)^{k}$, for some $k\in\Naturals$.

We denote the subspace of $\Complex^{d}$ spanned by the set of
generalised eigenvectors associated with some eigenvalue $\lambda$ by
$\mathcal{V}_{\lambda}$. We denote the subspace of $\Complex^{d}$
spanned by the set of generalised eigenvectors associated with some
real eigenvalue by $\mathcal{V}^{r}$.  We likewise denote the subspace
of $\Complex^{d}$ spanned by the set of generalised eigenvectors
associated to eigenvalues with non-zero imaginary part by
$\mathcal{V}^{c}$.

It is well known that each vector $\myvector{v}\in\Complex^{d}$
can be written uniquely as
\begin{equation*}
\myvector{v}=\displaystyle{
  \sum\limits_{\lambda\in\sigma(A)}\myvector{v}_{\lambda}},
\end{equation*}
where $\myvector{v}_{\lambda}\in\mathcal{V}_{\lambda}$.
It follows that $\myvector{v}$ can also be uniquely written as
$\myvector{v}=\myvector{v}^{r}+\myvector{v}^{c}$, where
$\myvector{v}^{r} \in\mathcal{V}^{r}$ and
$\myvector{v}^{c} \in\mathcal{V}^{c}$.

We will need the following result:
\begin{proposition}
\label{conj-relation}
Suppose that $\boldsymbol{v}\in\mathbb{R}^{d}$ and that $\boldsymbol{v}=\sum\limits_{\lambda\in\sigma(A)} \boldsymbol{v}_{\lambda}$, where $\boldsymbol{v}_{\lambda} \in\mathcal{V}_{\lambda}$. Then $\boldsymbol{v}_{\overline{\lambda}}$ and $\boldsymbol{v}_{\lambda}$ are component-wise complex conjugates.
\end{proposition}

\begin{proof}
  Note that $\boldsymbol{v}_{\lambda}\in \ker(A-\lambda I)^{k}$
  implies that
  $\overline{\boldsymbol{v}_{\lambda}} \in \ker(A-\overline{\lambda}
  I)^{k}$
  and hence that
  $\overline{ \boldsymbol{v}_{ \overline{\lambda}}} \in
  \mathcal{V}_{\lambda}$.  The result follows from the fact that
\begin{align*}
\boldsymbol{0}=\boldsymbol{v}-\overline{\boldsymbol{v}}=\sum\limits_{\lambda\in \sigma(A)}(\boldsymbol{v}_{\lambda}-\overline{ \boldsymbol{v}_{ \overline{\lambda}}})
\end{align*}
and from uniqueness of the above decomposition.
%
%For each $\lambda\in\sigma(A)$, let $f_{\lambda}$ be the function mapping each $\boldsymbol{v}$ to the corresponding $\boldsymbol{v}_{\lambda}$. Fix a basis $\mathcal{B}$ of generalised eigenvectors of $A$, with the property
%\begin{align*}
%\boldsymbol{x}\in\mathcal{B}\Rightarrow \overline{\boldsymbol{x}}\in\mathcal{B}
%\end{align*}
%For each $\boldsymbol{x}\in\mathcal{B}$, we define a linear functional $g_{\boldsymbol{x}}$ so that
%\begin{align*}
%g_{\boldsymbol{x}}(\sum\limits_{\boldsymbol{y}\in\mathcal{B}} \alpha_{\boldsymbol{y}}\boldsymbol{y})=\alpha_{\boldsymbol{x}}
%\end{align*}
%Then $\forall \boldsymbol{u}\in\mathbb{C}^{d}, \overline{g_{\overline{\boldsymbol{x}}}(\overline{\boldsymbol{u}})}=g_{\boldsymbol{x}}(\boldsymbol{u})$, as this equality certainly holds when $\boldsymbol{u}\in\mathcal{B}$, and by linearity of both sides. The result follows from the fact that
%\begin{align*}
%\forall \boldsymbol{u}\in\mathbb{R}^{d}, f_{\lambda}(\boldsymbol{u})= \sum\limits_{\boldsymbol{x}\in\mathcal{B}:\boldsymbol{x}\in\mathcal{V}_{\lambda}} g_{\boldsymbol{x}}(\boldsymbol{u})\boldsymbol{x}
%\end{align*}
%
\end{proof}

Moreover, we can write any matrix $A$ as $A=Q^{-1}JQ$ for some
invertible matrix $Q$ and block diagonal Jordan matrix
$J=\diag{J_{1},\ldots,J_{N}}$, with each block $J_{i}$ having the
following form:

\begin{equation*}
\begin{pmatrix}
\lambda	&&	1		&&	0		&&	\cdots	&&	0		\\
0		&&	\lambda	&&	1		&&	\cdots	&&	0		\\
\vdots	&&	\vdots	&&	\vdots	&&	\ddots	&&	\vdots	\\
0		&&	0		&&	0		&&	\cdots	&&	1		\\
0		&&	0		&&	0		&&	\cdots	&&	\lambda	\\
\end{pmatrix}
\end{equation*}

Given a rational matrix $A$, its Jordan Normal Form $A=Q^{-1}JQ$ can be
computed in polynomial time, as shown in \cite{Cai94}.

Note that each vector $\myvector{v}$ appearing as a column of the
matrix $Q^{-1}$ is a generalised eigenvector. We also note that the
index $\nu(\lambda)$ of some eigenvalue $\lambda$ corresponds to the
dimension of the largest Jordan block associated with it.

One can obtain a closed-form expression for powers of block diagonal
Jordan matrices, and use this to get a closed-form expression for
the powers of any matrix $A$. In fact, if $J_{i}$ is a
$k\times k$ Jordan block associated with some eigenvalue $\lambda$,
then
%\noindent
\begin{equation}
\label{eq:jordan_powers}
J_{i}^{n}=\begin{pmatrix}
\lambda^{n}	&&	n\lambda^{n-1}	&&	{n\choose 2}\lambda^{n-1}	&&
\cdots		&&	{n\choose k-1}\lambda^{n-k+1}				\\
0			&&	\lambda^{n}		&&	n\lambda^{n-1}				&&
\cdots		&&	{n\choose k-2}\lambda^{n-k+2}				\\
\vdots	&&	\vdots	&&	\vdots	&&	\ddots	&&	\vdots			\\
0		&&	0		&&	0		&&	\cdots	&&	n\lambda^{n-1}	\\
0		&&	0		&&	0		&&	\cdots	&&	\lambda^{n}		\\
\end{pmatrix}
\end{equation}
where ${n\choose j}$ is defined to be $0$ when $n<j$.
