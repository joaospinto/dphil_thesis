%BEGIN HSCC
\subsection{Jordan Canonical Forms}

Let $A \in \mathbb{Q}^{d \times d}$ be a square matrix with rational
entries.  The \textbf{minimal polynomial} of $A$ is the unique monic
polynomial $m(x) \in \mathbb{Q}[x]$ of least degree such that
$m(A)=0$.  By the Cayley-Hamilton Theorem the degree of $m$ is at most
the dimension of $A$. The set $\sigma(A)$ of eigenvalues is the set of
roots of $m$.  The \textbf{index} of an eigenvalue $\lambda$, denoted
by $\nu(\lambda)$, is defined as its multiplicity as a root of $m$. We
use $\nu(A)$ to denote $\max_{\lambda\in\sigma(A)} \nu(\lambda)$: the
maximum index over all eigenvalues of $A$.  Given an eigenvalue
$\lambda \in \sigma(A)$, we say that $\boldsymbol{v} \in \mathbb{C}^d$
is a \textbf{generalised eigenvector} of $A$ if
$\boldsymbol{v}\in \ker(A-\lambda I)^{k}$, for some $k\in\mathbb{N}$.

We denote the subspace of $\mathbb{C}^d$ spanned by the set of
generalised eigenvectors associated with some eigenvalue $\lambda$ by
$\mathcal{V}_{\lambda}$. We denote the subspace of $\mathbb{C}^d$
spanned by the set of generalised eigenvectors associated with some
real eigenvalue by $\mathcal{V}^{r}$.  We likewise denote the subspace
of $\mathbb{C}^d$ spanned by the set of generalised eigenvectors
associated to eigenvalues with non-zero imaginary part by
$\mathcal{V}^{c}$.

It is well known that each vector $\boldsymbol{v}\in\mathbb{C}^{d}$
can be written uniquely as
$\boldsymbol{v}=\displaystyle{
  \sum\limits_{\lambda\in\sigma(A)}\boldsymbol{v}_{\lambda}}$,
where $\boldsymbol{v}_{\lambda}\in\mathcal{V}_{\lambda}$.
It follows that $\boldsymbol{v}$ can also be uniquely written as
$\boldsymbol{v}=\boldsymbol{v}^{r}+\boldsymbol{v}^{c}$, where
$\boldsymbol{v}^{r} \in\mathcal{V}^{r}$ and
$\boldsymbol{v}^{c} \in\mathcal{V}^{c}$.

Moreover, we can write any matrix $A$ as $A=Q^{-1}JQ$ for some
invertible matrix $Q$ and block diagonal Jordan matrix
$J=diag(J_{1},\ldots,J_{N})$, with each block $J_{i}$ having the
following form:

\begin{equation*}
\begin{pmatrix}
\lambda	&&	1		&&	0		&&	\cdots	&&	0		\\
0		&&	\lambda	&&	1		&&	\cdots	&&	0		\\
\vdots	&&	\vdots	&&	\vdots	&&	\ddots	&&	\vdots	\\
0		&&	0		&&	0		&&	\cdots	&&	1		\\
0		&&	0		&&	0		&&	\cdots	&&	\lambda	\\
\end{pmatrix}
\end{equation*}

Given a rational matrix $A$, its Jordan Normal Form $A=Q^{-1}JQ$ can be
computed in polynomial time, as shown in \cite{Cai94}.

Note that each vector $\boldsymbol{v}$ appearing as a column of the
matrix $Q^{-1}$ is a generalised eigenvector. We also note that the
index $\nu(\lambda)$ of some eigenvalue $\lambda$ corresponds to the
dimension of the largest Jordan block associated with it.

One can obtain a closed-form expression for powers of block diagonal
Jordan matrices, and use this to get a closed-form expression for
exponential block diagonal Jordan matrices. In fact, if $J_{i}$ is a
$k\times k$ Jordan block associated with some eigenvalue $\lambda$,
then
\noindent
%\resizebox{\linewidth}{!}{
\begin{equation*}
J_{i}^{n}=\begin{pmatrix}
\lambda^{n}	&&	n\lambda^{n-1}	&&	{n\choose 2}\lambda^{n-1}	&&
\cdots		&&	{n\choose k-1}\lambda^{n-k+1}				\\
0			&&	\lambda^{n}		&&	n\lambda^{n-1}				&&
\cdots		&&	{n\choose k-2}\lambda^{n-k+2}				\\
\vdots	&&	\vdots	&&	\vdots	&&	\ddots	&&	\vdots			\\
0		&&	0		&&	0		&&	\cdots	&&	n\lambda^{n-1}	\\
0		&&	0		&&	0		&&	\cdots	&&	\lambda^{n}		\\
\end{pmatrix}
\end{equation*}
%}

\begin{equation*}
\exp(J_{i}t)=\exp(\lambda t) \begin{pmatrix}
1		&&	t		&&	\cdots	&&	\frac{t^{k-1}}{(k-1)!}	\\
0		&&	1		&&	\cdots	&&	\frac{t^{k-2}}{(k-2)!}	\\
\vdots	&&	\vdots	&&	\ddots	&&	\vdots						\\
0		&&	0		&&	\cdots	&&	t							\\
0		&&	0		&&	\cdots	&&	1							\\
\end{pmatrix}
\end{equation*}

%SINGLE COLUMN
%\begin{align*}
%J_{i}^{n}&=\begin{pmatrix}
%\lambda^{n}	&&	n\lambda^{n-1}	&&	{n\choose 2}\lambda^{n-1}	&&
%\cdots		&&	{n\choose k-1}\lambda^{n-k+1}				\\
%0			&&	\lambda^{n}		&&	n\lambda^{n-1}				&&
%\cdots		&&	{n\choose k-2}\lambda^{n-k+2}				\\
%\vdots	&&	\vdots	&&	\vdots	&&	\ddots	&&	\vdots			\\
%0		&&	0		&&	0		&&	\cdots	&&	n\lambda^{n-1}	\\
%0		&&	0		&&	0		&&	\cdots	&&	\lambda^{n}		\\
%\end{pmatrix}
%\quad\mbox{and} \\
%\exp(J_{i}t)&=\exp(\lambda t) \begin{pmatrix}
%1		&&	t		&&	\cdots	&&	\frac{t^{k-1}}{(k-1)!}	\\
%0		&&	1		&&	\cdots	&&	\frac{t^{k-2}}{(k-2)!}	\\
%\vdots	&&	\vdots	&&	\ddots	&&	\vdots						\\
%0		&&	0		&&	\cdots	&&	t							\\
%0		&&	0		&&	\cdots	&&	1							\\
%\end{pmatrix}
%\end{align*}

In the above, ${n\choose j}$ is defined to be $0$ when $n<j$.

\begin{proposition}
  Let $\boldsymbol{v}$ lie in the generalised eigenspace
  $\mathcal{V}_{\lambda}$ for some $\lambda \in \sigma(A)$.  Then
  $\boldsymbol{b}^{T}\exp(At)\boldsymbol{v}$ is a linear combination
  of terms of the form $t^{n}\exp(\lambda t)$.
\label{prop:linear}
\end{proposition}
% Every expression of the form
%  $\boldsymbol{b}^{T}\exp(At)\boldsymbol{x}_{0}$ is a linear
%  combination of terms of the form $t^{n}\exp(\lambda t)$, where
%  $\lambda$ is an eigenvalue of $A$.

\begin{proof}
  Note that, if $A=Q^{-1}JQ$ and $J=diag(J_{1},\ldots,J_{N})$ is a
  block diagonal Jordan matrix, then $\exp(At)=Q^{-1}\exp(Jt)Q$ and
  $\exp(Jt)=diag(\exp(J_{1}t),\ldots,\exp(J_{N}t))$.
The result follows by observing that $Q\boldsymbol{v}$ is zero in every component
other than those pertaining the block corresponding to the eigenspace
$\mathcal{V}_{\lambda}$.
\end{proof}
%END HSCC
