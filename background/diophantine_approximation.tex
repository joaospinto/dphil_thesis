\subsection{Diophantine Approximation}

We will also need the following result, due to Leopold Kronecker, on simultaneous Diophantine approximation, which generalises Dirichlet's Approximation Theorem. We denote the \emph{group of additive relations} of $\myvector{v}$ by
\begin{align*}
A(\myvector{v}) = \lbrace \myvector{z} \in \Integers^{d} : \myvector{z} \cdot \myvector{v} \in \Integers \rbrace .
\end{align*}

Throughout this thesis, $\dist$ refers to the $l_{1}$ distance.

\begin{theorem}[Kronecker]
\label{thm:Kronecker}
Let $\myvector{\alpha}_{1}, \ldots, \myvector{\alpha_{k}} \in \Reals^{d}$ and $\myvector{\beta} \in \Reals^{d}$. The following are equivalent:
\begin{enumerate}
\item For any $\varepsilon > 0$, there exists $\myvector{n} \in \Naturals^{k}$ such that
\begin{align*}
\operatorname{dist}(\myvector{\beta} + \sum\limits_{i=1}^{k} n_{i} \myvector{\alpha}_{i}, \Integers^{d}) \leq \varepsilon .
\end{align*}
\item It holds that
\begin{align*}
\bigcap\limits_{i=1}^{k} A(\myvector{\alpha}_{i}) \subseteq A(\myvector{\beta}) .
\end{align*}
\end{enumerate}
\end{theorem}

A proof of this result can be found in~\cite{Cassels}.

Let $\mathbb{T}$ denote the group of complex numbers of modulus $1$, with multiplication as group operation. Then the function $x \mapsto \exp(2 \pi i x)$ is a homomorphism from the additive group of real numbers to $\mathbb{T}$, with kernel the subgroup of integers. By abuse of notation, we will also allow $\exp$ to be applied component-wise to a vector of reals.

We obtain the following simple corollary:
\begin{corollary}
  Let $\myvector{\theta} = (\theta_{1}, \ldots, \theta_{s}) \in \Reals^{s}$ be such that $1, \theta_1, \ldots, \theta_s$ are linearly independent over $\Rationals$.
  Then $\lbrace \exp(2 \pi i n \myvector{\theta}) : n \in \Naturals \rbrace$ is a dense subset of $\mathbb{T}^s$.
\label{corl:kronecker}
\end{corollary}
\begin{proof}
Since $\exp$ is surjective, an arbitrary element of $\mathbb{T}^{s}$ can be written in the form $\exp(2 \pi i \myvector{\psi})$ for some $\myvector{\psi} \in \Reals^{s}$. By hypothesis, we know that $A(\myvector{\theta}) = \lbrace \myvector{0} \rbrace$. Applying Kronecker's Theorem, we get that, for all $\varepsilon > 0$, there exists a positive integer $n$ such that
\begin{equation*}
  \dist(n \myvector{\theta} - \myvector{\psi}, \Integers^{s}) \leq \varepsilon .
\end{equation*}
By continuity of $\exp$ it follows that $\exp(2 \pi i \myvector{\psi})$ is a limit point of
\begin{equation*}
\lbrace \exp(2 \pi i n \myvector{\theta}) : n \in \Naturals \rbrace .
\end{equation*}
This establishes the result.
\end{proof}
