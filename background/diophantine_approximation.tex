\subsection{Diophantine Approximation}

We will also need the following result, due to Leopold Kronecker, on simultaneous Diophantine approximation, which generalises Dirichlet's Approximation Theorem. We denote the \emph{group of additive relations} of $\myvector{v}$ by
\begin{align*}
\cA(\myvector{v}) = \lbrace \myvector{z} \in \Integers^{d} : \myvector{z} \cdot \myvector{v} \in \Integers \rbrace .
\end{align*}

Throughout this thesis, $\dist$ refers to the $l_{1}$ distance.

\begin{theorem}[Kronecker]
\label{thm:Kronecker}
Let $\myvector{\alpha}_{1}, \ldots, \myvector{\alpha_{k}} \in \Reals^{d}$ and $\myvector{\beta} \in \Reals^{d}$. The following are equivalent:
\begin{enumerate}
\item For any $\varepsilon > 0$, there exists $\myvector{n} \in \Naturals^{k}$ such that
\begin{align*}
\operatorname{dist}(\myvector{\beta} + \sum\limits_{i=1}^{k} n_{i} \myvector{\alpha}_{i}, \Integers^{d}) \leq \varepsilon .
\end{align*}
\item It holds that
\begin{align*}
\bigcap\limits_{i=1}^{k} \cA(\myvector{\alpha}_{i}) \subseteq \cA(\myvector{\beta}) .
\end{align*}
\end{enumerate}
\end{theorem}

A proof of this result can be found in~\cite{Cassels}.
