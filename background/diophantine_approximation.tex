%BEGIN LICS

\section{Diophantine Approximation}

We will also need the following result, due to Leopold Kronecker, on simultaneous Diophantine approximation, which generalises Dirichlet's Approximation Theorem. We denote the \emph{group of additive relations} of $\boldsymbol{v}$ by
\begin{align*}
A(\boldsymbol{v}) = \lbrace \boldsymbol{z} \in \mathbb{Z}^{d} : \boldsymbol{z} \cdot \boldsymbol{v} \in \mathbb{Z} \rbrace .
\end{align*}

Throughout this paper, $\operatorname{dist}$ refers to the $l_{1}$ distance.

\begin{theorem}[Kronecker]
\label{Kronecker}
Let $\boldsymbol{\alpha}_{1}, \ldots, \boldsymbol{\alpha_{k}} \in \mathbb{R}^{d}$ and $\boldsymbol{\beta} \in \mathbb{R}^{d}$. The following are equivalent:
\begin{enumerate}
\item For any $\varepsilon > 0$, there exists $\boldsymbol{n} \in \mathbb{N}^{k}$ such that
\begin{align*}
\operatorname{dist}(\boldsymbol{\beta} + \sum\limits_{i=1}^{k} n_{i} \boldsymbol{\alpha}_{i}, \mathbb{Z}^{d}) \leq \varepsilon .
\end{align*}
\item It holds that
\begin{align*}
\bigcap\limits_{i=1}^{k} A(\boldsymbol{\alpha}_{i}) \subseteq A(\boldsymbol{\beta}) .
\end{align*}
\end{enumerate}
\end{theorem}

Many of these results, or slight variations thereof, can be found in \cite{HardyAndWright} and \cite{Cassels}.



%END LICS

%BEGIN HSCC
\subsection{Kronecker's Theorem}
Let $\mathbb{T}$ denote the group of complex numbers of modulus $1$,
with multiplication as group operation.  Then the function
$\phi:\mathbb{R} \rightarrow \mathbb{T}$ given by
$\phi(x)=\exp(2\pi i x)$ is a homomorphism from the additive
group of real numbers to $\mathbb{T}$, with kernel the subgroup of
integers.

Recall from~\cite{HardyAndWright} the following classical theorem of
Kronecker on simultaneous inhomogeneous Diophantine approximation.
\begin{theorem}[Kronecker]
  Let $\theta_1,\ldots,\theta_s$ be real numbers such that the set
  $\{\theta_1,\ldots,\theta_s,1\}$ is linearly independent over
  $\mathbb{Q}$.  Then for all $\psi_1,\ldots,\psi_s \in \mathbb{R}$
  and $\varepsilon > 0$, there exists a positive integer $n$
and integers $n_1,\ldots,n_s$ such that
\[ |n\theta_1 - \psi_1 - n_1| < \varepsilon, \ldots ,
   |n\theta_s - \psi_s - n_s| < \varepsilon \, .\]
\end{theorem}

We obtain the following simple corollary:
\begin{corollary}
  Let $\theta_1,\ldots,\theta_s$ be real numbers such that the set
  $\{ \theta_1,\ldots,\theta_s,1\}$ is linearly independent over
  $\mathbb{Q}$.  Then
\[ \{ (\phi(n\theta_1),\ldots,\phi(n\theta_s)) : n \in \mathbb{N} \} \]
is a dense subset of $\mathbb{T}^s$.
\label{corl:kronecker}
\end{corollary}
\begin{proof}
  Since $\phi$ is surjective, an arbitrary element of $\mathbb{T}^s$
  can be written in the form $(\phi(\psi_1),\ldots,\phi(\psi_s))$ for
  some real numbers $\psi_1,\ldots,\psi_s$.
Applying Kronecker's Theorem, we get that
for all $\varepsilon > 0$, there exists a positive integer $n$
and integers $n_1,\ldots,n_s$ such that
\[ |n\theta_1 - \psi_1 - n_1| < \varepsilon, \ldots , |n\theta_s
  - \psi_s - n_s| < \varepsilon \, .\] By continuity of $\phi$ it
follows that $(\phi(\psi_1),\ldots,\phi(\psi_s))$ is a limit point of
$ \{ (\phi(n\theta_1),\ldots,\phi(n\theta_s)) : n \in \mathbb{N} \}$.
This establishes the result.
\end{proof}
%END HSCC
