\label{sec:LRS}

A \emph{linear recurrence sequence} of order $k$ over $\Rationals$ is a sequence ${(u_{n})}_{n \in \Naturals} \subseteq \Rationals$ with the property that there exist coefficients $\alpha_{1}, \ldots, \alpha_{k} \in \Rationals$ such that, for all $n \geq k$,
\begin{equation*}
  u_{n} = \sum\limits_{i=1}^{k} \alpha_{i} u_{n-i}\, .
\end{equation*}

Equivalently, due to the Cayley-Hamilton theorem, linear recurrence sequences of order $k$ correspond to sequences of the form $u_{n} = \myvector{b}^{T} A^{n} \myvector{x}$. When the matrix $A$ is diagonalisable, the linear recurrence ${(u_{n})}_{n \in \Naturals}$ is said to be \emph{simple}. The \emph{characteristic roots} of $u$ correspond to the eigenvalues of $A$.

It is well-known that linear recurrence sequences correspond to sequences admitting a representation of the form
\begin{equation}
  \label{eq:poly-exp}
  u_{n} = \sum\limits_{i=1}^{k} p_{i}(n) \lambda_{i}^{n}
\end{equation}
where $\lambda_{1}, \ldots, \lambda_{k}$ are the characteristic roots of $u$, as can easily be shown by considering the Jordan canonical form of the matricial form of this recurrence.

Moreover, the sequences admitting a representation of the form described in \cref{eq:poly-exp}, where the polynomials $p_{i}$ are constant, correspond to simple linear recurrence sequences.

Finally, if ${(u_{n})}_{n \in \Naturals} \subseteq \Reals$, then the polynomials $p_{i}$ and constants $\lambda_{i}$ in \cref{eq:poly-exp} satisfy the relation
\begin{equation}
  \label{eq:real_property}
  \lambda_{i} = \overline{\lambda_{j}} \Rightarrow p_{i} = \overline{p_{j}} \, .
\end{equation}

As an example, the Fibonacci sequence $0, 1, 1, 2, 3, 5, 8, 13, \ldots$ is a simple linear recurrence sequence of order $2$, with recursion rule $f_{n+1} = f_{n} + f_{n-1}$. It can also be represented in the form
\begin{equation*}
  f_{n} = \frac{1}{\sqrt{5}} {\left(\frac{1+\sqrt{5}}{2} \right)}^{n} - \frac{1}{\sqrt{5}} {\left( \frac{1-\sqrt{5}}{2} \right)}^{n} \, .
\end{equation*}

For more information on linear recurrence sequences, see~\cite{BOOK}.

\subsection{Lower-bounding simple linear recurrence sequences}
\label{sec:s-units-app}

We are interested in lower-bounding expressions of the form
\begin{equation}
\label{eq:sum}
u_{n}=\sum\limits_{j=1}^{s}\alpha_{j}\lambda_j^{n}
\end{equation}
satisfying the property of \cref{eq:real_property}, where $\alpha_{1}, \lambda_{1}, \ldots, \alpha_{s}, \lambda_{s}$ are algebraic integers and $\lambda_{1},\ldots,\lambda_{s}$ have the same absolute value $\rho$.

Note that eigenvalues of integer-valued matrices are always algebraic integers, and that all algebraic numbers admit an integer multiple that is an algebraic integer. Therefore, assuming the $\alpha_{j}$ to be algebraic integers will only worsen the bound we derive in this subsection by a constant factor, which will be irrelevant.

In order to make use of~\cref{thm:s-units}, it is important to understand the set
\begin{equation}
\lbrace n\in\Naturals: \exists I\subseteq \lbrace 1,\ldots,s\rbrace, \sum\limits_{j\in I}\alpha_j\lambda_j^n=0\rbrace
\label{eq:genzeros}
\end{equation}

The following well-known theorem characterises the set of zeros of linear recurrence sequences. In particular, it gives us a sufficient condition for guaranteeing that the set of zeros of a non-identically zero linear recurrence sequence is finite. Namely, it suffices that the sequence is \emph{non-degenerate}, that is, that no ratio of two of its characteristic roots is a root of unity.

\begin{theorem}[Skolem-Mahler-Lech]
Let ${(u_{n})}_{n \in \Naturals}$ be a linear recurrence sequence. The set $\lbrace n\in\Naturals: u_n=0\rbrace$ is a union of a finite set and finitely many arithmetic progressions. Moreover, if $u_n$ is non-degenerate, this set is actually finite.
\end{theorem}

It follows from the Skolem-Mahler-Lech theorem that if $u_n$ is non-degenerate  then~\eqref{eq:genzeros} must be finite, assuming without loss of generality that
\begin{equation*}
\sum\limits_{j\in I}\alpha_j\lambda_j^n
\end{equation*}
is never eventually zero.

We can now apply~\cref{thm:s-units} to get a lower bound on~\eqref{eq:sum} that holds for all but finitely many $n$, by letting $K$ be the splitting field of the characteristic polynomial of $u_n$, $S$ be the set of prime ideals of the ring of integers of $K$ that appear in the factorisation of each of the algebraic integers $\alpha_j$ and $\lambda_j$, and $x_j=\alpha_j\lambda_j^n$ for each $j$, making~\eqref{eq:sum} a sum of $S$-units.

In the notation of the theorem, we have $Y=\Omega(\rho^n)$. If $\Lambda$ is an upper bound on the absolute value of the Galois conjugates of each $\lambda_j$ (that is, each $\sigma_i(\lambda_j)$), then $Z=O(\Lambda^n)$. Thus, for any $\varepsilon>0$, we know that
\begin{align*}
\sum\limits_{j=1}^s\alpha_j\lambda_j^n &=\Omega(YZ^{-\varepsilon})=
\Omega\left(\rho^n\Lambda^{-n\varepsilon}\right)
\end{align*}
Finally, we note that by picking $\varepsilon$ to be sufficiently small we can get $\rho\Lambda^{-\varepsilon}$ arbitrarily close to $\rho$.

Therefore, for any $\eta < \rho$, it is true for sufficiently large $n$ that
\begin{equation*}
  \bigg\lvert \sum \limits_{j=1}^{s} \alpha_{j} \lambda_{j}^{n} \bigg\lvert = \Omega(\eta^{n}) \, .
\end{equation*}
