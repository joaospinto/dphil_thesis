\subsection{Fourier-Motzkin Elimination}

Fourier-Motzkin elimination is a simple method for solving systems of
inequalities. Historically, it was the first algorithm used in solving
linear programming, before more efficient procedures such as the
simplex algorithm were discovered. The procedure consists in isolating one
variable at a time and matching all its lower and upper bounds. Note
that this method preserves the set of solutions on the remaining
variables, so a solution of the reduced system can always be extended
to a solution of the original one.

\begin{theorem}
\label{thm:fme}
  It is decidable whether a
  given convex polytope
  $\mathcal{P} = \lbrace \myvector{x} \in \Reals^{n} : \pi
  A\myvector{x} < \myvector{b} \rbrace$,
  where the entries of $A$ are all real algebraic numbers and
  those of $\myvector{b}$ are real linear forms in logarithms of
  algebraic numbers, is empty.  Moreover, if $\mathcal{P}$ is
  non-empty one can effectively find a rational vector
  $\myvector{q} \in \mathcal{P}$.
\end{theorem}

\begin{proof}
This is done by using Fourier-Motzkin elimination, isolating each term $\pi x_{i}$, instead of just isolating the variable $x_{i}$. Note that the coefficients of the terms $\pi x_{i}$ will always be algebraic, and the remaining terms will always be linear forms in logarithms of algebraic numbers, which are closed under multiplication by algebraic numbers, and which can be effectively compared by using \cref{thm:Baker} and \cref{thm:masser}.
\end{proof}
