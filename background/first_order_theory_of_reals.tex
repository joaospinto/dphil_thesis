%BEGIN SODA
\section{First-Order Theory of Reals}

Let $\boldsymbol{x}=(x_1,\ldots,x_m)$ be a list of $m$ real-valued
variables, and let $\sigma(\boldsymbol{x})$ be a Boolean combination
of atomic predicates of the form $g(\boldsymbol{x})\sim 0$, where each
$g(\boldsymbol{x})$ is a polynomial with integer coefficients in the
variables $\boldsymbol{x}$, and $\sim$ is either $>$ or $=$. Tarski
has famously shown that we can decide the truth over the field
$\mathbb{R}$ of sentences of the form $\phi=Q_1 x_1 \cdots Q_m x_n
\sigma(\boldsymbol{x})$, where $Q_i$ is either $\exists$ or
$\forall$. He did so by showing that this theory admits quantifier
elimination (Tarski-Seidenberg theorem \cite{Tar51}).

%\begin{definition}
%  A set $X\subseteq\mathbb{R}^d$ is said to be semi-algebraic if it is
%  a finite Boolean combination of sets of the form $\{
%  \boldsymbol{x}\in \mathbb{R}^d : f(\boldsymbol{x}) > 0\}$ and $\{
%  \boldsymbol{x}\in \mathbb{R}^d : g(\boldsymbol{x}) = 0\}$, where $f$
%  is a polynomial with integer coefficients.
%\end{definition}

All sets that are definable in the first-order theory of reals without
quantification are by definition semi-algebraic, and it follows from
Tarski's theorem that this is still the case if we allow
quantification. We also remark that our standard representation of
algebraic numbers allows us to write them explicitly in the
first-order theory of reals, that is, given $\alpha\in\mathbb{A}$,
there exists a sentence $\sigma(x)$ such that $\sigma(x)$ is true if
and only if $x=\alpha$. Thus, we allow their use when defining
semi-algebraic sets, for simplicity.

It follows from the undecidability of Hilbert's Tenth Problem that, in
general, we cannot decide whether a given semi-algebraic set has an
integer point.

We shall make use of the following result by Basu, Pollack, and Roy
\cite{BasuPR96}, which tells us how expensive it is, in terms of space
usage, to perform quantifier elimination on a formula in the
first-order theory of reals:

\begin{theorem}
  \label{thm:quant-elim}
  Given a set $\mathcal{Q}=\lbrace q_1,\ldots,q_s\rbrace$ of $s$
  polynomials each of degree at most $D$, in $h+d$ variables, and a
  first-order formula $\Phi(\boldsymbol x)=Q y_1 \ldots Q y_h
  F(q_1(\boldsymbol x,\boldsymbol y),\ldots,q_s(\boldsymbol
  x,\boldsymbol y))$, where $Q\in\lbrace \exists,\forall\rbrace$, $F$
  is a quantifier-free Boolean combination with atomic elements of the
  form $q_i(\boldsymbol x,\boldsymbol y)\sim 0$, then there exists a
  quantifier-free formula $\Psi(\boldsymbol
  x)=\bigwedge_{i=1}^J\bigvee_{j=1}^{J_i}q_{ij}(\boldsymbol x)\sim 0$,
  where $I\leq (sD)^{O(hd)}$, $J\leq (sD)^{O(d)}$, the degrees of the
  polynomials $q_{ij}$ are bounded by $D^d$, and the bit-sizes of the
  heights of the polynomials in the quantifier-free formula are only
  polynomially larger than those of $q_1,\ldots,q_s$.
\end{theorem}

We also make use of the following lemmas:
\begin{lemma}
  If $X\subseteq\mathbb{R}^d$ is semi-algebraic and non-empty,
  $X\cap\mathbb{A}^d\neq\emptyset$.
\end{lemma}

\begin{proof}
  We prove this result by strong induction on $d$. Since $X$ is
  semi-algebraic, there exists a quantifier-free sentence in the
  first-order theory of reals $\sigma$ such that $X=\lbrace
  x\in\mathbb{R}^d\mid \sigma(x)\rbrace$.

  Suppose that $d>1$. Letting $X_1=\lbrace x_d\in\mathbb{R}\mid
  \exists
  x_1,\ldots,x_{d-1}\in\mathbb{R}^{d-1},\sigma(x_1,\ldots,x_d)\rbrace$
  and since $X_1\neq\emptyset$ is semi-algebraic, by the induction
  hypothesis, there must be $x_d^*\in\mathbb{A}\cap X_1$. Moreover, we
  can define $X_2=\lbrace (x_2,\ldots,x_d)\in\mathbb{R}^{d-1}\mid
  \sigma(x_1^*,x_2,\ldots,x_n)\rbrace$, which is non-empty and
  semi-algebraic, and again by induction hypothesis there exists some
  $(x_2^*,\ldots,x_d^*)\in\mathbb{A}^{d-1}\cap X_2$.

  It remains to prove this statement for $d=1$. When $d=1$, $X$ must
  be a finite union of intervals and points, since semi-algebraic sets
  form an o-minimal structure on $\mathbb{R}$ \cite{Tar51}. Clearly
  $\mathbb{A}$ is dense in any interval, and each of these isolated
  points $x$ corresponds to some constraint $g(x)=0$, which implies
  that $x$ must be algebraic, since $g$ has integer coefficients.
\end{proof}

\begin{lemma}
If $X\subseteq\mathbb{R}^d$ is semi-algebraic, then $X\cap\mathbb{A}^d$ is dense in $X$.
\end{lemma}

\begin{proof}
  Pick $x\in X$ and $\varepsilon>0$ arbitrarily. Let
  $y\in\mathbb{Q}^d$ be such that $\| x-y \|<\varepsilon/2$. Since
  $B(y,\varepsilon/2)$ is semi-algebraic, so must be $X\cap
  B(y,\varepsilon/2)$, and so this set must contain an algebraic
  point, since it is nonempty ($x$ is in it), and that point must
  therefore be at distance at most $\varepsilon$ of $x$, by the
  triangular inequality. By letting $\varepsilon\rightarrow 0$, we get
  a sequence of algebraic points which converges to $x$.
\end{proof}

\begin{lemma}
If $X\subseteq\mathbb{R}^d$ is semi-algebraic, so is $\overline{X}$.
\end{lemma}

\begin{proof}
  Let $\sigma$ be a sentence in the first-order theory of reals such
  that $X=\lbrace x\in\mathbb{R}^d\mid \sigma(x)\rbrace$. Whence
\begin{equation*}
  \overline{X}=\lbrace x\in\mathbb{R}^d\mid
\forall \varepsilon>0,\exists y\in\mathbb{R}^d,\sigma(y)\wedge y\in B(x,\varepsilon) \rbrace .
\end{equation*}
\end{proof}

%END SODA

%BEGIN LICS
\section{Convex Polyhedra and Semi-Algebraic Sets}

A convex polyhedron is a subset of $\mathbb{R}^{n}$ of the form $\mathcal{P} = \lbrace \boldsymbol{x} \in \mathbb{R}^{n} : A \boldsymbol{x} \leq \boldsymbol{b} \rbrace$, where $A$ is a $d \times n$ matrix and $\boldsymbol{b} \in \mathbb{R}^{d}$. When all the entries of $A$ and coordinates of $\boldsymbol{b}$ are algebraic numbers, the convex polyhedron $\mathcal{P}$ is said to have an algebraic description.

A set $S \subseteq \mathbb{R}^{n}$ is said to be semi-algebraic if it is a Boolean combination of sets of the form $\lbrace \boldsymbol{x} \in \mathbb{R}^{n}: p(\boldsymbol{x}) \geq 0\rbrace$, where $p$ is a polynomial with integer coefficients. Equivalently, the semi-algebraic sets are those definable by the quantifier-free first-order formulas over the structure $(\mathbb{R}, <, +, \cdot, 0, 1)$.

It was shown by Alfred Tarski in \cite{Tarski} that the first-order theory of reals admits quantifier elimination. Therefore, the semi-algebraic sets are precisely the first-order definable sets.

\begin{theorem}[Tarski]
The first-order theory of reals is decidable.
\end{theorem}

See \cite{Renegar} and \cite{BPR06} for more efficient decision procedures for the first-order theory of reals.

\begin{definition}[Hilbert's Tenth Problem]
Given a polynomial $p \in \mathbb{Z}[x_{1}, \ldots, x_{k}]$, decide whether $p(\boldsymbol{x}) = 0$ admits a solution $\boldsymbol{x} \in \mathbb{N}^{k}$. Equivalently, given a semi-algebraic set $S \subseteq \mathbb{R}^{k}$, decide whether it intersects $\mathbb{Z}^{k}$.
\end{definition}

The following celebrated theorem, due to Yuri Matiyasevich, will be used in
our undecidability proof; see \cite{HTP} for a self-contained proof.

\begin{theorem}[Matiyasevich]
Hilbert's Tenth Problem is undecidable.
\end{theorem}

On the other hand, our proof of decidability of ALIP makes use of some techniques present in the proof of the following result, shown in \cite{KP}:

\begin{theorem}[Khachiyan and Porkolab]
It is decidable whether a given \emph{convex} semi-algebraic set $S \subseteq \mathbb{R}^{k}$ intersects $\mathbb{Z}^{k}$.
\end{theorem}
%END LICS

%BEGIN HSCC
Let $\boldsymbol{x}=(x_1,\ldots,x_m)$ be a list of $m$ real-valued
variables, and let $\sigma(\boldsymbol{x})$ be a Boolean combination
of atomic predicates of the form $g(\boldsymbol{x})\sim 0$, where each
$g(\boldsymbol{x})$ is a polynomial with integer coefficients in the
variables $\boldsymbol{x}$, and $\sim$ is either $>$ or $=$. Tarski
has famously shown that we can decide the truth over the field
$\mathbb{R}$ of sentences of the form
$\phi=Q_1 x_1 \cdots Q_m x_m \sigma(\boldsymbol{x})$, where $Q_i$ is
either $\exists$ or $\forall$. He did so by showing that this theory
admits quantifier elimination (Tarski-Seidenberg theorem
\cite{Tar51}). The set of all true sentences of such form is called
the first-order theory of the reals, and the set of all true sentences
where only existential quantification is allowed is called the
existential first-order theory of the reals. The complexity class
$\exists\mathbb{R}$ is defined as the set of problems having a
polynomial-time many-one reduction to the existential theory of
the reals. It was shown in \cite{Canny88} that
$\exists\mathbb{R}\subseteq \mathit{PSPACE}$.

We also remark that our standard representation of algebraic numbers
allows us to write them explicitly in the first-order theory of the
reals, that is, given $\alpha\in\mathbb{A}$, there exists a sentence
$\sigma(x)$ such that $\sigma(x)$ is true if and only if
$x=\alpha$. Thus, we allow their use when writing sentences in the
first-order theory of the reals, for simplicity.

The decision version of linear programming with canonically-defined algebraic coefficients is in $\exists\mathbb{R}$, as the emptiness of a convex polytope can easily be described by a sentence of the form $\exists x_1 \cdots \exists x_n \sigma(\boldsymbol{x})$.

Finally, we note that even though the decision version of linear
programming with rational coefficients is in $\mathit{P}$, allowing
algebraic coefficients makes things more complicated. While it has
been shown in \cite{AdlerB94} that this is solvable in time polynomial
in the size of the problem instance and on the degree of the smallest
number field containing all algebraic numbers in each instance, it
turns out that in the problem at hand the degree of that extension can
be exponential in the size of the input. In other words, the splitting
field of the characteristic polynomial of a matrix can have a degree
which is exponential in the degree of the characteristic polynomial.
%END HSCC
