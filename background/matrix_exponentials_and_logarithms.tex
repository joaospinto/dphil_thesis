\subsection{Matrix exponentials}

Given a matrix $A \in \mathbb{C}^{n \times n}$, its exponential is defined as
\begin{align*}
\exp(A) = \sum \limits_{i=0}^{\infty} \frac{A^{i}}{i!} .
\end{align*}
The series above always converges, and so the exponential of a matrix is always well defined. The standard way of computing $\exp(A)$ is by finding $P \in \mathit{GL}_{n}(\mathbb{C})$ such that $J=P^{-1}AP$ is in Jordan Canonical Form, and by using the fact that $\exp(A) = P \exp(J) P^{-1}$, where $\exp(J)$ is easy to compute. When $A \in \overline{\mathbb{Q}}^{n \times n}$, $P$ can be taken to be in $GL_{n}(\overline{\mathbb{Q}})$; note that

\begin{align*}
\mbox{if } J &= \begin{pmatrix}
\lambda && 1 && 0 && \cdots && 0 \\
0 && \lambda && 1 &&\cdots && 0 \\
\vdots && \vdots && \ddots && \ddots && \vdots \\
0 && 0 && \cdots && \lambda && 1 \\
0 && 0 && \cdots && 0 && \lambda
\end{pmatrix} \mbox{ then } \\
\exp(Jt) &= \exp(\lambda t) \begin{pmatrix}
1 && t && \frac{t^{2}}{2} && \cdots && \frac{t^{k-1}}{(k-1)!} \\
0 && 1 && t && \cdots && \frac{t^{k-2}}{(k-2)!} \\
\vdots && \vdots &&\ddots && \ddots && \vdots \\
0 && 0 && \cdots && 1 && t \\
0 && 0 && \cdots && 0 && 1
\end{pmatrix} .
\end{align*}

Then $\exp(J)$ can be obtained by setting $t=1$, in particular $\exp(J)_{ij} = \frac{\exp(\lambda)}{(j-i)!}$ if $j \geq i$ and $0$ otherwise.

When $A$ and $B$ commute, so must $\exp(A)$ and $\exp(B)$. Moreover, when $A$ and $B$ have algebraic entries, the converse also holds, as shown in \cite{MatrixExps}. Also, when $A$ and $B$ commute, it holds that $\exp(A)\exp(B) = \exp(A+B)$.

\subsection{Matrix logarithms}

The matrix $B$ is said to be a logarithm of the matrix $A$ if $\exp(B) = A$. It is well known that a logarithm of a matrix $A$ exists if and only if $A$ is invertible. However, matrix logarithms need not be unique. In fact, there exist matrices admitting uncountably many logarithms. See, for example, \cite{MatrixLogs1} and \cite{MatrixLogs2}.

A matrix is said to be unitriangular if it is triangular and all its diagonal entries equal $1$. Crucially, the following uniqueness result holds:

\begin{theorem}
\label{logarithm_uniqueness}
Given an upper unitriangular matrix $M \in \mathbb{C}^{n \times n}$, there exists a unique strictly upper triangular matrix $L$ such that $\exp(L)=M$. Moreover, the entries of $L$ lie in the number field $\mathbb{Q}(M_{i,j}: 1 \leq i,j \leq n)$.
\end{theorem}

\begin{proof}
Firstly, we show that, for any strictly upper triangular matrix $T$ and for any $1<m<n$ and $i<j$, the term $(T^{m})_{i,j}$ is polynomial on the elements of the set $\lbrace T_{r,s} : s-r<j-i \rbrace$. This can be seen by induction on $m$, as each $T^{m}$ is strictly upper triangular, and so
\begin{align*}
(T^{m})_{i,j} = \sum\limits_{l=1}^{n} (T^{m-1})_{i,l} T_{l,j} = \sum\limits_{l=i+1}^{j-1} (T^{m-1})_{i,l} T_{l,j} .
\end{align*}

Finally, we show, by induction on $j-i$, that each $L_{i,j}$ is polynomial on the elements of the set
\begin{align*}
\lbrace M_{i,j} \rbrace \cup \lbrace M_{r,s} : s-r < j-i \rbrace .
\end{align*}
If $j-i \leq 0$, then $L_{i,j}=0$, so the claim holds. When $j-i>0$, as $L$ is nilpotent,
\begin{align*}
M_{i,j} &= \exp(L)_{i,j} = L_{i,j} + \sum\limits_{m=2}^{n-1} \frac{1}{m!} (L^{m})_{i,j} \\ \Rightarrow L_{i,j} &= M_{i,j} - \sum\limits_{m=2}^{n-1} \frac{1}{m!} (L^{m})_{i,j} .
\end{align*}
The result now follows from the induction hypothesis and from our previous claim, as this argument can be used to both construct such a matrix $L$ and to prove that it is uniquely determined.
\end{proof}
