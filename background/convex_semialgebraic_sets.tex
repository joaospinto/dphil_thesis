\subsection{Integer Points of Convex Semi-algebraic Sets}

Due to \cref{thm:HTP}, it is undecidable whether a given semi-algebraic set $S$ in $\Reals^{d}$ intersects the integer lattice $\Integers^{d}$. However, if one knows that $S$ is convex, undecidability breaks down, as shown in~\cite{KhachiyanP97}. This result is central to our approach in \cref{chapter:soda}.

\begin{theorem}[Khachiyan and Porkolab]
\label{thm:kp}
Let $W\subseteq\Reals^{d}$ be a convex semi-algebraic set defined by
polynomials of degree at most $D$ and that can be represented in space\footnote{Here, the space of the representation of a polynomial is defined to be the sum of the base-$2$ logarithms of its coefficients.}
$S$. In that case, if $W\cap\Integers^{d}\neq\emptyset$, then $W$ must
contain an integral point that can be represented in space
$SD^{O(d^4)}$.
\end{theorem}
