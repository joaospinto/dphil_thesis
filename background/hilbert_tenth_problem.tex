\subsection{Hilbert's Tenth Problem}
Given a polynomial $p \in \Integers[x_{1}, \ldots, x_{k}]$, Hilbert's Tenth Problem consists in deciding whether $p(\myvector{x}) = 0$ admits a solution $\myvector{x} \in \Naturals^{k}$. Equivalently, it can be seen as the problem of deciding whether a given semi-algebraic set intersects the integer lattice. The following celebrated theorem, due to Yuri Matiyasevich, settled this question negatively; see~\cite{HTP} for a self-contained proof.
\begin{theorem}[Matiyasevich]
Hilbert's Tenth Problem is undecidable.
\end{theorem}
This result can be seen as a strengthening of the famous G\"{o}del-Rosser incompleteness theorem, which states that the first-order theory of integers is undecidable. Namely, this result shows that the existential branch of that theory is already undecidable.

We shall also need the following result, shown in~\cite{KP}:
\begin{theorem}[Khachiyan and Porkolab]
\label{thm:KP}
It is decidable whether a given \emph{convex} semi-algebraic set $S \subseteq \Reals^{k}$ intersects $\Integers^{k}$.
\end{theorem}
