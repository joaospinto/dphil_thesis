\subsection{Hilbert's Tenth Problem}
Given a polynomial $p \in \Integers[x_{1}, \ldots, x_{k}]$, Hilbert's Tenth Problem consists in deciding whether $p(\myvector{x}) = 0$ admits a solution $\myvector{x} \in \Naturals^{k}$. The following celebrated theorem, due to Yuri Matiyasevich, settled this question negatively; see~\cite{HTP} for a self-contained proof.
\begin{theorem}[Matiyasevich]
\label{thm:HTP}
%TODO: Actually state that RE = Diophantine.
Hilbert's Tenth Problem is undecidable.
\end{theorem}
This result can be seen as a strengthening of the famous G\"{o}del-Rosser incompleteness theorem, which states that the first-order theory of integers is undecidable. Namely, this result shows that the existential branch of that theory is already undecidable. As a consequence, it is undecidable whether a given semi-algebraic set intersects the integer lattice.

We shall also need the following result, shown in~\cite{KhachiyanP97}:
\begin{theorem}[Khachiyan and Porkolab]
\label{thm:kp}
Let $W\subseteq\Reals^{d}$ be a convex semi-algebraic set defined by
polynomials of degree at most $D$ and that can be represented in space
$S$. In that case, if $W\cap\Integers^{d}\neq\emptyset$, then $W$ must
contain an integral point that can be represented in space
$SD^{O(d^4)}$.
\end{theorem}

\subsection{Arithmetical Hierarchy, Turing Jumps, and Post's Theorem}
%TODO
The purpose of this section is to describe the \emph{arithmetical hierarchy} and Post's theorem, which will be needed to understand~\cref{sec:turing-degree-lics}.

Consider a quantifier-free formula $\Psi$ consisting of Boolean combinations of atomic predicates of the form $p(x_{1}, \ldots, x_{n}) \geq 0$, where $p \in \Integers[x_{1}, \ldots, x_{n}]$.
The class of such formulas is $\Sigma_{0}=\Pi_{0}$.
Moreover, if $\Phi \in \Pi_{n}$ and $x_{1}, \ldots, x_{n}$ are free variables in $\Phi$, then $\exists x_{1}, \ldots, \exists x_{n} \Phi$ belongs to $\Sigma_{n+1}$.
Similarly, if $\Phi \in \Sigma_{n}$ and $x_{1}, \ldots, x_{n}$ are free variables in $\Phi$, then $\forall x_{1}, \ldots, \forall x_{n} \Phi$ belongs to $\Pi_{n+1}$.
In order to extend these classes beyond formulas in prenex normal form, we make them closed under logical equivalence. Therefore, if $\Phi \in \Sigma_{n}$, then $\neg \Phi \in \Pi_{n}$, and vice-versa.
Furthermore, we define $\Delta_{n} = \Sigma_{n} \cap \Pi_{n}$.

%TODO: mention that \Sigma_1 = RE.
%TODO: define completeness
By abuse of notation, ...%TODO

\begin{theorem}[Post]
  \label{thm:Post}
  TODO
\end{theorem}
