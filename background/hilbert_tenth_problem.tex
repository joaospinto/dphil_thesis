\subsection{Hilbert's Tenth Problem}
Given a polynomial $p \in \Integers[x_{1}, \ldots, x_{k}]$, Hilbert's Tenth Problem consists in deciding whether $p(\myvector{x}) = 0$ admits a solution $\myvector{x} \in \Naturals^{k}$. The following celebrated theorem, due to Yuri Matiyasevich, settled this question negatively; see~\cite{HTP} for a self-contained proof.
\begin{theorem}[Matiyasevich]
\label{thm:HTP}
Hilbert's Tenth Problem is undecidable.
\end{theorem}
This result can be seen as a strengthening of the famous G\"{o}del-Rosser incompleteness theorem, which states that the first-order theory of integers is undecidable. Namely, this result shows that the existential branch of that theory is already undecidable. As a consequence, it is undecidable whether a given semi-algebraic set intersects the integer lattice.

We shall also need the following result, shown in~\cite{KhachiyanP97}:
\begin{theorem}[Khachiyan and Porkolab]
\label{thm:kp}
Let $W\subseteq\Reals^{d}$ be a convex semi-algebraic set defined by
polynomials of degree at most $D$ and that can be represented in space
$S$. In that case, if $W\cap\Integers^{d}\neq\emptyset$, then $W$ must
contain an integral point that can be represented in space
$SD^{O(d^4)}$.
\end{theorem}
