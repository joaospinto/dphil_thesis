\section{Introduction}
\label{sec:soda_intro}

Termination is a fundamental decision problem in program verification.
In particular, termination of programs with linear assignments and
linear conditionals has been extensively studied over the last decade.
This has led to the development of powerful techniques to
prove termination via synthesis of linear ranking
functions~\cite{Ben-AmramG13,BradleyMS05,ChenFM12,ColonS01,PodelskiR04},
many of which have been implemented in software-verification tools, such as
Microsoft's \textsc{Terminator}~\cite{CookPR06}.

A very simple form of imperative programs are \emph{simple linear
  loops}, that is, programs of the form
\begin{gather*}
\mathsf{P1:}\  \mbox{$\myvector{x}\gets \myvector{u}$ ;
\textit{while} $B\myvector{x} \geq \myvector{c}$ \textit{do}
$\myvector{x}\leftarrow A\myvector{x}+\myvector{a}$,}
\end{gather*}
where $\myvector{x}$ is vector of variables, $\myvector{u}$,
$\myvector{a}$, and $\myvector{c}$ are integer vectors, and $A$
and $B$ are integer matrices of the appropriate dimensions.  Here the
loop guard is a conjunction of linear inequalities and the loop body
consists of a simultaneous affine assignment to $\myvector{x}$.  If
the vectors $\myvector{a}$ and $\myvector{c}$ are both zero then
we say that the loop is \emph{homogeneous}.

Suppose that the vector $\myvector{x}$ has dimension $d$.  We say
that \textsf{P1} \emph{terminates} on a set $S\subseteq \Reals^{d}$
if it terminates for all initial vectors $\myvector{u} \in S$.
Tiwari~\cite{Tiw04} gave a procedure to decide whether a given simple
linear loop terminates on $\Reals^{d}$.  Later
Braverman~\cite{Bra06} showed decidability of termination on
$\Rationals^{d}$.  However the most natural problem from the point of
view of program verification is termination on $\Integers^{d}$.

While termination on $\Integers^{d}$ reduces to termination on
$\Rationals^{d}$ in the homogeneous case (by a straightforward scaling
argument), termination on $\Integers^{d}$ in the general case is stated
as an open problem in~\cite{BGM12,Bra06,Tiw04}.  The main result of
this chapter is a procedure to decide termination on $\Integers^{d}$ for
simple linear loops when the assignment matrix $A$ is diagonalisable.
This represents the first substantial progress on this open problem in
over $10$ years.

Termination of more complex programs can often be reduced to
termination of simple linear loops (see, e.g.,~\cite{CookPR06}
or~\cite[Section 6]{Tiw04}).  On the other hand, termination becomes
undecidable for mild generalisations of simple linear loops, for
example, allowing the update function in the loop body to be piecewise
linear~\cite{BGM12}.

To prove our main result we focus on \emph{eventual non-termination},
where \textsf{P1} is said to be eventually non-terminating on
$\myvector{u} \in \Integers^{d}$ if, starting from initial value
$\myvector{u}$, after executing the loop body $\myvector{x} \gets
A\myvector{x}+\myvector{a}$ a finite number of times \emph{while
  disregarding the loop guard} we eventually reach a value on which
\textsf{P1} fails to terminate.  Recall that $A$ and $\myvector{a}$
have integer coefficients, so clearly \textsf{P1}~fails to
terminate on $\Integers^{d}$ if and only if it is eventually
non-terminating on some $\myvector{u} \in \Integers^{d}$.

Given a simple linear loop we show how to compute a convex
semi-algebraic set $W \subseteq \Reals^{d}$ such that the integer
points $\myvector{u} \in W$ are precisely the eventually
non-terminating integer initial values.  Since it is decidable whether
a convex semi-algebraic set contains an integer
point~\cite{KhachiyanP97}, we can
decide whether an integer linear loop is terminating on
$\Integers^{d}$. Convexity is crucial here: deciding whether an arbitrary
semi-algebraic set contains any integer points is equivalent to
Hilbert's Tenth Problem, and this is undecidable.

Termination over the set of all integer points is easily seen to be
\coNP-hard.  Indeed, if the update function in the loop body
is the identity then the loop is non-terminating if and only if there
is an integer point satisfying the guard.  Thus non-termination
subsumes integer programming, which is \NP-hard.

While our algorithm for deciding termination requires exponential
space, it should be noted that the procedure actually solves a more
general problem than merely determining the existence of a
non-terminating integer point (or, equivalently, the existence of an
eventually non-terminating integer point).  In fact the algorithm
computes a representation of the set of all eventually non-terminating
integer points.  For reference, the closely related problem of
deciding termination on the integer points in a given convex polytope is
\EXPSPACE-hard~\cite{BGM12}. By contrast, even though not stated explicitly in~\cite{Tiw04} and~\cite{Bra06}, deciding termination on $\Reals^{d}$ and $\Rationals^{d}$ can be reduced to a problem in $\exists \Reals$, which is within \PSPACE.

As well as making extensive use of algorithms in real algebraic
geometry, the soundness of our decision procedure relies on powerful
lower bounds in Diophantine approximation that generalise Roth's
Theorem.  (The need for such bounds in the inhomogeneous setting was
conjectured in the conclusion of~\cite{Bra06}.)  We
also use classical results in number theory, such as the
Skolem-Mahler-Lech Theorem~\cite{Lec53,Mah35,Sko34} on linear
recurrences.  Crucially the well-known and notorious ineffectiveness
of Roth's Theorem (and its higher-dimensional and $p$-adic
generalisations) and of the Skolem-Mahler-Lech Theorem are not a
problem for deciding \emph{eventual} non-termination, which is key to
our approach.

\subsection{Related Work}

Consider the termination problem for a homogeneous simple linear loop
\begin{gather*}
\mathsf{P2:}\  \mbox{$\myvector{x}\gets \myvector{u}$ ;
\textit{while} $B\myvector{x} \geq 0$ \textit{do} $\myvector{x}\leftarrow A\myvector{x}$}
\end{gather*}
on a single initial value $\myvector{u}\in \Integers^{d}$.  Each row
$\myvector{b}^T$ of matrix $B$ corresponds to a loop condition
$\myvector{b}^T\myvector{x} \geq 0$.  For each such condition,
consider the integer sequence $\langle x_n : n \in \Naturals \rangle$
defined by $x_n = \myvector{b}^TA^n \myvector{u}$.  Then
\textsf{P2} fails to terminate on an initial value $\myvector{u}$ if
and only if each such sequence $\langle x_n \rangle$ is
\emph{positive}, i.e., $x_n \geq 0$ for all $n$.  It is not difficult
to show that each sequence $\langle x_n \rangle$ considered above is a
\emph{linear recurrence sequence}, as we saw in~\cref{sec:LRS}.
Thus deciding whether a homogeneous simple linear loop
terminates on a given initial value is at least as hard as the
\emph{Positivity Problem} for linear recurrence sequences, that is,
the problem of deciding whether a given linear recurrence sequence has
exclusively non-negative terms.

The Positivity Problem has been studied at least as far back as the
1970s~\cite{BG07,HHH06,Liu10,RS94,Sal76}.  Thus far decidability is
known only for sequences satisfying recurrences of order $5$ or less.
It is moreover known that showing decidability at order $6$ will
necessarily entail breakthroughs in transcendental number theory,
specifically significant new results in Diophantine
approximation~\cite{OW14:SODA}. Even for \emph{simple} linear recurrence
sequences, decidability has only been established for order at
most~$9$~\cite{OW13:constructive-positivity}.

The key difference between studying termination of simple linear loops
over $\Integers^{d}$ rather than a single initial value is that the
former problem can be approached through eventual termination.  In
this sense the termination problem is related to the \emph{Ultimate
  Positivity Problem} for linear recurrence sequences, which asks
whether all but finitely many terms of a given sequence are
positive~\cite{OuaknineW13b}.  This allows us to bring to bear powerful
non-effective Diophantine-approximation techniques,
specifically the $S$-units Theorem of Evertse, van der Poorten, and
Schlickewei~\cite{Evertse84,PS82}.  Such tools enable us to obtain
decidability of termination for matrices of arbitrary dimension,
assuming diagonalisability.

The paper~\cite{COW13} studies higher dimensional versions of
Kannan and Lipton's Orbit Problem~\cite{KL86}.  These can be seen as
versions of the termination problem for linear loops on a fixed
initial value.  That work uses substantially different technology from
that of this chapter, including Baker's Theorem on linear forms
in logarithms~\cite{BW93}, and correspondingly relies on restrictions
on the dimension of data in problem instances to obtain decidability.

%Since Kannan and Lipton solved the \emph{Orbit Problem}~\cite{KL86},
%which consists in deciding whether $\exists n\in\Naturals: A^n x=y$
%for given $x,y\in\Rationals^n,A\in\Rationals^{n\times n}$, a lot of
%work has been done in related problems. For example, higher
%dimensional versions of the Orbit Problem have been considered, namely
%replacing the target point $y$ by some target subspace $V$
%\cite{COW13}. Whenever $dim(V)\leq 3$, this problem is in
%\PSPACE. On the other hand, when $V$ is a hyperplane of the
%embedding vector space, this corresponds to the famous
%\emph{Skolem's Problem}, which can equivalently be stated as that of
%finding whether the set $Z(u_n)=\lbrace n\in\Naturals\mid
%u_n=0\rbrace$ is empty for a given linear recurrence sequence (LRS)
%$u_n$. The celebrated Skolem-Mahler-Lech theorem
%\cite{Lec53,Mah35,Sko34,TUCS05,Hansel85} characterises $Z(u_n)$ as the
%union of a finite set and finitely many (effectively computable
%\cite{BM76}) arithmetic progressions. Blondel and Tsitsiklis
%\cite{BlondelT00} remark that ``the present consensus among number
%theorists is that an algorithm [for Skolem's Problem] should exist'',
%even though such algorithms are only known for LRS of order up to $4$,
%as proved independently by Vereschagin~\cite{Ver85} and Mignotte,
%Shorey, and Tijdeman~\cite{MST84}. This problem has been discussed at
%length by Terence Tao in~\cite{Tao07} and by Lipton in
%\cite{Lip09}. Deciding the \emph{Positivity} problem (whether
%$\forall n\geq 0,u_n\geq 0$) and the \emph{Ultimate Positivity}
%problem (whether $\lbrace n\in\Naturals\mid u_n< 0\rbrace$ is finite)
%is also only known to be possible for LRS of order up to 5 in general
%\cite{OW14:SODA} and up to 9 for a relevant subclass
%\cite{OW13:constructive-positivity}. In fact, there is a one-line
%reduction from Skolem's Problem to the Positivity Problem which does
%not, however, preserve orders. Ouaknine and Worrell
%\cite{OuaknineW13b} showed that Ultimate Positivity is decidable in
%polynomial space (and in polynomial time for a fixed order) for all
%simple LRS (those whose characteristic polynomial has no repeated
%roots), and we borrow many of the techniques used there. Finally, the
%Polyhedron Hitting Problem, where one replaces the target subspace $V$
%of the generalised Orbit Problem by a polytope $\mathcal{P}$, has
%recently seen some interesting progress.

%Even though the results of Tiwari~\cite{Tiw04} and Braverman~\cite{Bra06} rely almost exclusively on elementary linear algebra and linear programming, stronger results from both analytic and algebraic number theory tend to be necessary to tackle all the other problems we discussed, namely those involving linear recurrence sequences. In fact, the same holds for this chapter, as conjectured to be needed by Braverman in~\cite{Bra06}. Moreover, even though the procedures for solving the universal termination problem over $\Reals$ and $\Rationals$ run in polynomial time (requiring some non-trivial subroutines such as computing Jordan Canonical Forms~\cite{Cai94} and solving Linear Programming problems with algebraic numbers~\cite{AdlerB94}\footnote{In order to find $z_{max}$ and $index_{z_{max}}$, we can simply determine $index_v$ and $index_{-v}$ for each eigenvector $v$ associated with a positive real eigenvalue and take $index_{z_{max}}$ to be the maximum of these $index_v$ and $index_{-v}$, after which we only have to solve a linear programming that is guaranteed to have a solution in order to determine $z_{max}$}), it is easy to see that deciding the existence of a non-terminating point becomes \NP-hard over the integers, by a trivial reduction from the problem of deciding the existence of an integer point in a polytope. Such a hardness result suggests that this problem is thus intrinsically different from Tiwari's or Braverman's, therefore further emphasising our point.

%Braverman conjectured in~\cite{Bra06} that one would need to decide whether, for a given $x\in\Integers^k$, $\forall n\in\Naturals,f^{n}(x)\in\mathcal{P}$ (that is, point-wise termination, which is equivalent to deciding the Positivity Problem) in order to decide universal termination. As it turns out, we only needed to be able to decide whether, for a given $x\in\Integers^k$, $\exists N\in\Naturals:\forall n\geq N,f^{n}(x)\in\mathcal{P}$ (that is, eventual pointwise termination, which is equivalent to deciding the Ultimate Positivity Problem).

Termination of $\mathsf{P1}$ under the assumption that all eigenvalues
of $A$ are real was studied in~\cite{RMM,RMM1} using spectral
techniques. However, as will become clear throughout the course of
this chapter, most of the machinery that we use is needed to tackle the
case where there are both real and complex eigenvalues with the same
absolute value. In the setting of~\cite{RMM,RMM1}, the set of
eventually non-terminating points is in fact a polytope, which can be
effectively computed resorting only to straightforward linear algebra.

While we use spectral and number-theoretic techniques in this chapter,
another well-studied approach for proving termination of linear loops
involves designing linear ranking functions, that is, linear functions
from the state space to a well-founded domain such that each iteration
of the loop strictly decreases the value of the ranking
function. However, this approach is incomplete: it is not hard to
construct an example of a terminating loop which admits no linear
ranking function.  Sound and relatively complete methods for
synthesising linear ranking functions can be found in
\cite{PodelskiR04} and~\cite{Ben-AmramG13}. Whether a linear
ranking function exists can be decided in polynomial time when the
state space is $\Rationals^{d}$ and is \coNP-complete when the
state space is $\Integers^{d}$.
