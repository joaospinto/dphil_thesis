\section{Conclusion}

%% Our techniques can actually be altered in order to solve another
%% related problem. One may ask if, given a convex semi-algebraic set, an
%% affine operator (under the same restrictions that we previously
%% imposed), and a polytope, whether some integral point in that convex,
%% semi-algebraic set will be such that its orbit by the affine operator
%% will intersect the polytope infinitely often. These set-to-set
%% reachibility problems remain largely unexplored. For example, it would
%% be interesting to consider the version of this problem where we omit
%% the requirement that this intersection holds infinitely often, but
%% instead ask whether it holds at least once, even though this would
%% probably involve some technology that was not needed in this chapter,
%% such as Baker's theorem~\cite{BW93}.

%% As opposed to what happens with the problems of deciding termination
%% over the reals and rationals, it seems to be hard to come up with a
%% non-constructive procedure for deciding termination over the
%% integers. Braverman conjectured in~\cite{Bra06} that it would be
%% necessary that one is able to decide pointwise termination in order to
%% decide universal termination over the integers, but it seems unlikely
%% that one can formally reduce one problem to the other.

%% It is remarkable that, in some cases, solving the termination problem
%% over the integers is much easier than solving its real and rational
%% counterparts, namely when the polytope defined by the loop guard is
%% bounded, in which case this problem reduces to that of finding a cycle
%% in a graph.

%% While the \NP-hardness of this problem suggests that
%% termination over the integers is intrinsically harder, the gap between
%% this lower bound and our \EXPSPACE upper bound is far from
%% satisfactory.

We have shown decidability of termination of simple linear loops over
the integers under the assumption that the update matrix is
diagonalisable, partially answering an open problem
of~\cite{Tiw04,Bra06}.  As we have explained before, the termination
problem on the same class of linear loops, but for fixed initial
values, seems to have a different character and to be more difficult.
In this respect it is interesting to note that there are other
settings in which universal termination is an easier problem than
pointwise termination. For example, universal termination of Petri
nets (also known as \textit{structural boundedness}) is
\PTIME-decidable, but the pointwise termination problem is
\EXPSPACE-hard.

Braverman conjectured in~\cite{Bra06} that it would be necessary that one be able to decide pointwise termination in order to decide universal termination over the integers. By contrast, the approach we have taken in this chapter has focused \emph{eventual} termination: even for \emph{simple} linear loops, the question of pointwise termination remains open.

A natural subject for further work would be the extension of our result to instances with non-diagonalisable matrices, or showing that there are unavoidable number-theoretic obstacles to proving decidability of this problem, as is the case with Ultimate Positivity~\cite{OW14:SODA}.
Another relevant problem would be the computational complexity of the termination problem.  While there is a large gap between the \coNP{} lower complexity bound mentioned in~\cref{sec:soda_intro} and the \EXPSPACE{} upper bound of our procedure, this may be connected with the fact that our procedure computes a representation of the set of all integer eventually non-terminating points.

Finally, there is the question of whether the respective sets of terminating and non-terminating points are semi-algebraic.  Note that an \emph{effective} semi-algebraic characterisation of the set of terminating points would allow us to solve the termination problem over fixed initial values.

%% We conjecture that deciding eventual termination for all integer
%% points in a given convex semi-algebraic set is \EXPSPACE-hard,
%% as is the problem of deciding termination for all integer points in a
%% given polytope~\cite{BGM12}, but believe that better bounds on the
%% universal termination problem over the integers should be possible to
%% achieve.

%% Finally, we conjecture that deciding arbitrary instances of this
%% problem with non-diagonalisable update matrices in dimension at least
%% $5$ should be hard, in a similar sense to the one described in
%%~\cite{OW14:SODA}.
