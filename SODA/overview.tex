\section{Overview of Main Results}
\label{sec:overview}
The main result of this paper is as follows:
\begin{theorem}
  The termination over the integers of simple linear loops of the form
\begin{gather*}
\mathsf{P1:}\  \mbox{$\boldsymbol{x}\gets \boldsymbol{u}$ ;
\textit{while} $B\boldsymbol{x} \geq \boldsymbol{c}$ \textit{do}
$\boldsymbol{x}\leftarrow A\boldsymbol{x}+\boldsymbol{a}$}
\end{gather*}
is decidable using exponential space if $A$ is diagonalisable and
using polynomial space if $A$ has dimension at most $4$.
\label{thm:main}
\end{theorem}
In this section we give a high-level overview of the proof of Theorem~\ref{thm:main}.

Let $f:\mathbb{R}^d\rightarrow \mathbb{R}^d$ be the affine function
$f(\boldsymbol{x})=A\boldsymbol{x}+\boldsymbol a$ computed by the body
of the while loop in \textsf{P1} and $P=\{ \boldsymbol{x} \in
\mathbb{R}^d: B \boldsymbol{x}\geq \boldsymbol{c}\}$ the convex polytope
corresponding to the loop guard.  We define the set of
\emph{non-terminating points} to be
\[ \mathit{NT} = \{ \boldsymbol{u}\in \mathbb{R}^d : \forall n \in
\mathbb{N},\, f^n(\boldsymbol{u}) \in P \} \, .\] Following
Braverman~\cite{Bra06}, we moreover define the set of \emph{eventually
  non-terminating points} to be
\[ \mathit{ENT} = \{ \boldsymbol{u}\in \mathbb{R}^d : \exists n \in
\mathbb{N},\, f^n(\boldsymbol{u}) \in \mathit{NT} \} \, .\]
It is easily seen from the above definitions that both \textit{NT} and
\textit{ENT} are convex sets.

By definition, \textsf{P1} is non-terminating on $\mathbb{Z}^d$ if
and only if $\mathit{NT}$ contains an integer point.  It is moreover
clear that $\mathit{NT}$ contains an integer point if and only if
$\mathit{ENT}$ contains an integer point.

%% It is immediate that the problem of termination on $\mathbb{Z}^d$ is
%% \textbf{coNP}-hard.  Indeed, taking $f$ to be the identity function,
%% loop \textsf{P1} is non-terminating on initial value $\boldsymbol{u}$
%% if and only if $\boldsymbol{u}$ satisfies the loop guard.  Thus the
%% non-termination problem subsumes the problem of
%% whether a given polytope contains an integer point (i.e., integer
%% programming), which is \textbf{NP}-hard.  By contrast, even though not
%% stated explicitly in~\cite{Tiw04} and~\cite{Bra06}, deciding
%% termination on $\mathbb{R}^d$ and $\mathbb{Q}^d$ can be done in
%% polynomial time.  This observation relies on the facts that one can
%% compute Jordan canonical forms of integer matrices and solve instances
%% of linear programming problems with algebraic numbers in polynomial
%% time~\cite{Cai94,AdlerB94}.

Recall that a subset of $\mathbb{R}^d$ is said to be
\emph{semi-algebraic} if it is a Boolean combination of sets of the
form $\{ \boldsymbol{x} \in \mathbb{R}^d : p(\boldsymbol{x})\geq 0\}$,
where $p$ is a polynomial with integer coefficients.  Equivalently the
semi-algebraic sets are those definable by quantifier-free first-order
formulas over the structure $(\mathbb{R},<,+,\cdot,0,1)$.  In fact,
since the first-order theory of the reals admits quantifier
elimination~\cite{Tar51}, the semi-algebraic sets are precisely the
first-order definable sets.

Define $W \subseteq \mathbb{R}^d$ to be a
\emph{non-termination witness set} (or simply a witness set) if it
satisfies the following two properties (where $\mathbb{A}$ denotes the
set of algebraic numbers):
\begin{itemize}
\item[(i)] $W$ is convex and semi-algebraic;
\item[(ii)] $W\cap \mathbb{A}^d=\mathit{ENT}\cap \mathbb{A}^d$.
\end{itemize}

The integer points in a witness set $W$ are precisely the integer
points of \textit{ENT}, and so \textsf{P1} is non-terminating on
$\mathbb{Z}^d$ precisely when $W$ contains an integer point.  Our
approach to solving the termination problem consists in computing a
witness set $W$ for a given program and then using the following
theorem of Khachiyan and Porkolab~\cite{KhachiyanP97} to decide whether $W$
contains an integer point.
\begin{theorem}[Khachiyan and Porkolab]
Let $W\subseteq\mathbb{R}^d$ be a convex semi-algebraic set defined by
polynomials of degree at most $D$ and that can be represented in space
$S$. In that case, if $W\cap\mathbb{Z}^d\neq\emptyset$, then $W$ must
contain an integral point that can be represented in space
$SD^{O(d^4)}$.
\end{theorem}

Our approach does not attempt to characterise the set \textit{ENT}
directly, but rather uses the witness set $W$ as a proxy. However, our
techniques do allow us to establish that
$\overline{\mathit{ENT}}=\overline{W}$, which in particular implies
that $\overline{\mathit{ENT}}$ is semi-algebraic, since the closure of
a semi-algebraic set is semi-algebraic (see the Appendix for
details). A natural question is whether the set $ENT$
itself is semi-algebraic, which we leave as an open problem.

We next describe some restrictions on linear loops that can
be made without loss of generality and that will ease our upcoming
analysis.

We first reduce the problem of computing witness sets in the general
case to the same problem in the homogeneous case.  Note that Program
\textsf{P1} terminates on a given initial value $\boldsymbol{u}\in
\mathbb{Z}^d$ if and only if the homogeneous program \textsf{P3} below
terminates for the same value of $\boldsymbol{u}$:
\begin{align*}
\mathsf{P3:} \boldsymbol{x}\gets
\begin{pmatrix} \boldsymbol{u}\\ 1\end{pmatrix} \mbox{\textit{while}} \begin{pmatrix}B &-\boldsymbol{c} \end{pmatrix}
\boldsymbol{x} \geq 0 \mbox{ \textit{do} } \boldsymbol{x}\leftarrow
\begin{pmatrix}A&\boldsymbol{a}\\ 0 & 1
\end{pmatrix} \boldsymbol{x}
\end{align*}

Note that if $A$ is diagonalisable then all
eigenvalues of
$\begin{pmatrix}A&\boldsymbol{a}\\ 0 & 1
\end{pmatrix}$ are simple, with the possible exception of the
eigenvalue $1$.  (Recall that an eigenvalue is said to be simple if it
has multiplicity one as a root of the minimal polynomial of $A$.)  Now
if $W$ is a witness set for program \textsf{P3} then $\left\{
  \boldsymbol{u} \in \mathbb{R}^d : \begin{pmatrix} \boldsymbol{u}\\
    1\end{pmatrix} \in W \right\}$ is a witness set for \textsf{P1}.
We conclude that, in order to settle the inhomogeneous case with a
diagonalisable matrix, it suffices to compute a witness set in the
case of a homogeneous linear loop \textsf{P2} in which the only
repeated eigenvalues of the new matrix $A$ are positive and real.
Likewise, to handle the inhomogeneous case for matrices of dimension
at most $d$, it suffices to be able to compute witness sets in the
homogeneous case for matrices of dimension at most $d+1$.%
\footnote{Note that whilst Braverman~\cite{Bra06} shows how to decide
  termination over the integers for homogeneous programs with
  arbitrary update matrices, he does \emph{not} compute a witness set
  for such programs---indeed this remains an open problem since it
  would enable one to solve termination over the integers for
  arbitrary inhomogeneous programs.}

We can further simplify the homogeneous case by restricting to loop
guards that comprise a single linear inequality.  To see this, first
note that program \textsf{P2} above is eventually non-terminating on
$\boldsymbol{u}$ if and only if for each row $\boldsymbol{b}^T$ of $B$
program \textsf{P4} below is eventually non-terminating on
$\boldsymbol{u}$:
\begin{gather*}
\mathsf{P4:}\ \mbox{$\boldsymbol{x}\gets \boldsymbol{u}$ ;
\textit{while} $\boldsymbol{b}^T\boldsymbol{x} \geq 0$ \textit{do} $\boldsymbol{x}\leftarrow A\boldsymbol{x}$.}
\end{gather*}
Noting that the finite intersection of convex semi-algebraic sets is
again convex and semi-algebraic, we can compute a witness set
for \textsf{P2} as the intersection of witness sets for each version
of \textsf{P4}.

The final simplification concerns the notion of
non-degeneracy.  We say that matrix $A$ is \emph{degenerate} if it has
distinct eigenvalues $\lambda_1 \neq \lambda_2$ whose quotient
$\lambda_1/\lambda_2$ is a root of unity.

Given an arbitrary matrix $A$, let $L$ be the least common multiple of all orders of quotients of distinct eigenvalues of $A$ which are roots
of unity. It is known that $L=2^{O(d\sqrt{\log d})}$ \cite{BOOK}.  The
eigenvalues of the matrix $A^L$ have the form $\lambda^L$ for
$\lambda$ an eigenvalue of $A$, by the spectral mapping theorem.  It follows that
$A^L$ is non-degenerate, since if $\lambda_1,\lambda_2$ are
eigenvalues of $A$ such that $\lambda^L_1/\lambda^L_2$ is a root of
unity then $\lambda_1/\lambda_2$ is a root of unity and hence
$\lambda^L_1/\lambda^L_2=1$. Note that all eigenvectors of $A$ are still eigenvectors of $A^L$, thus $A^L$ will be diagonalisable whenever $A$ is.

Now program \textsf{P4} is eventually non-terminating on
$\boldsymbol{u}\in \mathbb{Z}^d$ if and only if program \textsf{P5}
below is eventually non-terminating on the set
$\{\boldsymbol{u},A\boldsymbol{u}, \ldots,A^{L-1}\boldsymbol{u}\}$:
\begin{gather*}
\mathsf{P5:}\ \mbox{$\boldsymbol{x}\gets \boldsymbol{v}$ ;
\textit{while} $\boldsymbol b^T\boldsymbol{x} \geq 0$ \textit{do} $\boldsymbol{x}\leftarrow A^L\boldsymbol{x}$.}
\end{gather*}
Thus if $W$ is a witness set for \textsf{P5} then $\bigcap_{i=0}^{L-1}
\{ \boldsymbol{u} \in \mathbb{Z}^d : A^i\boldsymbol{u} \in W\}$ is a witness set for
\textsf{P4}.

The main technical result of the paper is the following proposition:
\begin{proposition}
Given a homogeneous simple linear loop
\begin{gather*}
\mathsf{P4:}\ \mbox{$\boldsymbol{x}\gets \boldsymbol{u}$ ;
\textit{while} $\boldsymbol{b}^T\boldsymbol{x} \geq 0$ \textit{do} $\boldsymbol{x}\leftarrow A\boldsymbol{x}$,}
\end{gather*}
such that $A$ is non-degenerate and either $A$ has dimension at most
$5$ or all complex eigenvalues of $A$ are simple,  we can compute a
witness set for \textsf{P4} using exponential space if $A$ is diagonalisable and using polynomial space if $A$ has dimension at most $5$.
\label{prop:main}
\end{proposition}

Bearing in mind that the transformation from $\mathsf{P1}$ to
$\mathsf{P4}$ increases the dimension of $A$ by one and does not
introduce repeated complex eigenvalues, it follows from
Proposition~\ref{prop:main} that we can also compute witness sets for
simple linear loops of the form \textsf{P1}
under the assumptions of Theorem~\ref{thm:main}, and thus we obtain
the decidability part of Theorem~\ref{thm:main}.  The
exponential-space bound in Theorem~\ref{thm:main} is obtained by
bounding the representation of the witness set in
Proposition~\ref{prop:main} (see Section~\ref{sec:complexity}).

In the rest of this section we give a brief summary of the proof of
Proposition~\ref{prop:main}.

To compute a witness set $W$ for \textsf{P4} we first partition the
eigenvalues of the update matrix $A$ by grouping eigenvalues of equal
modulus.  Correspondingly we write $\mathbb{R}^d$ as a direct sum
$\mathbb{R}^d = V_1 \oplus \ldots \oplus V_m$, where each subspace
$V_i$ is the sum of (generalised) eigenspaces of $A$ associated to
eigenvalues of the same modulus.  Assume that $V_1$ corresponds to the
eigenvalues of maximum modulus, $V_2$ the next greatest modulus,
etc.  Then there are two main steps in the construction of $W$:
\begin{enumerate}
\item
 By analysing multiplicative relationships among eigenvalues of the
 same modulus, we show that for each subspace $V_i$ the set
 $\mathit{ENT}\cap V_i$ of eventually non-terminating initial values
 in $V_i$ is semi-algebraic.
\item
Given $\boldsymbol{v} \in \mathbb{R}^d$, we can write
$\boldsymbol{v}=\boldsymbol{v}_1+\ldots+\boldsymbol{v}_m$, with
$\boldsymbol{v}_i \in V_i$.  Using Theorem~\ref{thm:s-units} on
$S$-units, we show that if all entries of $\boldsymbol{v}$ are
algebraic numbers then the eventual non-termination of \textsf{P4} on
$\boldsymbol{v}$ is a function of its eventual non-termination on each
$\boldsymbol{v}_i$ separately.  More precisely we look for the first
$\boldsymbol{v}_i$ such that the sequence $\langle
\boldsymbol{b}^TA^n\boldsymbol{v}_i : n \in \mathbb{N}\rangle$ is infinitely often non-zero. Then \textsf{P4} is eventually non-terminating on
$\boldsymbol{v}$ if and only if it is eventually non-terminating on
$\boldsymbol{v}_i$.
\end{enumerate}

The computability of a witness set $W$ easily follows from items 1 and
2 above. Our techniques require that the update matrix in the
original linear loop \textsf{P1} either be diagonalisable or have
dimension at most $4$.  Eliminating these restrictions seems to
require solving the Ultimate Positivity Problem for linear recurrence
sequences of order greater than $5$, which in turn requires solving
hard open problems in the theory of Diophantine approximation
\cite{OW14:SODA}.
