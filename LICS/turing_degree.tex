\section{Turing-degree of MESP}
\label{sec:turing-degree-lics}

We are interested in classifying the Turing-degree of MESP\@.

\begin{theorem}
\label{thm:turing-degree-2}
  MESP is in $\Sigma_{2}$.
\end{theorem}

\begin{proof}
We recall that, by~\cref{thm:Post}, $\emptyset^{(2)}$ (that is, the second Turing jump of the empty set\footnote{Note that any computable set could have been used instead of $\emptyset$.}) is $\Sigma_{2}$-complete.

Below, we present a reduction from MESP to $\emptyset^{(2)}$.

Consider the functions
\begin{equation*}
    f_{\myvector{w}}(\myvector{t}) = {\left \| \prod \limits_{i=1}^{\lvert \myvector{w} \rvert} \exp(A_{w_{i}} t_{i}) - C \right \|}_{2}^{2}, \, \myvector{w} \in {\lbrace 1, \ldots, k \rbrace}^{\star}
\end{equation*}
and note that each $f_{\myvector{w}}$ is an exponential-polynomial which has the non-negative reals as co-domain. Clearly, this instance of MESP is positive if and only if some $f_{\myvector{w}}$ has a tangential zero.

Before proceeding, note that exponential-polynomials are closed under differentiation, and that they are computable functions.

Let $\myvector{b} : \Naturals \rightarrow {\lbrace 1, \ldots, k \rbrace}^{\star}$ be any computable surjection.

For each $n \in \Naturals$ and $\myvector{w} \in \lbrace \myvector{b}(1), \ldots, \myvector{b}(n) \rbrace$, consider the Turing machine $A_{\myvector{w}, n}$ which does the following:
for each $m \in \Naturals$, partition ${[0,n]}^{\lvert \myvector{w} \rvert}$ in an uniform grid, with mesh size $m^{-1}$, and compute the approximate value of $f_{\myvector{w}}$ with error at most $m^{-1}$ at each grid point;
if it is ever the case that all the approximate values of $f_{\myvector{w}}$ are greater than $2 L_{\myvector{w}, n} m^{-1}$ (where $L_{\myvector{w}, n}$ is an upper bound on ${\| \nabla f_{\myvector{w}} \|}\restriction_{[0,n]}$, which we can compute by using the triangle inequality and the monotonicity of the exponential function), $A_{\myvector{w}, n}$ halts. Due to the Mean Value Theorem, $A_{\myvector{w}, n}$ halts if and only if $f_{\myvector{w}} \restriction_{[0,n]}$ does not have zeroes.
Thus, the instance of MESP is positive if and only if some $A_{\myvector{w}, n}$ loops.

Now, consider the Turing machine $B$ with access to a Halting Problem oracle (that is, a $\emptyset^{(1)}$ oracle) which, for each $n \in \Naturals$ and $\myvector{w} \in \lbrace \myvector{b}(1), \ldots, \myvector{b}(n) \rbrace$, uses the oracle to decide whether $A_{\myvector{w}, n}$ halts; if that is ever not the case, $B$ halts. Finally, note that $B$ halts if and only if the MESP instance in consideration is positive.
\end{proof}

Moreover, the following result holds.

\begin{theorem}
  If Schanuel's conjecture is true, MESP is $\Sigma_{1}$-complete.
\end{theorem}

\begin{proof}
  It is clear that MESP is $\Sigma_{1}$-hard.
  Let $f_{\myvector{w}}, \myvector{w} \in {\lbrace 1, \ldots, k \rbrace}^{\star}$ be as in the proof of~\cref{thm:turing-degree-2}.
  Consider the Turing Machine $T$ which, for each $n \in \Naturals$ and $\myvector{w} \in \lbrace \myvector{b}(1), \ldots, \myvector{b}(n) \rbrace$, uses~\cref{thm:wilkie-macintyre} to decide whether $f_{\myvector{w}}$ admits a zero in the region ${[0,n]}^{\lvert \myvector{w} \rvert}$.
  Then $T$ terminates if and only if the instance of MESP under consideration is positive.
\end{proof}
