\section{Other Undecidability Results}

Consider the following generalisation of the Continuous Orbit Problem: given $k$ matrices $A_{1}, \ldots, A_{k} \in \overline{\mathbb{Q}}^{n \times n}$ and two vectors $\boldsymbol{x}, \boldsymbol{y} \in \overline{\mathbb{Q}}^{n}$, all with real coordinates, do there exist $t_{1}, \ldots, t_{k} \geq 0$ such that
\begin{equation}
\prod \limits_{i=1}^{k} \exp(A_{i} t_{i}) \boldsymbol{x} = \boldsymbol{y} ?
\end{equation}

\begin{theorem}
The Generalised Continuous Orbit Problem is undecidable.
\end{theorem}

\begin{proof}
This can be shown by reduction from MEP. Suppose we are trying to decide whether there exist $t_{1}, \ldots, t_{k}$ such that
\begin{equation*}
\prod \limits_{i=1}^{k} \exp(B_{i} t_{i}) = C .
\end{equation*}
Let $\boldsymbol{c}_{1}, \ldots, \boldsymbol{c}_{n}$ be the columns of $C$, from left to right, and let $\boldsymbol{e}_{1}, \ldots, \boldsymbol{e}_{n}$ denote the canonical basis of $\mathbb{R}^{n}$. For each $i \in \lbrace 1, \ldots, n \rbrace$, we define
\begin{equation*}
A_{i} =
\begin{pmatrix}
B_{i} && \cdots && 0 \\
\vdots && \ddots && \vdots \\
0 && \cdots && B_{i}
\end{pmatrix}
\end{equation*}
Then
\begin{equation*}
\prod \limits_{i=1}^{k} \exp(B_{i} t_{i}) = C \Leftrightarrow \prod \limits_{i=1}^{k} \exp(A_{i} t_{i}) \begin{pmatrix} \boldsymbol{e}_{1} \\ \vdots \\ \boldsymbol{e}_{n} \end{pmatrix} = \begin{pmatrix} \boldsymbol{c}_{1} \\ \vdots \\ \boldsymbol{c}_{n} \end{pmatrix} .
\end{equation*}
\end{proof}

Moreover, consider the following generalisation of the Continuous Skolem Problem: given $k$ matrices $A_{1}, \ldots, A_{k} \in \overline{\mathbb{Q}}^{n \times n}$ and two vectors $\boldsymbol{x}, \boldsymbol{y} \in \overline{\mathbb{Q}}^{n}$, all with real coordinates, do there exist $t_{1}, \ldots, t_{k} \geq 0$ such that
\begin{equation}
\boldsymbol{x}^{T} \prod \limits_{i=1}^{k} \exp(A_{i} t_{i}) \boldsymbol{y} = 0 ?
\end{equation}

\begin{theorem}
The Generalised Continuous Skolem Problem is undecidable.
\end{theorem}

\begin{proof}
This can be shown by reduction from the Generalised Continuous Orbit Problem. Suppose we are trying to decide whether there exist $t_{1}, \ldots, t_{k}$ such that
\begin{equation*}
\prod \limits_{i=1}^{k} \exp(A_{i} t_{i}) \boldsymbol{x} = \boldsymbol{y} .
\end{equation*}
Let $\boldsymbol{e}_{1}, \ldots, \boldsymbol{e}_{n}$ denote the canonical basis of $\mathbb{R}^{n}$. Moreover, let
\begin{equation*}
B_{i} = \begin{pmatrix} A_{i} && 0 \\ 0 && 0 \end{pmatrix},
\boldsymbol{u}_{j} = \begin{pmatrix} \boldsymbol{e}_{j} \\ - \boldsymbol{e}_{j} \end{pmatrix},
\boldsymbol{v} = \begin{pmatrix} \boldsymbol{x} \\ \boldsymbol{y} \end{pmatrix} .
\end{equation*}
Moreover, let
\begin{equation*}
C_{i} = \begin{pmatrix} B_{i} \otimes I + I \otimes B_{i} && \cdots && 0 \\ \vdots && \ddots && \vdots \\ 0 && \cdots && B_{i} \otimes I + I \otimes B_{i} \end{pmatrix} .
\end{equation*}
Then
\begin{align*}
&\prod \limits_{i=1}^{k} \exp(A_{i} t_{i}) \boldsymbol{x} = \boldsymbol{y} \\
\Leftrightarrow &\sum \limits_{j=1}^{n} \left( \boldsymbol{u}_{j}^{T} \prod \limits_{i=1}^{k} \exp(B_{i} t_{i}) \boldsymbol{v} \right)^{2} = 0 \\
\Leftrightarrow &\sum \limits_{j=1}^{n} \left( \left( \boldsymbol{u}_{j} \otimes \boldsymbol{u}_{j} \right) \prod \limits_{i=1}^{k} \left( \exp( B_{i} t_{i} ) \otimes \exp(B_{i} t_{i}) \right) \left( \boldsymbol{v} \otimes \boldsymbol{v} \right) \right) = 0 \\
\Leftrightarrow &\sum \limits_{j=1}^{n} \left( \left( \boldsymbol{u}_{j} \otimes \boldsymbol{u}_{j} \right) \prod \limits_{i=1}^{k} \exp\left((B_{i} \otimes I + I \otimes B_{i}) t_{i} \right) \left( \boldsymbol{v} \otimes \boldsymbol{v} \right) \right) = 0 \\
\Leftrightarrow &\begin{pmatrix} \boldsymbol{u}_{1} \otimes \boldsymbol{u}_{1} \\ \vdots \\ \boldsymbol{u}_{n} \otimes \boldsymbol{u}_{n} \end{pmatrix} \prod \limits_{i=1}^{k} \exp( C_{i} t_{i} ) \begin{pmatrix} \boldsymbol{v} \otimes \boldsymbol{v} \\ \vdots \\ \boldsymbol{v} \otimes \boldsymbol{v} \end{pmatrix} = 0 .
\end{align*}
\end{proof}