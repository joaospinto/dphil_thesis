\section{Introduction}

Reachability problems are a fundamental staple of theoretical computer
science and verification, one of the best-known examples being the
Halting Problem for Turing machines. In this paper, our motivation
originates from systems that evolve continuously subject to linear
differential equations; such objects arise in the analysis of a range
of models, including linear hybrid automata, continuous-time Markov
chains, linear dynamical systems and cyber-physical systems as they
are used in the physical sciences and engineering---see,
e.g.,~\cite{Alu15}.

More precisely, consider a system consisting of a finite number of
discrete locations (or control states), having the property that the
continuous variables of interest evolve in each location according to
some linear differential equation of the form $\dot{\myvector{x}} = A
\myvector{x}$; here $\myvector{x}$ is a vector of continuous
variables, and $A$ is a square `rate' matrix of appropriate
dimension. As is well-known, in each location the closed form solution
$\myvector{x}(t)$ to the differential equation admits a
matrix-exponential representation of the form $\myvector{x}(t) =
\exp(At)\myvector{x}(0)$. Thus if a system evolves through a series
of $k$ locations, each with rate matrix $A_i$, and spending time $t_i
\geq 0$ in each location, the overall effect on the initial continuous
configuration is given by the matrix
\begin{align*}
\prod \limits_{i=1}^{k} \exp(A_{i} t_{i}) \, ,
\end{align*}
viewed as a linear transformation on $\myvector{x}(0)$.\footnote{In
  this motivating example, we are assuming that there are no discrete
  resets of the continuous variables when transitioning between
  locations.}

A particularly interesting situation arises when the matrices $A_i$
commute; in such cases, one can show that the order in which the
locations are visited (or indeed whether they are visited only once or
several times) is immaterial, the only relevant data being the total
time spent in each location. Natural questions then arise as to what
kinds of linear transformations can thus be achieved by such systems.

\subsection{Related Work}

Consider the following problems, which can be seen as discrete analogues of the question we deal with in this paper.

\begin{definition}[Matrix Semigroup Membership Problem]
Given $k+1$ square matrices $A_{1}, \ldots, A_{k}, C$, all of the same dimension, whose entries are algebraic, does the matrix $C$ belong to the multiplicative semigroup generated by $A_{1}, \ldots, A_{k}$?
\end{definition}

\begin{definition}[Solvability of Multiplicative Matrix Equations]
Given $k+1$ square matrices $A_{1}, \ldots, A_{k}, C$, all of the same dimension, whose entries are algebraic, does the equation
\begin{align*}
\prod\limits_{i=1}^{k} A_{i}^{n_{i}} = C
\end{align*}
admit any solution $n_{1}, \ldots, n_{k} \in \Naturals$?
\end{definition}

In general, both problems have been shown to be undecidable, in
\cite{Paterson} and \cite{MEHTP}, by reductions from Post's
Correspondence Problem and Hilbert's Tenth Problem, respectively.

When the matrices $A_{1}, \ldots, A_{k}$ commute, these problems are
identical, and known to be decidable, as shown in
\cite{MultiplicativeMatrixEquations}, generalising the solution of the
matrix powering problem, shown to be decidable in \cite{KL}, and the
case with two commuting matrices, shown to be decidable in \cite{ABC}.

See \cite{HalavaSurvey} for a relevant survey, and \cite{CK05} for
some interesting related problems.

The following continuous analogue of \cite{KL}'s Orbit Problem was
shown to be decidable in \cite{Hainry}:

\begin{definition}[Continuous Orbit Problem]
Given an $n \times n$ matrix $A$ with algebraic entries and two
$n$-dimensional vectors $\myvector{x}, \myvector{y}$ with
algebraic coordinates, does there exist a non-negative real $t$ such
that $\exp(At) \myvector{x} = \myvector{y}$?
\end{definition}

The paper \cite{ContinuousOrbitIPL} simplifies the argument of
\cite{Hainry} and shows polynomial-time decidability. Moreover, a
continuous version of the Skolem-Pisot problem was dealt with in
\cite{ContinuousSkolem}, where a decidability result is presented for
some instances of the problem.

As mentioned earlier, an important motivation for our work comes from
the analysis of hybrid automata. In addition to~\cite{Alu15},
excellent background references on the topic are
\cite{HenzingerSTOC,HenzingerLICS}.

\subsection{Decision Problems}

We start by defining three decision problems that will be the main
object of study in this paper: the \emph{Matrix-Exponential Problem},
the \emph{Linear-Exponential Problem}, and the
\emph{Algebraic-Logarithmic Integer Programming} problem.

\begin{definition}
  An instance of the Matrix-Exponential Problem (MEP) consists of
  square matrices $A_{1}, \ldots, A_{k}$ and $C$, all of the same
  dimension, whose entries are real algebraic numbers.  The problem
asks to determine whether there exist real numbers
$t_1,\ldots,t_k \geq 0$ such that
\begin{align}
\label{MEP}
\prod \limits_{i=1}^{k} \exp(A_{i} t_{i}) = C \, .
\end{align}
\label{def:MEP}
\end{definition}

We will also consider a generalised version of this problem, called
the \emph{Generalised MEP}, in which the matrices $A_1,\ldots,A_k$ and
$C$ are allowed to have complex algebraic entries and in which the
input to the problem also mentions a polytope
$\mathcal{P}\subseteq\Reals^{2k}$ that is specified by linear
inequalities with real algebraic coefficients.  In the generalised problem
we seek $t_1,\ldots,t_k \in \Complex$ that satisfy (\ref{MEP}) and
such that the vector
$(\Re(t_1),\ldots,\Re(t_k),
\Im(t_1),\ldots,\Im(t_k))$ lies in $\mathcal{P}$.

In the case of commuting matrices, the Generalised Matrix-Exponential
Problem can be analysed block-wise, which leads us to the following
problem:

\begin{definition}
  An instance of the Linear-Exponential Problem (LEP) consists of a system
  of equations
\begin{align}
\label{single_eqn_form}
  \exp\left(\sum_{i \in I} \lambda_i^{(j)} t_i \right) = c_j \exp (d_j)
\quad (j \in J),
\end{align}
where $I$ and $J$ are finite index sets, the $\lambda_i^{(j)}$, $c_j$
and $d_j$ are complex algebraic constants, and the $t_i$ are complex
variables, together with a polytope
$\mathcal{P} \subseteq \Reals^{2k}$ that is specified by a system
of linear inequalities with algebraic coefficients.  The problem asks
to determine whether there exist $t_1,\ldots,t_k\in \Complex$ that
satisfy the system (\ref{single_eqn_form}) and such that
$(\Re(t_1),\ldots,\Re(t_k),\Im(t_1),\ldots,\Im(t_k))$
lies in $\mathcal{P}$.
\label{def:LEP}
\end{definition}

To establish decidability of the Linear-Exponential Problem, we reduce
it to the following
\emph{Algebraic-Logarithmic Integer Programming}
problem.  Here a \emph{linear form in logarithms of algebraic numbers}
is a number of the form
$\beta_{0} + \beta_{1} \log(\alpha_{1}) + \cdots + \beta_{m}
\log(\alpha_{m})$,
where
$\beta_{0}, \alpha_{1}, \beta_{1}, \ldots, \alpha_{m}, \beta_{m}$ are
algebraic numbers and $\log$ denotes a fixed branch of the complex
logarithm function.

% \exp(\myvector{\lambda} \cdot \myvector{t}) = c_{\myvector{\lambda}} \exp(d_{\myvector{\lambda}}), \quad \myvector{\lambda} \in \Algebraics^{n}, c_{\myvector{\lambda}}, d_{\myvector{\lambda}} \in \Algebraics
% \end{align}
% and a convex polytope $\mathcal{P} \subseteq \Reals^{2k}$ with an algebraic description, decide whether (\ref{single_eqn_form}) admits a solution $\myvector{t}$ such that $(\Re(\myvector{t}), \Im(\myvector{t})) \in \mathcal{P}$.
% \end{definition}

% In both cases, when a constraint of the form $(\Re(\myvector{t}), \Im(\myvector{t})) \in \mathcal{P}$ is imposed, where $\mathcal{P} \subseteq \Reals^{2k}$ is a convex polytope with an algebraic description, we refer to the problem as the \emph{constrained MEP/LEP}; the predicates defined above may often be combined, e.g. \emph{constrained commuting MEP}.

\begin{definition}
An instance of the Algebraic-Logarithmic Integer Programming Problem (ALIP) consists of a finite system of equations of the form
\begin{align*}
A \myvector{x} \leq \frac{1}{\pi} \myvector{b}
\end{align*}
where $A$ is an $m\times n$ matrix with real algebraic entries and
where the coordinates of $\myvector{b}$ are real linear forms in
logarithms of algebraic numbers. The problem asks to determine whether
such a system admits a solution $\myvector{x} \in \Integers^{n}$.
\end{definition}

\subsection{Paper Outline}

After introducing the main mathematical techniques that are used in
the paper, we present a reduction from the Generalised Matrix
Exponential Problem with commuting matrices to the Linear-Exponential
Problem, as well as a reduction from the Linear-Exponential Problem to
the Algebraic-Logarithmic Integer Programming Problem, before finally showing that the Algebraic-Logarithmic Integer Programming Problem is decidable. By way of hardness, we will prove that the Matrix-Exponential Problem is
undecidable (in the non-commutative case), by reduction from Hilbert's
Tenth Problem.
