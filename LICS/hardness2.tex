\subsection{Enforcing a matrix product order}

In this section, we will present a gadget matrix-exponential semigroup that can
enforce a certain partial order on the matrices reaching a particular target.
More precisely, we will exhibit fives matrices $W,X,Y,Z$ and $G$ such that any
product $G=\prod_{i=1}^pe^{A_it_i}$
where $t_i>0$ and $A_i\in\{W,X,Y,Z\}$ is such that all the ``$X$'' appear before
the ``$Y$''. Define the following matrices:

\[
\begin{array}{r@{}c@{}cr@{}c@{}cr@{}c}
W=&\begin{bmatrix}0&0&0\\0&1&0\\0&0&2\end{bmatrix},&&
X=&\begin{bmatrix}0&0&0\\0&0&1\\0&0&0\end{bmatrix},&&
G=&\begin{bmatrix}1&1&0\\0&1&1\\0&0&1\end{bmatrix},\\
Y=&\begin{bmatrix}0&1&0\\0&0&0\\0&0&0\end{bmatrix},&&
Z=&\begin{bmatrix}0&0&0\\0&-1&0\\0&0&-2\end{bmatrix}.&&&
\end{array}
\]
One easily computes the exponential of these matrices:
\[
\begin{array}{r@{}cp{.5cm}r@{}c}
e^{Wt}=&\begin{bmatrix}1&0&0\\0&e^t&0\\0&0&e^{2t}\end{bmatrix},&&
e^{Xt}=&\begin{bmatrix}1&0&0\\0&1&t\\0&0&1\end{bmatrix},\\
e^{Yt}=&\begin{bmatrix}1&t&0\\0&1&0\\0&0&1\end{bmatrix},&&
e^{Zt}=&\begin{bmatrix}1&0&0\\0&e^{-t}&0\\0&0&e^{-2t}\end{bmatrix}.
\end{array}
\]

The crux of the proof will be based on the following asymmetry between $X$ and $Y$,
which leaves the top right corner zero in one case but nonzero in the other. As we will later
observe, once the top right corner is nonzero, it cannot be cleared.
\[
e^{Xt}e^{Yu}=\begin{bmatrix}1&u&0\\0&1&t\\0&0&1\end{bmatrix},\qquad
e^{Yt}e^{Xu}=\begin{bmatrix}1&u&tu\\0&1&t\\0&0&1\end{bmatrix}.
\]

\begin{proposition}\label{prop:eq_has_forced_order}
If there exists $p\in\Naturals$, $A_i\in\{W,X,Y,Z\}$ and $t_i>0$, for $i\in\{1,\ldots,p\}$,
such that $\prod_{i=1}^pe^{A_it_i}=G$, then the product contains at least one\footnote{Note
that this is not entirely trivial because we required only positive times $t_i$ in the product.}
``$X$'' and one ``$Y$'', and all the ``$X$'' appear before the ``$Y$''.
Formally, there exists $i$ and $j$ such that $A_i=X$ and $A_j=Y$, and for
any such choice we have $i<j$.
\end{proposition}

\begin{proof}
First observe that any such product is always an upper triangular matrix with
non-negative entries (because all the matrices have non-negative entries) and positive
entries on the diagonal. Let $M$ be such a matrix, we denote its coefficients by
$0$ if they are zero and $+$ if they are positive. One easily checks that the following diagrams holds.
It should be read as follows: an arrow from $M$ to $M'$ annotated with $A$ means that any product
of a matrix of the shape of $M$ by $e^{At}$ with $t>0$ will give a matrix with the shape of $M'$.
Note that the empty product gives the identity.
\begin{center}
\begin{tikzpicture}
    \node(start) at (-1.5,0) {};
    \node(id) at (0,0) {$\begin{bmatrix}+&0&0\\0&+&0\\0&0&+\end{bmatrix}$};
    \node(cp) at (0,2) {$\begin{bmatrix}+&0&0\\0&+&+\\0&0&+\end{bmatrix}$};
    \node(acp) at (3,2) {$\begin{bmatrix}+&+&0\\0&+&+\\0&0&+\end{bmatrix}$};
    \node(ap) at (3,0) {$\begin{bmatrix}+&+&0\\0&+&0\\0&0&+\end{bmatrix}$};
    \node(abcp) at (6,2) {$\begin{bmatrix}+&+&+\\0&+&+\\0&0&+\end{bmatrix}$};
    \draw[->] (start) -- (id);
    \draw[->] (id) to node[auto] {$X$} (cp);
    \draw[->] (id) to node[auto] {$Y$} (ap);
    \draw[->] (cp) to node[auto] {$Y$} (acp);
    \draw[->] (acp) to node[auto] {$X$} (abcp);
    \draw[->] (ap) to node[auto] {$X$} (abcp);
    \draw[->,loop below] (id) to node[auto] {$W,Z$} (id);
    \draw[->,loop below] (ap) to node[auto] {$W,Z,Y$} (ap);
    \draw[->,loop above] (cp) to node[auto] {$W,Z,X$} (cp);
    \draw[->,loop above] (acp) to node[auto] {$W,Z,Y$} (acp);
    \draw[->,loop above] (abcp) to node[auto] {$W,X,Y,Z$} (abcp);
\end{tikzpicture}
\end{center}
Starting from the identity and applying the different products $e^{A_it_i}$ in the diagram,
it is clear that the only way to reach a matrix of the shape of $G$ is to have all
the ``$X$'' before ``$Y$''. Formally, by contradiction, if there were $i<j$ such
that $A_i=Y$ and $A_j=X$ then by the diagram, we would end up with a matrix where
the top right corner in nonzero, which is absurd.
\end{proof}

The previous lemma shows that this semigroup enforces a partial order on the matrices
in products that reach the matrix $G$. The next lemma shows that $G$ can indeed be reached using this
kind of products, essentially proving that $G$ belongs to this semigroup.

\begin{proposition}\label{prop:forced_order_exists}
For any $t>0$, there exists $u\geqslant0$ such that:
\[e^{Wu}e^{Xt}e^{Yt}e^{Zu}=G\quad\text{ or }\quad e^{Zu}e^{Xt}e^{Yt}e^{Wu}=G.\]
\end{proposition}

\begin{proof}
Consider the following products for an arbitrary $u\geqslant0$:
\begin{align*}
e^{Wu}e^{Xt}e^{Yt}e^{Zu}&=\begin{bmatrix}1&0&0\\0&e^u&0\\0&0&e^{2u}\end{bmatrix}
    \begin{bmatrix}1&t&0\\0&1&t\\0&0&1\end{bmatrix}
    \begin{bmatrix}1&0&0\\0&e^{-u}&0\\0&0&e^{-2u}\end{bmatrix}\\
&=\begin{bmatrix}1&0&0\\0&e^u&0\\0&0&e^{2u}\end{bmatrix}
    \begin{bmatrix}1&te^{-u}&0\\0&e^{-u}&te^{-2u}\\0&0&e^{-2u}\end{bmatrix}\\
&=\begin{bmatrix}1&te^{-u}&0\\0&1&te^{-u}\\0&0&1\end{bmatrix},
\end{align*}
and
\[
e^{Zu}e^{Xt}e^{Yt}e^{Wu}=\begin{bmatrix}1&te^{u}&0\\0&1&te^{u}\\0&0&1\end{bmatrix}.
\]
If $t\geqslant1$ then choosing $u=\ln t\geqslant0$ in the first product gives $G$,
otherwise choosing $u=\ln\frac{1}{t}\geqslant0$ in the second product gives $G$.
\end{proof}

\subsection{Undecidability of the semigroup problem}

We consider the following natural variant of MEP where the order of the matrices
in the product is not fixed and matrices can be used more than once.

\begin{definition}
  An instance of the Matrix-Exponential Semigroup Problem (MESP) consists of
  square matrices $A_{1}, \ldots, A_{k}$ and $C$, all of the same
  dimension, whose entries are real algebraic numbers.  The problem
 asks to determine whether $C$ a member of the matrix semigroup generated by
\begin{align*}
\lbrace \exp(A_{i} t_{i}) : t_{i} \geq 0 , i=1,\ldots,k \rbrace.
\end{align*}
\end{definition}

In the case where the matrices commute, MESP is equivalent to MEP. Will show that
if the matrices do not commute, this problem is undecidable by reducing from MEP,
which is undecidable in the non-commuting case.

\begin{theorem}
  MESP is undecidable in the non-commutative case.
\end{theorem}

\begin{proof}
We have seen in the previous section that MEP is undecidable in the non-commutative
case. Thus it suffices to reduce MEP to MESP to show undecidability.

Let $A_1,\ldots,A_k,C\in\Algebraics^{n\times n}$ be an instance of MEP. Denote by $I_m$ the identity of size $m$,
$0_m$ the zero matrix of size $m$. Recall that $W,X,Y,Z$ and $G$ were defined in the previous
subsection. For every $i\in\{2,\ldots,k-1\}$, define the following matrices:
\begin{align*}
B_i&=\begin{bmatrix}A_i&&&&\\&0_{3(i-2)}&&&\\&&Y&&\\&&&X&\\&&&&0_{3(k-1-i)}\end{bmatrix},\\
B_i'&=\begin{bmatrix}0_p&&&&\\&0_{3(i-2)}&&&\\&&Y&&\\&&&X&\\&&&&0_{3(k-1-i)}\end{bmatrix}.\\
\end{align*}
Also define the following matrices:
\[
\begin{array}{r@{}c@{}cr@{}c}
B_1=&\begin{bmatrix}A_1&&\\&X&\\&&0_{3(k-2)}\end{bmatrix},&&
B_1'=&\begin{bmatrix}0_p&&\\&X&\\&&0_{3(k-2)}\end{bmatrix},\\
B_k=&\begin{bmatrix}A_1&&\\&0_{3(k-2)}&\\&&Y\end{bmatrix},&&
B_k'=&\begin{bmatrix}0_p&&\\&0_{3(k-2)}&\\&&Y\end{bmatrix}.\\
\end{array}\]
Also define for every $i\in\{1,\ldots,k-1\}$:
\begin{align*}
W_i&=\begin{bmatrix}0_p&&&\\&0_{3(i-1)}&&\\&&W&\\&&&0_{3(k-1-i)}\end{bmatrix},\\
Z_i&=\begin{bmatrix}0_p&&&\\&0_{3(i-1)}&&\\&&Z&\\&&&0_{3(k-1-i)}\end{bmatrix}.\\
\end{align*}
And finally define the target matrix:
\[C'=\begin{bmatrix}C&&&\\&G&&\\&&\ddots&\\&&&G\end{bmatrix}.\]
We can now define our MESP instance as follows, the target matrix is $C'$ and the semigroup $\mathcal{G}$
is generated by:
\begin{align*}
&\left\{e^{B_it_i},e^{B_i't_i}:t_i\geqslant0,i=1,\ldots,k\right\}\\
&\cup\left\{e^{W_it_i},e^{Z_it_i}:t_i\geqslant0,i=1,\ldots,k-1\right\}.
\end{align*}
We claim the original instance of MEP is satisfiable if and only if $C'\in\mathcal{G}$.
Let us examine both direction independently.

Assume that the MEP instance is satisfiable. Then there exists $t_1,\ldots,t_k\geqslant0$ such that:
\[\prod_{i=1}^ke^{A_it_i}=C.\]
Define $\tau=\max\{t_1,\ldots,t_k\}+1$ (note that $\tau>0$) and $t_i'=\tau-t_i\geqslant0$ for every $i\in\{1,\ldots,k\}$.
A straightforward calculation shows that:
\begin{align*}
\prod_{i=1}^k\left(e^{B_it_i}e^{B_i't_i'}\right)
    &=\begin{bmatrix}\prod_{i=1}^ke^{A_it_i}&&&\\&e^{X\tau}e^{Y\tau}&&\\&&\ddots&\\&&&e^{X\tau}e^{Y\tau}\end{bmatrix}\\
    &=\begin{bmatrix}C&&&\\&U&&\\&&\ddots&\\&&&U\end{bmatrix},
\end{align*}
where $U=e^{X\tau}e^{Y\tau}$. Apply \cref{prop:forced_order_exists} to get $\lambda\geqslant0$
such that either $e^{W\lambda}Ue^{Z\lambda}=G$ or $e^{Z\lambda}Ue^{W\lambda}=G$. In the first case, conclude by
checking that:
\[
\prod_{i=1}^{k-1}e^{W_i\lambda}\prod_{i=1}^k\left(e^{B_it_i}e^{B_i't_i'}\right)\prod_{i=1}^{k-1}e^{Z_i\lambda}
    =\begin{bmatrix}C&&&\\&G&&\\&&\ddots&\\&&&G\end{bmatrix}=C'
\]
In the second case, exchange the $W_i$ and $Z_i$ to get same result. This concludes the proof that
the MESP instance is satisfiable, since all the products belong to $\mathcal{G}$.

Assume that the MESP instance is satisfiable. Then there exists $t_1,\ldots,t_m>0$ (we can always take them positive)
and $M_1,\ldots,M_m\in\big\{B_i,B_i':i=1,\ldots,k\}\cup\big\{W_i,Z_i:i=1,\ldots,k-1\big\}$
such that:
\begin{equation}\label{eq:mesp_to_mep:prod_eq}
\prod_{j=1}^me^{M_jt_j}=C'.
\end{equation}
Observe that by construction, this product has the following form:
\[\prod_{j=1}^me^{M_jt_j}=\begin{bmatrix}V&&&\\&U_1&&\\&&\ddots&\\&&&U_{k-1}\end{bmatrix},\]
where $V$ belongs to the semigroup generated by $\{e^{A_it}:t\geqslant0\}$ and $U_i$ belongs
to the semigroup generated by $\{e^{Wt},e^{Xt},e^{Yt},e^{Zt}:t\geqslant0\}$. Since
\eqref{eq:mesp_to_mep:prod_eq} implies that $U_i=G$, we can apply \cref{prop:eq_has_forced_order}
to get each product producing $U_i$ must have all its ``$X$'' before its ``$Y$''. Furthermore, each $U_i$
must contain at least one $X$ and one $Y$ in its product. For any $i\in\{1,\ldots,k\}$,
let $k_i$ (resp. $k_i'$) denote
the first (resp. last) index $j$ such that $M_j=B_i\text{ or }B_i'$. Those indices exists because
of the proposition since at least one $B_i$ or $B_i$' must appear for every $i$
to get both an $X$ and a $Y$ in each product giving $U_i$. Obviously $k_i\leqslant k_i'$
by definition. We now claim that the proposition implies that:
\[k_1\leqslant k_1'<k_2\leqslant k_2'<k_3\cdots<k_{k-1}\leqslant k_{k-1}'.\]
Indeed, \cref{prop:eq_has_forced_order} ensures that in the product giving $U_1$,
all the ``$X$'' appear before the ``$Y$'',
but the only matrices that contribute some $X$ to $U_1$ are $B_1$ and $B_1'$,
and the only matrices that contribute some $Y$ to $U_1$ are $B_2$ and $B_2'$.
Thus $k_1'<k_2$, \emph{i.e.} the last ``$X$'' coming from $B_1$ or $B_1'$ is before the first ``$Y$''
coming from $B_2$ or $B_2'$. A similar reasoning ensures that $k_2'<k_3$ and so on.
This shows that for any $i\in\{1,\ldots,k\}$, if $M_j=B_i$ then $j\in\{k_i,\ldots,k_i'\}$. Thus all
the $B_1$ appear before the $B_2$ which appear before the $B_3$ and so on.
But since the $B_i$ are the only one to contribute to $V$, $V$ must be of the form:
\[V=\prod_{i=1}^ke^{A_it_i'},\]
where $t_i'\geqslant0$ is the sum of all $t_j$ such that $M_j=B_i$.
Finally $V=C$, so the instance of MEP is satisfiable.

\end{proof}