\section{Conclusion}

We have shown that the Matrix-Exponential Problem is undecidable in general, but decidable when the matrices $A_{1}, \ldots, A_{k}$ commute. We have also showed that the Matrix-Exponential Semigroup Problem is undecidable, by designing a gadget to enforce an order in the products, and derived the undecidability of the generalised versions of the Continuous Orbit and Continuous Skolem problems to a multi-matrix setting. This is analogous to what was known for the discrete version of this problem, in which the matrix exponentials $e^{At}$ are replaced by matrix powers $A^n$. Finally, we show that both these problems are Turing-equivalent, by showing how they can be decided with an oracle to the set of halting Turing machines. Whilst the latter is obvious in the discrete variant of these problems, due to the countability of the set of candidate solutions $\Integers^{d}$, the same cannot be said about this result.

It would be interesting to look at possibly decidable restrictions of the MEP/MESP, for example the case where $k=2$ with a non-commuting pair of matrices, which was shown to be decidable for the discrete analogue of this problem in~\cite{MEHTP}. Bounding the dimension of the ambient vector space could also yield decidability, which has been partly accomplished in the discrete case in~\cite{PS2Z}.

Finally, deriving upper and lower bounds for the computational complexity of the commutative case of this problem would also be a worthwhile task.
