\chapter{Introduction}
\label{sec:introduction}

Dynamical systems have long been of interest to computer scientists. Many problems related to the behaviour of such systems, including the dynamics of polynomial differential equations, finitely generated matrix semigroups, or cellular automata, have been shown to be Turing-complete. However, despite the fact that some of these topics were studied over half a century ago, a surprising number of problems remain open, and we tackle some of them in this thesis.

In particular, we study dynamical systems with an algebraic nature, where the state space is typically a set of vectors or a set of matrices, and where the evolution of the system is determined by applying a \emph{linear} operator to its state. A common property that is often studied is whether its state may ever reach a certain target. These systems are often described as \emph{discrete}- or \emph{continuous-time} depending on the nature of their evolution rule.

On the discrete-time front, \emph{linear recurrence sequences} have been exhaustively studied. These correspond to sequences where each term is a fixed linear combination of the previous $k$ terms ($k$ is said to be the \emph{order} of the linear recurrence sequence). The Skolem-Mahler-Lech theorem~\cite{Sko34,Mah35,Lec53,Hansel85} characterises the set of zeroes of such sequences as the union of finitely many arithmetic progressions and a finite set. Whilst it is known how to compute these
arithmetic progressions~\cite{BM76}, the general problem of determining whether a linear recurrence sequence ever hits zero, known as the \emph{Skolem Problem}, remains open, and has been conjectured to be decidable. It has been shown that, when~$k \leq 4$, the conjecture does indeed hold~\cite{Ver85}, but the problem remains open for~$k\geq 5$.
Furthermore, other properties of these sequences, such as the Positivity Problem~\cite{Bell2007,HHH06,LT09,OW13:constructive-positivity,OW14:SODA,Liu10} (which amounts to deciding whether a given linear recurrence sequence is always non-negative) and the Ultimate Positivity Problem~\cite{OuaknineW13b}, have been studied.
In particular, these problems are known to be decidable when $k \leq 5$, and have shown to be hard when $k \geq 6$, in the sense that their decidability would entail a substantial breakthrough in analytic number theory. Note that the Positivity Problem is at least as hard as Skolem's Problem~\cite{OW14:SODA}, although the reduction causes a quadratic increase in the order of the sequence. Some of these results will be put to use in~\cref{chapter:soda}, and so we defer further details until then.

Kannan and Lipton's \emph{Orbit Problem}~\cite{KL86}, which consists in deciding whether an orbit of the form ${(A^{n} \myvector{x})}_{n \in \Naturals}$ ever hits a target point $\myvector{y}$, was shown to be in~\PTIME{} in~1986. Further generalisations where the target is replaced by a small-dimension linear subspace~\cite{COW13} or a convex polytope~\cite{COW15:SODA} in a small-dimensional space have been shown to be decidable, but the general instances of these problems are, respectively, Skolem- and Positivity-hard.

As early as~1947, Markov showed in~\cite[in Russian]{Markov1947} that the membership problem for finitely generated matrix sub-semigroups of $\Integers^{6 \times 6}$ is undecidable.
In~1966, Mikhailova showed in~\cite[in Russian]{Mik66} that this problem is already undecidable for matrices in $SL(4,\Integers)$ (that is, $4\times 4$ integer matrices of determinant $1$).
In~1970, Paterson established the undecidability of testing whether the zero matrix belongs to a given finitely generated sub-semigroup of $3 \times 3$ integer matrices~\cite{Paterson}.
The decidability of the membership problem for finitely generated semigroups of $2 \times 2$ integer matrices was only established in 2016, by Potapov and Semukhin in~\cite{PS2Z}, and the case with \emph{invertible} $3\times 3$ integer matrices remains open.
Moreover, the membership problem for finitely generated matrix semigroups of \emph{commuting} integer matrices was shown to be decidable in any dimension by Babai, Beals, Cai, Ivanyos, and Luks in~\cite{MultiplicativeMatrixEquations}.

Another interesting problem relates to the study of matrix equations of the form
\begin{equation*}
  \prod\limits_{i=1}^{k} A_{i}^{n_{i}} = C \, ,
\end{equation*}
which were studied in~\cite{MEHTP}, where it was shown to be undecidable whether they have any solution. This problem is of a very similar nature to the membership problem for finitely generated matrix semigroups, but here an order in the matrix products is enforced. Note that this undecidability result was established by reducing from Hilbert's Tenth Problem, as opposed to the Post Correspondence Problem, in similarity to what will frequently happen throughout this dissertation. In fact, we will need a strengthened version of this result where the matrices $A_{1}, \ldots, A_{k}, C$ are required to be invertible, as we shall see in~\cref{sec:matrix-undecidability}.

On a different note, the study of \emph{continous} models of computation started as early as 1941, when Shannon studied the General Purpose Analog Computer (GPAC).
In particular, he showed that the class of GPAC-computable functions corresponds to the set of functions which are components of a solution to an ordinary differential equation with a polynomial right-hand side~\cite{Shannon1941}.
Modulo technicalities, GPAC-computable functions have been shown to correspond to Turing-computable functions~\cite{Bournez1,Bournez2}, and even a neat characterisation of \PTIME{} is known.
Therefore, it is unsurprising that researchers have focused on linear ordinary differential equations, where numerous decidability results have been achieved.

The \emph{Continuous Orbit Problem}, which amounts to deciding whether the unique solution of a given linear ordinary differential equation ever reaches a given target point, has been shown to be decidable in polynomial time~\cite{Hainry08,ContinuousOrbitIPL}.
Some work has also been done on the \emph{Continuous Skolem Problem}, which analogously to the Orbit Problem asks whether such a trajectory hits a given hyperplane (instead of a point).
In~\cite{ContinuousSkolem}, this problem was shown to be decidable when the ambient vector space has dimension~$2$. The dimension~$3$ case was shown to be decidable in~\cite{VCthesis}, where it was also shown that, if Schanuel's conjecture is true, then all cases up to dimension~$7$ are decidable. The dimension~$9$ case was also shown to be hard in a number-theoretic sense; in particular, decidability would entail a major breakthrough in analytic number theory.
Moreover, the version of this problem with a bounded time-horizon was shown to be decidable in~\cite{ContinuousSkolem3} conditionally on Schanuel's conjecture being true.

Many other similar problems have been studied, and the reader should refer to the cited papers for further information, as well as the chapter-specific introductions.
