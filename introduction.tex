%BEGIN SODA
\chapter{Introduction}
\label{sec:introduction}
Termination is a fundamental decision problem in program verification.
In particular, termination of programs with linear assignments and
linear conditionals has been extensively studied over the last decade.
This has led to the development of powerful techniques to
prove termination via synthesis of linear ranking
functions~\cite{Ben-AmramG13,BradleyMS05,ChenFM12,ColonS01,PodelskiR04},
many of which have been implemented in software-verification tools, such as
Microsoft's \textsc{Terminator}~\cite{CookPR06}.

A very simple form of linear programs are \emph{simple linear
  loops}, that is, programs of the form
\begin{gather*}
\mathsf{P1:}\  \mbox{$\boldsymbol{x}\gets \boldsymbol{u}$ ;
\textit{while} $B\boldsymbol{x} \geq \boldsymbol{c}$ \textit{do}
$\boldsymbol{x}\leftarrow A\boldsymbol{x}+\boldsymbol{a}$,}
\end{gather*}
where $\boldsymbol{x}$ is vector of variables, $\boldsymbol{u}$,
$\boldsymbol{a}$, and $\boldsymbol{c}$ are integer vectors, and $A$
and $B$ are integer matrices of the appropriate dimensions.  Here the
loop guard is a conjunction of linear inequalities and the loop body
consists of a simultaneous affine assignment to $\boldsymbol{x}$.  If
the vectors $\boldsymbol{a}$ and $\boldsymbol{c}$ are both zero then
we say that the loop is \emph{homogeneous}.

Suppose that the vector $\boldsymbol{x}$ has dimension $d$.  We say
that \textsf{P1} \emph{terminates} on a set $S\subseteq \mathbb{R}^d$
if it terminates for all initial vectors $\boldsymbol{u} \in S$.
Tiwari~\cite{Tiw04} gave a procedure to decide whether a given simple
linear loop terminates on $\mathbb{R}^d$.  Later
Braverman~\cite{Bra06} showed decidability of termination on
$\mathbb{Q}^d$.  However the most natural problem from the point of
view of program verification is termination on $\mathbb{Z}^d$.

While termination on $\mathbb{Z}^d$ reduces to termination on
$\mathbb{Q}^d$ in the homogeneous case (by a straightforward scaling
argument), termination on $\mathbb{Z}^d$ in the general case is stated
as an open problem in~\cite{BGM12,Bra06,Tiw04}.  The main result of
this paper is a procedure to decide termination on $\mathbb{Z}^d$ for
simple linear loops when the assignment matrix $A$ is diagonalisable.
This represents the first substantial progress on this open problem in
over $10$ years.

Termination of more complex linear programs can often be reduced to
termination of simple linear loops (see, e.g.,~\cite{CookPR06}
or~\cite[Section 6]{Tiw04}).  On the other hand, termination becomes
undecidable for mild generalisations of simple linear loops, for
example, allowing the update function in the loop body to be piecewise
linear~\cite{BGM12}.

To prove our main result we focus on \emph{eventual non-termination},
where \textsf{P1} is said to be eventually non-terminating on
$\boldsymbol{u} \in \mathbb{Z}^d$ if, starting from initial value
$\boldsymbol{u}$, after executing the loop body $\boldsymbol{x} \gets
A\boldsymbol{x}+\boldsymbol{a}$ a finite number of times \emph{while
  disregarding the loop guard} we eventually reach a value on which
\textsf{P1} fails to terminate.  Clearly \textsf{P1} fails to
terminate on $\mathbb{Z}^d$ if and only if it is eventually
non-terminating on some $\boldsymbol{u} \in \mathbb{Z}^d$.

Given a simple linear loop we show how to compute a convex
semi-algebraic set $W \subseteq \mathbb{R}^d$ such that the integer
points $\boldsymbol{u} \in W$ are precisely the eventually
non-terminating integer initial values.  Since it is decidable whether
a convex semi-algebraic set contains an integer
point~\cite{KhachiyanP97},\footnote{By contrast, recall that the
  existence of an integer point in an \emph{arbitrary} (i.e., not
  necessarily convex) semi-algebraic set---which is equivalent to
  Hilbert's tenth problem---is well-known to be undecidable.} we can
decide whether an integer linear loop is terminating on
$\mathbb{Z}^d$.

Termination over the set of all integer points is easily seen to be
\textbf{coNP}-hard.  Indeed, if the update function in the loop body
is the identity then the loop is non-terminating if and only if there
is an integer point satisfying the guard.  Thus non-termination
subsumes integer programming, which is \textbf{NP}-hard.  By contrast,
even though not stated explicitly in~\cite{Tiw04} and~\cite{Bra06},
deciding termination on $\mathbb{R}^d$ and $\mathbb{Q}^d$ can be done
in polynomial time.\footnote{This observation relies on the facts that
  one can compute Jordan canonical forms of integer matrices and solve
  instances of linear programming problems with algebraic numbers in
  polynomial time~\cite{Cai94,AdlerB94}.}

While our algorithm for deciding termination requires exponential
space, it should be noted that the procedure actually solves a more
general problem than merely determining the existence of a
non-terminating integer point (or, equivalently, the existence of an
eventually non-terminating integer point).  In fact the algorithm
computes a representation of the set of all eventually non-terminating
integer points.  For reference, the closely related problem of
deciding termination on the integer points in a given convex polytope is
\textbf{EXPSPACE}-hard \cite{BGM12}.

As well as making extensive use of algorithms in real algebraic
geometry, the soundness of our decision procedure relies on powerful
lower bounds in Diophantine approximation that generalise Roth's
Theorem.  (The need for such bounds in the inhomogeneous setting was
conjectured in the discussion in the conclusion of~\cite{Bra06}.)  We
also use classical results in number theory, such as the
Skolem-Mahler-Lech Theorem~\cite{Lec53,Mah35,Sko34} on linear
recurrences.  Crucially the well-known and notorious ineffectiveness
of Roth's Theorem (and its higher-dimensional and $p$-adic
generalisations) and of the Skolem-Mahler-Lech Theorem are not a
problem for deciding \emph{eventual} non-termination, which is key to
our approach.

\section{Related Work}

Consider the termination problem for a homogeneous linear loop program
\begin{gather*}
\mathsf{P2:}\  \mbox{$\boldsymbol{x}\gets \boldsymbol{u}$ ;
\textit{while} $B\boldsymbol{x} \geq 0$ \textit{do} $\boldsymbol{x}\leftarrow A\boldsymbol{x}$}
\end{gather*}
on a single initial value $\boldsymbol{u}\in \mathbb{Z}^d$.  Each row
$\boldsymbol{b}^T$ of matrix $B$ corresponds to a loop condition
$\boldsymbol{b}^T\boldsymbol{x} \geq 0$.  For each such condition,
consider the integer sequence $\langle x_n : n \in \mathbb{N} \rangle$
defined by $x_n = \boldsymbol{b}^TA^n \boldsymbol{u}$.  Then
\textsf{P2} fails to terminate on an initial value $\boldsymbol{u}$ if
and only if each such sequence $\langle x_n \rangle$ is
\emph{positive}, i.e., $x_n \geq 0$ for all $n$.  It is not difficult
to show that each sequence $\langle x_n \rangle$ considered above is a
\emph{linear recurrence sequence}, thanks to the Cayley-Hamilton
theorem.  Thus deciding whether a homogeneous linear loop program
terminates on a given initial value is at least as hard as the
\emph{Positivity Problem} for linear recurrence sequences, that is,
the problem of deciding whether a given linear recurrence sequence has
exclusively non-negative terms.

The Positivity Problem has been studied at least as far back as the
1970s~\cite{BG07,HHH06,Liu10,RS94,Sal76}.  Thus far decidability is
known only for sequences satisfying recurrences of order $5$ or less.
It is moreover known that showing decidability at order $6$ will
necessarily entail breakthroughs in transcendental number theory,
specifically significant new results in Diophantine
approximation~\cite{OW14:SODA}.

The key difference between studying termination of simple linear loops
over $\mathbb{Z}^d$ rather than a single initial value is that the
former problem can be approached through eventual termination.  In
this sense the termination problem is related to the \emph{Ultimate
  Positivity Problem} for linear recurrence sequences, which asks
whether all but finitely many terms of a given sequence are
positive~\cite{OuaknineW14a}.  This allows us to bring to bear powerful
non-effective Diophantine-approximation techniques,
specifically the $S$-units Theorem of Evertse, van der Poorten, and
Schlickewei~\cite{Evertse84,PS82}.  Such tools enable us to obtain
decidability of termination for matrices of arbitrary dimension,
assuming diagonalisability.

The paper~\cite{COW14:SODA} studies higher dimensional versions of
Kannan and Lipton's Orbit Problem~\cite{KL86}.  These can be seen as
versions of the termination problem for linear loops on a fixed
initial value.  That work uses substantially different technology from
that of the current paper, including Baker's Theorem on linear forms
in logarithms~\cite{BW93}, and correspondingly relies on restrictions
on the dimension of data in problem instances to obtain decidability.

%Since Kannan and Lipton solved the \textbf{Orbit Problem} \cite{KL86},
%which consists in deciding whether $\exists n\in\mathbb{N}: A^n x=y$
%for given $x,y\in\mathbb{Q}^n,A\in\mathbb{Q}^{n\times n}$, a lot of
%work has been done in related problems. For example, higher
%dimensional versions of the Orbit Problem have been considered, namely
%replacing the target point $y$ by some target subspace $V$
%\cite{COW13}. Whenever $dim(V)\leq 3$, this problem is in
%\textbf{PSPACE}. On the other hand, when $V$ is a hyperplane of the
%embedding vector space, this corresponds to the famous
%\textbf{Skolem's problem}, which can equivalently be stated as that of
%finding whether the set $Z(u_n)=\lbrace n\in\mathbb{N}\mid
%u_n=0\rbrace$ is empty for a given linear recurrence sequence (LRS)
%$u_n$. The celebrated Skolem-Mahler-Lech theorem
%\cite{Lec53,Mah35,Sko34,TUCS05,Hansel85} characterises $Z(u_n)$ as the
%union of a finite set and finitely many (effectively computable
%\cite{BM76}) arithmetic progressions. Blondel and Tsitsiklis
%\cite{BlondelT00} remark that ``the present consensus among number
%theorists is that an algorithm [for Skolem's problem] should exist'',
%even though such algorithms are only known for LRS of order up to $4$,
%as proved independently by Vereschagin \cite{Ver85} and Mignotte,
%Shorey, and Tijdeman \cite{MST84}. This problem has been discussed at
%length by Terence Tao in \cite{Tao07} and by Lipton in
%\cite{Lip09}. Deciding the \textbf{positivity} problem (whether
%$\forall n\geq 0,u_n\geq 0$) and the \textbf{ultimate positivity}
%problem (whether $\lbrace n\in\mathbb{N}\mid u_n< 0\rbrace$ is finite)
%is also only known to be possible for LRS of order up to 5 in general
%\cite{OW14:SODA} and up to 9 for a relevant subclass
%\cite{OW13:constructive-positivity}. In fact, there is a one-line
%reduction from Skolem's problem to the positivity problem which does
%not, however, preserve orders. Ouaknine and Worrell
%\cite{OuaknineW13b} showed that ultimate positivity is decidable in
%polynomial space (and in polynomial time for a fixed order) for all
%simple LRS (those whose characteristic polynomial has no repeated
%roots), and we borrow many of the techniques used there. Finally, the
%Polyhedron Hitting Problem, where one replaces the target subspace $V$
%of the generalised Orbit Problem by a polytope $\mathcal{P}$, has
%recently seen some interesting progress.

%Even though the results of Tiwari \cite{Tiw04} and Braverman \cite{Bra06} rely almost exclusively on elementary linear algebra and linear programming, stronger results from both analytic and algebraic number theory tend to be necessary to tackle all the other problems we discussed, namely those involving linear recurrence sequences. In fact, the same holds for this paper, as conjectured to be needed by Braverman in \cite{Bra06}. Moreover, even though the procedures for solving the universal termination problem over $\mathbb{R}$ and $\mathbb{Q}$ run in polynomial time (requiring some non-trivial subroutines such as computing Jordan Canonical Forms \cite{Cai94} and solving Linear Programming problems with algebraic numbers \cite{AdlerB94}\footnote{In order to find $z_{max}$ and $index_{z_{max}}$, we can simply determine $index_v$ and $index_{-v}$ for each eigenvector $v$ associated with a positive real eigenvalue and take $index_{z_{max}}$ to be the maximum of these $index_v$ and $index_{-v}$, after which we only have to solve a linear programming that is guaranteed to have a solution in order to determine $z_{max}$}), it is easy to see that deciding the existence of a non-terminating point becomes \textbf{NP}-hard over the integers, by a trivial reduction from the problem of deciding the existence of an integer point in a polytope. Such a hardness result suggests that this problem is thus intrinsically different from Tiwari's or Braverman's, therefore further emphasising our point.

%Braverman conjectured in \cite{Bra06} that one would need to decide whether, for a given $x\in\mathbb{Z}^k$, $\forall n\in\mathbb{N},f^{n}(x)\in\mathcal{P}$ (that is, point-wise termination, which is equivalent to deciding the positivity problem) in order to decide universal termination. As it turns out, we only needed to be able to decide whether, for a given $x\in\mathbb{Z}^k$, $\exists N\in\mathbb{N}:\forall n\geq N,f^{n}(x)\in\mathcal{P}$ (that is, eventual pointwise termination, which is equivalent to deciding the ultimate positivity problem).

Termination of $\mathsf{P1}$ under the assumption that all eigenvalues
of $A$ are real was studied in \cite{RMM,RMM1} using spectral
techniques. However, as will become clear throughout the course of
this paper, most of the machinery that we use is needed to tackle the
case where there are both real and complex eigenvalues with the same
absolute value. In the setting of \cite{RMM,RMM1}, the set of
eventually non-terminating points is in fact a polytope, which can be
effectively computed resorting only to straightforward linear algebra.

While we use spectral and number-theoretic techniques in this paper,
another well-studied approach for proving termination of linear loops
involves designing linear ranking functions, that is, linear functions
from the state space to a well-founded domain such that each iteration
of the loop strictly decreases the value of the ranking
function. However, this approach is incomplete: it is not hard to
construct an example of a terminating loop which admits no linear
ranking function.  Sound and relatively complete methods for
synthesising linear ranking functions can be found in
\cite{PodelskiR04} and \cite{Ben-AmramG13}. Whether a linear
ranking function exists can be decided in polynomial time when the
state space is $\mathbb{Q}^d$ and is \textbf{coNP}-complete when the
state space is $\mathbb{Z}^d$.

%END SODA

%BEGIN LICS
\section{Introduction}

Reachability problems are a fundamental staple of theoretical computer
science and verification, one of the best-known examples being the
Halting Problem for Turing machines. In this paper, our motivation
originates from systems that evolve continuously subject to linear
differential equations; such objects arise in the analysis of a range
of models, including linear hybrid automata, continuous-time Markov
chains, linear dynamical systems and cyber-physical systems as they
are used in the physical sciences and engineering---see,
e.g.,~\cite{Alu15}.

More precisely, consider a system consisting of a finite number of
discrete locations (or control states), having the property that the
continuous variables of interest evolve in each location according to
some linear differential equation of the form $\dot{\boldsymbol{x}} = A
\boldsymbol{x}$; here $\boldsymbol{x}$ is a vector of continuous
variables, and $A$ is a square `rate' matrix of appropriate
dimension. As is well-known, in each location the closed form solution
$\boldsymbol{x}(t)$ to the differential equation admits a
matrix-exponential representation of the form $\boldsymbol{x}(t) =
\exp(At)\boldsymbol{x}(0)$. Thus if a system evolves through a series
of $k$ locations, each with rate matrix $A_i$, and spending time $t_i
\geq 0$ in each location, the overall effect on the initial continuous
configuration is given by the matrix
\begin{align*}
\prod \limits_{i=1}^{k} \exp(A_{i} t_{i}) \, ,
\end{align*}
viewed as a linear transformation on $\boldsymbol{x}(0)$.\footnote{In
  this motivating example, we are assuming that there are no discrete
  resets of the continuous variables when transitioning between
  locations.}

A particularly interesting situation arises when the matrices $A_i$
commute; in such cases, one can show that the order in which the
locations are visited (or indeed whether they are visited only once or
several times) is immaterial, the only relevant data being the total
time spent in each location. Natural questions then arise as to what
kinds of linear transformations can thus be achieved by such systems.

\subsection{Related Work}

Consider the following problems, which can be seen as discrete analogues of the question we deal with in this paper.

\begin{definition}[Matrix Semigroup Membership Problem]
Given $k+1$ square matrices $A_{1}, \ldots, A_{k}, C$, all of the same dimension, whose entries are algebraic, does the matrix $C$ belong to the multiplicative semigroup generated by $A_{1}, \ldots, A_{k}$?
\end{definition}

\begin{definition}[Solvability of Multiplicative Matrix Equations]
Given $k+1$ square matrices $A_{1}, \ldots, A_{k}, C$, all of the same dimension, whose entries are algebraic, does the equation
\begin{align*}
\prod\limits_{i=1}^{k} A_{i}^{n_{i}} = C
\end{align*}
admit any solution $n_{1}, \ldots, n_{k} \in \mathbb{N}$?
\end{definition}

In general, both problems have been shown to be undecidable, in
\cite{Paterson} and \cite{MEHTP}, by reductions from Post's
Correspondence Problem and Hilbert's Tenth Problem, respectively.

When the matrices $A_{1}, \ldots, A_{k}$ commute, these problems are
identical, and known to be decidable, as shown in
\cite{MultiplicativeMatrixEquations}, generalising the solution of the
matrix powering problem, shown to be decidable in \cite{KL}, and the
case with two commuting matrices, shown to be decidable in \cite{ABC}.

See \cite{HalavaSurvey} for a relevant survey, and \cite{CK05} for
some interesting related problems.

The following continuous analogue of \cite{KL}'s Orbit Problem was
shown to be decidable in \cite{Hainry}:

\begin{definition}[Continuous Orbit Problem]
Given an $n \times n$ matrix $A$ with algebraic entries and two
$n$-dimensional vectors $\boldsymbol{x}, \boldsymbol{y}$ with
algebraic coordinates, does there exist a non-negative real $t$ such
that $\exp(At) \boldsymbol{x} = \boldsymbol{y}$?
\end{definition}

The paper \cite{ContinuousOrbitIPL} simplifies the argument of
\cite{Hainry} and shows polynomial-time decidability. Moreover, a
continuous version of the Skolem-Pisot problem was dealt with in
\cite{ContinuousSkolem}, where a decidability result is presented for
some instances of the problem.

As mentioned earlier, an important motivation for our work comes from
the analysis of hybrid automata. In addition to~\cite{Alu15},
excellent background references on the topic are
\cite{HenzingerSTOC,HenzingerLICS}.

\subsection{Decision Problems}

We start by defining three decision problems that will be the main
object of study in this paper: the \emph{Matrix-Exponential Problem},
the \emph{Linear-Exponential Problem}, and the
\emph{Algebraic-Logarithmic Integer Programming} problem.

\begin{definition}
  An instance of the Matrix-Exponential Problem (MEP) consists of
  square matrices $A_{1}, \ldots, A_{k}$ and $C$, all of the same
  dimension, whose entries are real algebraic numbers.  The problem
asks to determine whether there exist real numbers
$t_1,\ldots,t_k \geq 0$ such that
\begin{align}
\label{MEP}
\prod \limits_{i=1}^{k} \exp(A_{i} t_{i}) = C \, .
\end{align}
\label{def:MEP}
\end{definition}

We will also consider a generalised version of this problem, called
the \emph{Generalised MEP}, in which the matrices $A_1,\ldots,A_k$ and
$C$ are allowed to have complex algebraic entries and in which the
input to the problem also mentions a polyhedron
$\mathcal{P}\subseteq\mathbb{R}^{2k}$ that is specified by linear
inequalities with real algebraic coefficients.  In the generalised problem
we seek $t_1,\ldots,t_k \in \mathbb{C}$ that satisfy (\ref{MEP}) and
such that the vector
$(\Re(t_1),\ldots,\Re(t_k),
\Im(t_1),\ldots,\Im(t_k))$ lies in $\mathcal{P}$.

In the case of commuting matrices, the Generalised Matrix-Exponential
Problem can be analysed block-wise, which leads us to the following
problem:

\begin{definition}
  An instance of the Linear-Exponential Problem (LEP) consists of a system
  of equations
\begin{align}
\label{single_eqn_form}
  \exp\left(\sum_{i \in I} \lambda_i^{(j)} t_i \right) = c_j \exp (d_j)
\quad (j \in J),
\end{align}
where $I$ and $J$ are finite index sets, the $\lambda_i^{(j)}$, $c_j$
and $d_j$ are complex algebraic constants, and the $t_i$ are complex
variables, together with a polyhedron
$\mathcal{P} \subseteq \mathbb{R}^{2k}$ that is specified by a system
of linear inequalities with algebraic coefficients.  The problem asks
to determine whether there exist $t_1,\ldots,t_k\in \mathbb{C}$ that
satisfy the system (\ref{single_eqn_form}) and such that
$(\Re(t_1),\ldots,\Re(t_k),\Im(t_1),\ldots,\Im(t_k))$
lies in $\mathcal{P}$.
\label{def:LEP}
\end{definition}

To establish decidability of the Linear-Exponential Problem, we reduce
it to the following
\emph{Algebraic-Logarithmic Integer Programming}
problem.  Here a \emph{linear form in logarithms of algebraic numbers}
is a number of the form
$\beta_{0} + \beta_{1} \log(\alpha_{1}) + \cdots + \beta_{m}
\log(\alpha_{m})$,
where
$\beta_{0}, \alpha_{1}, \beta_{1}, \ldots, \alpha_{m}, \beta_{m}$ are
algebraic numbers and $\log$ denotes a fixed branch of the complex
logarithm function.

% \exp(\boldsymbol{\lambda} \cdot \boldsymbol{t}) = c_{\boldsymbol{\lambda}} \exp(d_{\boldsymbol{\lambda}}), \quad \boldsymbol{\lambda} \in \overline{\mathbb{Q}}^{n}, c_{\boldsymbol{\lambda}}, d_{\boldsymbol{\lambda}} \in \overline{\mathbb{Q}}
% \end{align}
% and a convex polyhedron $\mathcal{P} \subseteq \mathbb{R}^{2k}$ with an algebraic description, decide whether (\ref{single_eqn_form}) admits a solution $\boldsymbol{t}$ such that $(\Re(\boldsymbol{t}), \Im(\boldsymbol{t})) \in \mathcal{P}$.
% \end{definition}

% In both cases, when a constraint of the form $(\Re(\boldsymbol{t}), \Im(\boldsymbol{t})) \in \mathcal{P}$ is imposed, where $\mathcal{P} \subseteq \mathbb{R}^{2k}$ is a convex polyhedron with an algebraic description, we refer to the problem as the \emph{constrained MEP/LEP}; the predicates defined above may often be combined, e.g. \emph{constrained commuting MEP}.

\begin{definition}
An instance of the Algebraic-Logarithmic Integer Programming Problem (ALIP) consists of a finite system of equations of the form
\begin{align*}
A \boldsymbol{x} \leq \frac{1}{\pi} \boldsymbol{b}
\end{align*}
where $A$ is an $m\times n$ matrix with real algebraic entries and
where the coordinates of $\boldsymbol{b}$ are real linear forms in
logarithms of algebraic numbers. The problem asks to determine whether
such a system admits a solution $\boldsymbol{x} \in \mathbb{Z}^{n}$.
\end{definition}

\subsection{Paper Outline}

After introducing the main mathematical techniques that are used in
the paper, we present a reduction from the Generalised Matrix
Exponential Problem with commuting matrices to the Linear-Exponential
Problem, as well as a reduction from the Linear-Exponential Problem to
the Algebraic-Logarithmic Integer Programming Problem, before finally showing that the Algebraic-Logarithmic Integer Programming Problem is decidable. By way of hardness, we will prove that the Matrix-Exponential Problem is
undecidable (in the non-commutative case), by reduction from Hilbert's
Tenth Problem.

%END LICS

%BEGIN HSCC
In ambient space $\mathbb{R}^{d}$, a \textbf{continuous linear
  dynamical system} is a trajectory $\boldsymbol{x}(t)$, where $t$
ranges over the non-negative reals, defined by a differential equation
$\dot{\boldsymbol{x}}(t)=f(\boldsymbol{x}(t))$ in which the function
$f$ is \emph{affine} or \emph{linear}. If the initial point
$\boldsymbol{x}(0)$ is given, the differential equation uniquely
defines the entire trajectory. (Linear) dynamical systems have been
extensively studied in Mathematics, Physics, and Engineering, and more
recently have played an increasingly important role in Computer
Science, notably in the modelling and analysis of cyber-physical
systems; a recent and authoritative textbook on the matter
is~\cite{Alu15}.

In the study of dynamical systems, particularly from the perspective
of control theory, considerable attention has been given to the study
of \emph{invariant sets}, i.e., subsets of $\mathbb{R}^{d}$ from which
no trajectory can escape; see, e.g.,
\cite{CastelanH92,BlondelT00,BM07,SDI08}. Our focus in the present
paper is on sets with the dual property that \emph{no trajectory
  remains trapped}. Such sets play a key role in analysing
\emph{liveness} properties in cyber-physical systems (see, for
instance,~\cite{Alu15}): discrete progress is ensured by
guaranteeing that all trajectories (i.e., from any initial starting
point) must eventually reach a point at which they `escape'
(temporarily or permanently) the set in question.

More precisely, given an affine function
$f:\mathbb{R}^{d}\rightarrow \mathbb{R}^{d}$ and a convex polytope
$\mathcal{P}\subseteq\mathbb{R}^d$, both specified using rational
coefficients encoded in binary, we consider the \textbf{Polytope
  Escape Problem} which asks whether there is some point
$\boldsymbol{x}_0$ in $\mathcal{P}$ for which the corresponding
trajectory of the solution to the differential equation
\begin{equation*}
\begin{displaystyle} \begin{cases}
\dot{\boldsymbol{x}}(t)=f(\boldsymbol{x}(t)) \\
\boldsymbol{x}(0)=\boldsymbol{x}_{0}
\end{cases} \end{displaystyle}
\end{equation*}
is entirely contained in $\mathcal{P}$. Our main result is to show
that this problem is decidable by reducing it in polynomial time to
the decision version of linear programming with real algebraic
coefficients, which itself reduces in polynomial time to deciding the
truth of a sentence in the first-order theory of the reals, a problem
whose complexity is known to lie between $\mathit{NP}$ and
$\mathit{PSPACE}$ \cite{Canny88}. Our algorithm makes use of spectral
techniques and relies among others on tools from Diophantine
approximation.

It is interesting to note that a seemingly closely related problem,
that of determining whether a given trajectory of a linear dynamical
system ever hits a given hyperplane (also known as the
\emph{continuous Skolem Problem}), is not known to be decidable; see,
in particular, \cite{BellDJB10,COW16a:ICALP16,COW16b:LICS16}. When the
target is instead taken to be a single point (rather than a
hyperplane), the corresponding reachability question (known as the
\emph{continuous Orbit Problem}) can be decided in polynomial
time~\cite{Hainry08}.
%END HSCC
