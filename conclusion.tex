\chapter{Conclusion}

%BEGIN SODA

%% Our techniques can actually be altered in order to solve another
%% related problem. One may ask if, given a convex semi-algebraic set, an
%% affine operator (under the same restrictions that we previously
%% imposed), and a polytope, whether some integral point in that convex,
%% semi-algebraic set will be such that its orbit by the affine operator
%% will intersect the polytope infinitely often. These set-to-set
%% reachibility problems remain largely unexplored. For example, it would
%% be interesting to consider the version of this problem where we omit
%% the requirement that this intersection holds infinitely often, but
%% instead ask whether it holds at least once, even though this would
%% probably involve some technology that was not needed in this paper,
%% such as Baker's theorem \cite{BW93}.

%% As opposed to what happens with the problems of deciding termination
%% over the reals and rationals, it seems to be hard to come up with a
%% non-constructive procedure for deciding termination over the
%% integers. Braverman conjectured in \cite{Bra06} that it would be
%% necessary that one is able to decide pointwise termination in order to
%% decide universal termination over the integers, but it seems unlikely
%% that one can formally reduce one problem to the other.

%% It is remarkable that, in some cases, solving the termination problem
%% over the integers is much easier than solving its real and rational
%% counterparts, namely when the polytope defined by the loop guard is
%% bounded, in which case this problem reduces to that of finding a cycle
%% in a graph.

%% While the \NP-hardness of this problem suggests that
%% termination over the integers is intrinsically harder, the gap between
%% this lower bound and our \EXPSPACE upper bound is far from
%% satisfactory.

We have shown decidability of termination of simple linear loops over
the integers under the assumption that the update matrix is
diagonalisable, partially answering an open problem
of~\cite{Tiw04,Bra06}.  As we have explained before, the termination
problem on the same class of linear loops, but for fixed initial
values, seems to have a different character and to be more difficult.
In this respect it is interesting to note that there are other
settings in which universal termination is an easier problem than
pointwise termination. For example, universal termination of Petri
nets (also known as \textit{structural boundedness}) is
\PTIME-decidable, but the pointwise termination problem is
\EXPSPACE-hard.

A natural subject for further work is whether our techniques can be
extended to non-diagonalisable matrices, or whether, as is the case
for pointwise termination~\cite{OW14:SODA}, there are unavoidable
number-theoretic obstacles to proving decidability.  We would also like
to further study the computational complexity of the termination
problem.  While there is a large gap between the \coNP lower
complexity bound mentioned in the Introduction and the exponential
space upper bound of our procedure, this may be connected with the
fact that our procedure computes a representation of the set of all
integer eventually non-terminating points. Finally we would like to
examine more carefully the question of whether the respective sets of
terminating and non-terminating points are semi-algebraic.  Note that
an \emph{effective} semi-algebraic characterisation of the set of
terminating points would allow us to solve the termination problem
over fixed initial values.

%% We conjecture that deciding eventual termination for all integer
%% points in a given convex semi-algebraic set is \EXPSPACE-hard,
%% as is the problem of deciding termination for all integer points in a
%% given polytope \cite{BGM12}, but believe that better bounds on the
%% universal termination problem over the integers should be possible to
%% achieve.

%% Finally, we conjecture that deciding arbitrary instances of this
%% problem with non-diagonalisable update matrices in dimension at least
%% $5$ should be hard, in a similar sense to the one described in
%% \cite{OW14:SODA}.

%END SODA

%BEGIN LICS

We have shown that the Matrix-Exponential Problem is undecidable in
general, but decidable when the matrices $A_{1}, \ldots, A_{k}$ commute.
This is analogous to what was known for the discrete version
of this problem, in which the matrix exponentials $e^{At}$ are
replaced by matrix powers $A^n$.

A natural variant of this problem is the following:
\begin{definition}[Matrix-Exponential Semigroup Problem]
  Given square matrices $A_{1}, \ldots, A_{k}$ and $C$, all of the
  same dimension and all with real algebraic entries, is $C$ a member
  of the matrix semigroup generated by
\begin{align*}
\lbrace \exp(A_{i} t_{i}) : t_{i} \geq 0 , i=1,\ldots,k \rbrace ?
\end{align*}
\end{definition}
When the matrices $A_1,\ldots,A_k$ all commute, the above problem is
equivalent to the Matrix-Exponential Problem, and therefore decidable. In the non-commutative case, the following result holds:
\begin{theorem}
The Matrix-Exponential Semigroup Problem is undecidable.
\end{theorem}
A proof will appear in a future journal version of this paper. This can be done by reduction from the Matrix-Exponential Problem, using a set of gadgets to force a desired order in the multiplication of the matrix exponentials.

It would also be interesting to look at possibly decidable
restrictions of the MEP/MESP, for example the case where $k=2$ with a
non-commuting pair of matrices, which was shown to be decidable for
the discrete analogue of this problem in \cite{MEHTP}. Bounding the dimension of the ambient vector space could also yield decidability, which has been partly accomplished in the discrete case in \cite{CK05}. Finally, upper bounding the complexity of our decision procedure for the commutative case would also be a worthwhile task.
%END LICS

%BEGIN HSCC
We have shown that the Polytope Escape Problem for continuous-time
linear dynamical systems is decidable, and in fact, polynomial-time
reducible to the decision problem for the existential theory of real
closed fields.  Given an instance of the problem $(f,\mathcal{P})$,
with $f$ an affine map, our decision procedure involves analysing the real
eigenstructure of the linear operator
$g(\myvector{x}):=f(\myvector{x})-f(\myvector{0})$. In fact, we
showed that all complex eigenvalues could essentially be ignored for
the purposes of deciding this problem.

Interestingly, the seemingly closely related question of whether a
given single trajectory of a linear dynamical system remains trapped
within a given polytope, on the other hand, appears to be
considerably more challenging and is not known to be decidable. In
that instance, it seems that the influence of the complex
eigenstructure cannot simply be discarded.
%END HSCC
