\section{Some undecidable matrix problems}

\label{matrix-undecidability}

Given $k+1$ invertible matrices $A_{1}, \ldots, A_{k},C \in \mathbb{Q}^{d \times d}$, the \emph{generalized matrix powering problem for invertible matrices} consists in deciding whether there exist $n_{1}, \ldots, n_{k} \in \mathbb{Z}$ such that
\begin{equation*}
\prod\limits_{i=1}^{k} A_{i}^{n_{i}} = C.
\end{equation*}

\begin{theorem}
The generalized matrix powering problem for invertible matrices is undecidable.
\end{theorem}

\begin{proof}
We will show this result by reducing from Hilbert's Tenth Problem. Given a polynomial $p \in \mathbb{Z}[n_{1}, \ldots, n_{k}]$, it is easy to express $p(n_{1}, \ldots, n_{k})$ as a conjunction of relations of the following form (noting that we may need to introduce new variables):
\begin{itemize}
\item $z = k$, where $k \in \mathbb{Z}$
\item $z = x+y$
\item $z = xy$.
\end{itemize}
We start by showing how to encode each of these as an instance of the generalized matrix powering problem for invertible matrices.
Firstly, note that
\begin{equation*}
z = k \Leftrightarrow
\begin{pmatrix}
    1 & 1 \\
    0 & 1
\end{pmatrix}^{z} =
\begin{pmatrix}
    1 & k \\
    0 & 1
\end{pmatrix}.
\end{equation*}
Secondly, note that
\begin{equation*}
    z = x + y \Leftrightarrow
\begin{pmatrix}
    1 & 1 \\
    0 & 1
\end{pmatrix}^{x}
\begin{pmatrix}
    1 & 1 \\
    0 & 1
\end{pmatrix}^{y}
\begin{pmatrix}
    1 & -1 \\
    0 & 1
\end{pmatrix}^{z} =
\begin{pmatrix}
    1 & 0 \\
    0 & 1
\end{pmatrix}.
\end{equation*}
Thirdly, note that
\begin{equation*}
    z = xy \Leftrightarrow
    \exists x',y' \in \mathbb{Z},
    \begin{pmatrix}
        1 & 0 & 0 \\
        0 & 1 & 0 \\
        0 & 0 & 1
    \end{pmatrix} =
    \begin{pmatrix}
        1 & x-x' & z-xy \\
        0 & 1 & y-y' \\
        0 & 0 & 1
    \end{pmatrix}
\end{equation*}
and that the latter matrix is just equal to
\begin{equation*}
    \begin{pmatrix}
        1 & 0 & -1 \\
        0 & 1 & 0 \\
        0 & 0 & 1
    \end{pmatrix}^{z}
    \begin{pmatrix}
        1 & 0 & 0 \\
        0 & 1 & -1 \\
        0 & 0 & 1
    \end{pmatrix}^{y'}
    \begin{pmatrix}
        1 & 1 & 0 \\
        0 & 1 & 0 \\
        0 & 0 & 1
    \end{pmatrix}^{x}
    \begin{pmatrix}
        1 & 0 & 0 \\
        0 & 1 & 1 \\
        0 & 0 & 1
    \end{pmatrix}^{y}
    \begin{pmatrix}
        1 & -1 & 0 \\
        0 & 1 & 0 \\
        0 & 0 & 1
    \end{pmatrix}^{x'}.
\end{equation*}
Finally, conjunction can be achieved by making use of separate matrix blocks:
\begin{equation*}
    \prod\limits_{i=1}^{k} A_{i}^{n_{i}} = C \wedge \prod\limits_{i=1}^{k} B_{i}^{n_{i}} = D \Leftrightarrow
    \prod\limits_{i=1}^{k} \begin{pmatrix}A_{i} & 0 \\ 0 & B_{i}\end{pmatrix}^{n_{i}} = \begin{pmatrix}C & 0 \\ 0 & D\end{pmatrix}.
\end{equation*}
\end{proof}

\begin{definition}
Given invertible matrices $A_{1}, \ldots, A_{k} \in \mathbb{Q}^{d \times d}$ and two non-zero vectors $\boldsymbol{x}, \boldsymbol{y} \in \mathbb{Q}^{d}$, the \emph{vector reachability problem for invertible matrices} consists in deciding whether there exist $n_{1}, \ldots, n_{k} \in \mathbb{Z}$ such that
\begin{equation*}
\prod\limits_{i=1}^{k}A_{i}^{n_{i}} \boldsymbol{x} = \boldsymbol{y}.
\end{equation*}
\end{definition}

\begin{theorem}
The vector reachability problem for invertible matrices is undecidable.
\end{theorem}

\begin{proof}
This can be shown by reduction from the generalised matrix powering problem for invertible matrices. In particular, given invertible matrices $A_{1}, \ldots, A_{k}, B \in \mathbb{Q}^{d \times d}$, letting $\boldsymbol{b}_{1}, \ldots, \boldsymbol{b}_{d}$ denote the columns of $B$, and letting $\boldsymbol{e}_{1}, \ldots, \boldsymbol{e}_{d}$ denote the canonical basis of $R^{d}$, the result follows from the fact that
    \begin{equation*}
        \prod\limits_{i=1}^{k} A_{i}^{n_{i}} = B \Leftrightarrow
        \prod\limits_{i=1}^{k}
        \begin{pmatrix}
            A_{i} & \cdots & 0 \\
            \vdots& \ddots & \vdots \\
            0 & \cdots & A_{i}
        \end{pmatrix}^{n_{i}}
        \begin{pmatrix}
            \boldsymbol{e}_{1} \\
            \vdots \\
            \boldsymbol{e}_{d}
        \end{pmatrix} =
        \begin{pmatrix}
            \boldsymbol{b}_{1} \\
            \vdots \\
            \boldsymbol{b}_{d}
        \end{pmatrix}.
    \end{equation*}
\end{proof}
