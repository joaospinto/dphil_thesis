\section{Conclusion}

Complete state controllability for LTI systems has long been characterised by Kalman's criterion, which states that such a system is controllable (that is, any state can be controlled to any other one in finite time) if and only if the controllability matrix has full row rank. We studied the hardness of deciding controllability for a given pair of states. In particular, if the set of controls is rich enough (finite union of convex polytopes), we showed that this problem is undecidable by encoding Hilbert's Tenth Problem. Even when the set of controls is a convex polytope, we proved Skolem-hardness (actually, we proved Positivity-hardness, which is stronger). The problem becomes decidable when the set of controls is a linear subspace, by reduction to the Continuous Orbit Problem.

It would be interesting to get a tighter characterisation of the hardness of this problem when the set of controls is a convex polytope, either by showing undecidability, or by giving an algorithm that queries an oracle to a more famous open problem, such as the Positivity Problem.
