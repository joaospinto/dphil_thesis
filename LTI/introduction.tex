\section{Introduction}
In this chapter, we study a basic problem in control theory, namely the point-to-point controllability problem for both continuous- and discrete-time linear time-invariant (henceforth LTI) systems.

A discrete-time LTI system $(\myvector{x}_{n})_{n \in \Naturals} \subseteq \Reals^{d}$ with control set $\mathcal{U} \subseteq \Reals^{d}$ satisfies the evolution rule $\myvector{x}_{n+1} = A \myvector{x}_{n} + \myvector{u}_{n}$, where $A$ is a matrix of adequate dimensions and $\myvector{u}_{n} \in \mathcal{U}$.
Therefore, the next state of the system is obtained by applying a time-invariant linear function to the previous state and adding a control from a time-invariant set of controls.

Given two points $\myvector{s}, \myvector{t} \in \mathbb{R}^{d}$, the \emph{point-to-point controllability question for discrete-time LTI systems} consists in deciding whether there exist $n \in \Naturals$ and $(\myvector{u}_{i})_{i=0}^{n-1} \subseteq \mathcal{U}$ such that $\myvector{x}_{0} = \myvector{s}$ and $\myvector{x}_{n} = \myvector{t}$.

We show that this problem is undecidable when $\mathcal{U}$ is non-convex (in particular, when it is a finite union of convex polytopes) and that it is Skolem-hard even when $\mathcal{U}$ is a convex polytope. This contrasts with the case where $\mathcal{U}$ is a linear subspace, in which case the problem reduces to the Orbit Problem, known to be decidable.

Similarly, a continuous-time LTI system $\myvector{x}(t)$ with control set $\mathcal{U}$ is a function satisfying the evolution rule $\dot{\myvector{x}}(t) = A \myvector{x}(t) + \myvector{u}(t)$
where $A$ is a matrix of adequate dimensions and $\myvector{u}$ is a measurable function with codomain $\mathcal{U}$. The \emph{point-to-point controllability question for continuous-time LTI systems} consists in deciding whether there exist $t \geq 0$ and $\myvector{u} : [0,t] \rightarrow \mathcal{U}$ such that $\myvector{x}(0) = \myvector{s}$ and $\myvector{x}(t) = \myvector{t}$.

A survey of computational complexity results in control theory can be found in~\cite{BlondelT00}.
