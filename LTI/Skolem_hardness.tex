\section{Hard Instances}

In this section, we present a hardness result for the point-to-point controllability problem for LTI systems where the set of admissible controls is a compact convex polytope.

\begin{definition}
Given a matrix $A \in \Rationals^{d \times d}$ and two vectors
$\myvector{u}, \myvector{v} \in \Rationals^{d}$, \emph{Skolem's problem} consists in determining
whether there exists a number $n \in \Naturals$ such that
$\myvector{u}^{T} A^{n} \myvector{v} = 0$, and the \emph{positivity problem} consists in determining whether there exists a number $n \in \Naturals$ such that $\myvector{u}^{T} A^{n} \myvector{v} \geq 0$. A decision problem is said to be \emph{Skolem-hard} if Skolem's problem can be reduced to it; we define \emph{positivity-hardness} accordingly.
\end{definition}

Skolem's problem is a long-standing open problem in logic and number theory, whose decidability has not yet been determined. It is well known that the positivity-problem is Skolem-hard \cite{PP}.

Instead of reducing from the positivity problem directly, we shall proceed by reducing from the following problem, which has been shown to be positivity-hard in \cite{MRP}:
\begin{definition}
Given a (column-)stochastic matrix $M \in \Rationals^{d \times d}$ and a number $r \in \Rationals \cap [0,1]$,
the \emph{Markov Reachability Problem} consists in determining whether there exists a number $n \in \Naturals$ such that $\left( M^{n} \right)_{1,2} \geq r$.
\end{definition}

We will now show the following result:

\begin{theorem}
The point-to-point controllability problem for LTI systems whose set of admissible controls are compact convex polytopes is positivity-hard.
\end{theorem}

\begin{proof}
We do this by presenting a (polynomial-time) reduction from the Markov Reachability Problem to the problem at hand.

Given a (column-)stochastic matrix $M \in \Rationals^{d \times d}$ and a number $r \in \Rationals \cap [0,1]$, we define the matrix $A=\diag{M, 0, 1, 0, 1} \in \Rationals^{(d+4)\times (d+4)}$ and
the compact convex polytope
\begin{equation*}
\mathcal{P} = \lbrace (-\myvector{x}, y, y, \myvector{x} \cdot \myvector{1}, \myvector{x} \cdot \myvector{1}), \myvector{x} \geq \myvector{0}, \myvector{x} \cdot \myvector{1} \leq 1, y \leq \myvector{x} \cdot \myvector{e}_{1} \rbrace \subseteq \Reals^{d+4},
\end{equation*}
as well as the source $\myvector{s} = (\myvector{e}_{2}, 0, 0, 0, 0) \in \Rationals^{d+4}$ and target $\myvector{t} = (\myvector{0}, r, r, 1, 1) \in \Rationals^{d+4}$.

First, suppose that there exists $n \in \Naturals$ such that $\left( M^{n} \right)_{1,2} \geq r$. Consider the sequence of controls $\left( \myvector{u}_{i} \right)_{i=0}^{i=n-1} \subseteq \mathcal{P}$ given by $\myvector{u}_{0} = \cdots = \myvector{u_{n-2}} = \myvector{0}$ and $\myvector{u}_{n-1}$ is the element from $\mathcal{P}$ corresponding to $\myvector{x} = M^{n} \myvector{e}_{2}$ and $y=\left( M^{n} \right)_{1,2}$. This sequence clearly controls $\myvector{s}$ to $\myvector{t}$.

On the other hand, suppose there exists a sequence $\left( \myvector{u}_{i} \right)_{i=0}^{i=n-1}$ controlling $\myvector{s}$ to $\myvector{t}$.
From the fact that the coordinates $d+1$ and $d+2$ of $\myvector{t}$ are equal, as are the coordinates $d+3$ and $d+4$, and noting that the matrix $A$ erases coordinates $d+1$ and $d+3$ but not $d+2$ and $d+4$, it follows that $\myvector{u}_{n-1}$ is the only non-zero control, that is, $\myvector{u}_{0} = \cdots = \myvector{u}_{n-2} = \myvector{0}$.
Therefore, at time $n-1$, we will be in state $(M^{n-1} \myvector{e}_{2}, 0, 0, 0, 0)$, and the only way to reach $\myvector{t}$ in the remaining step is to take $\myvector{u}_{n-1} \in \mathcal{P}$ with $\myvector{x} = M^{n} \myvector{e}_{2}$ and $y=r$, otherwise one of the first $d$ coordinates will be non-zero. It then follows that $\left( M^{n} \right)_{1,2} \geq r$, by looking at coordinate $d+1$.
\end{proof}
