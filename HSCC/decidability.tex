\section{The Polytope Escape Problem}

The Polytope Escape Problem for continuous linear dynamical systems
consists of deciding, given an affine function
$f:\Reals^{d}\rightarrow \Reals^{d}$ and a convex polytope
$\mathcal{P}\subseteq\Reals^{d}$, whether there exists an initial point
$\myvector{x}_{0} \in \mathcal{P}$ for which the trajectory of the unique
solution to the differential equation
$\dot{\myvector{x}}(t)=f(\myvector{x}(t)),\myvector{x}(0)=\myvector{x}_{0},
t\geq 0$,
is entirely contained in $\mathcal{P}$.  A starting point
$\myvector{x}_{0}\in\mathcal{P}$ is said to be \textbf{trapped} if
the trajectory of the corresponding solution is contained in $\mathcal{P}$,
and \textbf{eventually trapped} if the trajectory of the corresponding
solution contains a trapped point. Therefore, the Polytope Escape
Problem amounts to deciding whether a trapped point exists, which in
turn is equivalent to deciding whether an eventually trapped point exists.

The goal of this section is to prove the following result:

\begin{theorem}
  The Polytope Escape Problem is polynomial-time reducible to the
  decision version of linear programming with algebraic coefficients.
\end{theorem}

A $d$-dimensional instance of the Polytope Escape Problem is a pair
$(f,\mathcal{P})$, where $f:\Reals^{d}\rightarrow \Reals^{d}$
is an affine function and $\mathcal{P}\subseteq\Reals^{d}$ is a
convex polytope. In this formulation we assume that all numbers
involved in the definition of $f$ and $\mathcal{P}$ are
rational.\footnote{The assumption of rationality is required to
  justify some of our complexity claims (e.g., Jordan Canonical Forms
  are only known to be polynomial-time computable for matrices with
  rational coordinates). Nevertheless, our procedure remains valid in
  a more general setting, and in fact, the overall
  $\exists \Reals$ complexity of our algorithm would not be
  affected if one allowed real algebraic numbers when defining problem
  instances.}

An instance $(f,\mathcal{P})$ of the Polytope Escape Problem is said
to be \textbf{homogeneous} if $f$ is a linear function and
$\mathcal{P}$ is a convex polytope cone (in particular,
$\myvector{x}\in\mathcal{P},\alpha>0\Rightarrow \alpha\myvector{x}
\in\mathcal{P}$).
%Moreover, an homogeneous instance of the Polytope Escape Problem is said to be \textbf{real} if all eigenvalues of $f$ are real.

The restriction of the Polytope Escape Problem to homogeneous
instances is called the homogeneous Polytope Escape Problem.

\begin{lemma}
  The Polytope Escape Problem is polynomial-time reducible to the
  homogeneous Polytope Escape Problem.
\end{lemma}

\begin{proof}
  Let $(f,\mathcal{P})$ be an instance of the Polytope Escape
  Problem in $\Reals^{d}$, and write
\begin{equation*}
f(\myvector{x})=A\myvector{x}+\myvector{a}\mbox{ and } \mathcal{P}=\lbrace\myvector{x}\in\Reals^{d}: B_{1}\myvector{x}>\myvector{b}_{1} \wedge B_2\myvector{x}\geq \myvector{b}_{2}\rbrace \, .
\end{equation*}
Now define
\begin{align*}
& A'=\begin{pmatrix}A & \myvector{a}\\ \myvector{0}^T & 0\end{pmatrix},
B_{1}'=\begin{pmatrix}B_{1} & -\myvector{b}_{1} \\ \myvector{0}^T & 1\end{pmatrix},
B_{2}'=\begin{pmatrix}B_{2} & -\myvector{b}_{2} \end{pmatrix} \, ,\\
& \mathcal{P}'=
\left\lbrace
\begin{pmatrix}\myvector{x}\\y
\end{pmatrix} \in\Reals^{d+1}:B_{1}'\begin{pmatrix}\myvector{x}\\y
\end{pmatrix}
  > \myvector{0}
  \wedge B_{2}'\begin{pmatrix}\myvector{x}\\y
\end{pmatrix} \geq \myvector{0} \right\rbrace \, ,
\intertext{and}
& g\begin{pmatrix}\myvector{x}\\y
\end{pmatrix} = A'\begin{pmatrix}\myvector{x}\\y
\end{pmatrix} \, .
\end{align*}
Then $(g,\mathcal{P}')$ is a homogeneous instance
of the Polytope Escape Problem.

It is clear that $\myvector{x}(t)$ satisfies the differential equation
$\dot{\myvector{x}}(t)=f(\myvector{x}(t))$ if and only if
$\begin{pmatrix}
\myvector{x}(t)\\1 \end{pmatrix}$ satisfies
the differential equation
$\begin{pmatrix}\dot{\myvector{x}}\\
\dot{y}
\end{pmatrix} = g\begin{pmatrix}\myvector{x}\\y
\end{pmatrix}$.  In general,
in any trajectory $\begin{pmatrix}\myvector{x}\\y
\end{pmatrix}$ that satisfies this last differential equation, the
$y$-component must be constant.

We claim that $(f,\mathcal{P})$ is a positive instance
of the Polytope Escape Problem if and
only if $(g,\mathcal{P}')$ is a positive instance.
Indeed, if the point
$\myvector{x}_0 \in \Reals^{d}$ is trapped
in $(f,\mathcal{P})$
then the point $\begin{pmatrix}\myvector{x}_0\\1
\end{pmatrix}$ is trapped in $(g,\mathcal{P}')$.
Conversely, suppose that $\begin{pmatrix}\myvector{x}_0\\y_0
\end{pmatrix}$ is trapped in  $(g,\mathcal{P}')$.  Then,
since $B_{1}'\begin{pmatrix}\myvector{x}_0\\y_0
\end{pmatrix}
> \myvector{0}$,
we must have $y_0>0$.  Scaling, it follows that
$\begin{pmatrix} y_0^{-1}\myvector{x}_0\\1
\end{pmatrix}$ is also trapped in $(g,\mathcal{P}')$.  This implies
that $y_0^{-1}\myvector{x}_0$ is trapped in $(f,\mathcal{P})$.
%
%In fact, if $\myvector{x}_{0}$ is
%a trapped point for $(f,\mathcal{P})$, then $(\myvector{x}_{0},1)$
%will be a trapped point for $(A',\mathcal{P}')$. Conversely, if
%$(\myvector{x}_{0},c)$ is a trapped point for $(A',\mathcal{P}')$,
%then $c^{-1}\myvector{x}_{0}$ is a trapped point for
%$(f,\mathcal{P})$. Note that $c>0$ must hold.
\end{proof}

We remind the reader that the unique solution to the differential equation $\dot{\myvector{x}}(t)=f(\myvector{x}(t)),\myvector{x}(0)=\myvector{x}_{0},t\geq 0$, where $f(\myvector{x})=A\myvector{x}$, is given by
$\myvector{x}(t)=\exp(At)\myvector{x}_{0}$.

In this setting, the sets of trapped and eventually trapped points are, respectively:
\begin{align*}
\mathit{T}&=\lbrace \myvector{x}_0\in\Reals^{d}: \forall t\geq 0, \exp(At)\myvector{x}_{0} \in \mathcal{P}\rbrace \\
\mathit{ET}&=\lbrace\myvector{x}_{0}\in\Reals^{d}:\exists t\geq 0,\exp(At)\myvector{x}_{0}\in\mathit{T}\rbrace
\end{align*}

Note that both $\mathit{T}$ and $\mathit{ET}$ are convex subsets of $\Reals^{d}$.

\begin{lemma}
  The homogeneous Polytope Escape Problem is polynomial-time
  reducible to the decision version of linear programming with
  algebraic coefficients.
\end{lemma}

\begin{proof}
  % Given an homogeneous instance of the Polytope Escape Problem
  % $(A,\mathcal{P})$, we can create an equivalent real homogeneous
  % instance of the Polytope Escape Problem $(A',\mathcal{P}')$ by
  % picking a basis
  % $\mathcal{B}=\lbrace \myvector{u}_{1},\ldots,\myvector{u}_{k}
  % \rbrace$
  % of $\mathcal{V}^{r}$ and taking $A'$ to be the matrix
  % representation of $A\vert_{\mathcal{V}^{r}}$ with respect to
  % $\mathcal{B}$ and
%\begin{align*}
%\mathcal{P}'=\lbrace\myvector{x}\in \Reals^{k} : \sum\limits_{i=1}^{k} x_{i}\myvector{u}_{i}\in\mathcal{P}\rbrace
%\end{align*}
%which is still a cone.

  Let
  $\myvector{x}_{0}=\myvector{x}_{0}^{r}+\myvector{x}_{0}^{c}$,
  where $\myvector{x}_{0}^{r}\in \mathcal{V}^{r}$ and
  $\myvector{x}_{0}^{c}\in \mathcal{V}^{c}$. We start by showing
  that if $\myvector{x}_{0}$ lies in the set $T$ of trapped points
  then its component $\myvector{x}_{0}^{r}$ in the real eigenspace
  $\mathcal{V}^{r}$ lies in the set $\mathit{ET}$ of eventually
  trapped points.
Due to the
fact that the intersection of finitely many convex polytopes is still
a convex polytope, it suffices to prove this claim for the case when
  $\mathcal{P}$ is defined by a single inequality---say
  $\mathcal{P}=\lbrace \myvector{x}\in\Reals^{d}:
  \myvector{b}^{T}\myvector{x}\triangleright 0\rbrace$, where
  $\triangleright$ is either $>$ or $\geq$.

  We may assume that
  $\myvector{b}^{T} \exp(At) \myvector{x}_{0}^{c}$ is not identically
  zero, as in that case
  $\myvector{b}^{T} \exp(At)\myvector{x}_{0} \equiv
  \myvector{b}^{T} \exp(At)\myvector{x}_{0}^{r}$ and our claim
  holds trivially.
  Also, if $\myvector{x}_{0}\in\mathit{T}$, it cannot hold that
  $\myvector{b}^{T} \exp(At) \myvector{x}_{0}^{r} \equiv 0$, since
  $\myvector{b}^{T} \exp(At) \myvector{x}_{0}^{c}$ is negative
  infinitely often by \cref{cor:liminf}.

  Suppose that $\myvector{x}_{0}\in\mathit{T}$ and let $(\rho,m)$
  and $(\eta,j)$ be the dominant indices for
  $\myvector{b}^{T} \exp(At) \myvector{x}_{0}^{r}$ and
  $\myvector{b}^{T} \exp(At) \myvector{x}_{0}^{c}$ respectively.
Then by \cref{prop:linear} we have
\begin{gather}
\myvector{b}^{T} \exp(At) \myvector{x}_{0}^{r} = \exp(\rho t)t^m (c
  + o(1))
\label{eq:real}
\end{gather}
 as $t \rightarrow \infty$, where $c$ is a non-zero real
number.  We will show that $c>0$, from which it follows  that
$\myvector{x}_{0}^{r} \in \mathit{ET}$.

%From this it follows that
%\begin{equation*}
%c=\lim\limits_{t\rightarrow \infty}
%\frac{\myvector{b}^{T}\exp(At)\myvector{x}_{0}^{r}}{\exp(\rho t)
%  t^{m}} \, .
%\end{equation*}
% From the fact that $(\rho,m)$ is dominant for
% $\myvector{b}^{T} \exp(At) \myvector{x}_{0}^{r}$, it follows that
% $\myvector{b}^{T} \exp(At) \myvector{x}_{0}^{r}

% $c$ is the coefficient of the term $t^me^{\rho t}$ in
% $\myvector{b}^{T} \exp(At) \myvector{x}_{0}^{r}$, and is therefore
% non-zero.

It must hold that $(\eta,j)\preceq (\rho,m)$.  Indeed, if $(\eta,j)\succ
(\rho,m)$, then, as $t\rightarrow\infty$,
\begin{align*}
\myvector{b}^{T} \exp(At) \myvector{x}_{0} = \exp(\eta t)t^{j} \Bigg(
\underbrace{\frac{\myvector{b}^{T} \exp(At) \myvector{x}_{0}^{c}}{\exp(\eta t)t^{j}}}_{\mytag{termA}} + o(1)\Bigg) ,
\end{align*}
%
but the limit inferior of the term~\ref{termA} above is strictly
negative by \cref{cor:liminf},
contradicting the fact that $\myvector{x}_{0}\in\mathit{T}$.

If $(\eta,j)=(\rho,m)$, then, as $t\rightarrow\infty$,
\begin{align*}
\myvector{b}^{T} \exp(At) \myvector{x}_{0} = \exp(\rho t)t^{m} \left( c + \frac{\myvector{b}^{T} \exp(At) \myvector{x}_{0}^{c}}{\exp(\rho t)t^{m}} + o(1) \right) ,
\end{align*}
and by invoking \cref{cor:liminf} as above, it follows that
$c > 0$.

% from which we conclude that
% $\myvector{x}_{0}^{r}\in\mathit{ET}$, since, as
% $t\rightarrow\infty$,
% \begin{align*}
% \myvector{b}^{T} \exp(At) \myvector{x}_{0}^{r} = \exp(\rho t)t^{m}
%   \left( c +o(1) \right) \, .
% \end{align*}
Finally, if $(\eta,j)\prec (\rho,m)$, then, as $t\rightarrow\infty$,
\begin{gather}\myvector{b}^{T} \exp(At) \myvector{x}_{0}^c =
\exp(\rho t)t^{m} \cdot o(1) \, ,
\label{eq:complex}
\end{gather} and hence, by (\ref{eq:real}) and (\ref{eq:complex}), it
follows that
\begin{align*}
\myvector{b}^{T} \exp(At) \myvector{x}_{0} = \exp(\rho t)t^{m} \left( c +o(1) \right) \, .
\end{align*}
From the fact that $\myvector{x}_{0} \in T$ and that $c\neq 0$ we must have $c>0$.

In all cases it holds that $c>0$ and hence
$\myvector{x}_{0}^{r}\in\mathit{ET}$.


%\end{proof}
%\begin{lemma}
%The restriction of the Polytope Escape Problem to real homogeneous instances is decidable in polynomial time.
%\end{lemma}
%\begin{proof}

%  Suppose we are given a real homogeneous instance of the Polytope
%  Escape Problem $(A,\mathcal{P})$. We claim that, in this setting,
%  the set $\mathit{ET}$ is a convex polytope that we can
%  efficiently compute.



Having argued that
$\mathit{ET}\neq\emptyset$ iff $\mathit{ET}\cap
\mathcal{V}^{r}\neq\emptyset$,
we will now show that the set $\mathit{ET}\cap \mathcal{V}^{r}$ is a
convex polytope that we can efficiently compute. As before, it
suffices to prove this claim for the case when
$\mathcal{P}=\lbrace \myvector{x}\in\Reals^{d}:
\myvector{b}^{T}\myvector{x}\triangleright 0\rbrace$
(where $\triangleright$ is either $>$ or $\geq$).

In what follows, we let $[K]$ denote the set $\lbrace 0,\ldots,K-1 \rbrace$. We can write
\begin{align*}
\myvector{b}^{T}\exp(At)=\sum\limits_{(\eta,j)\in\sigma(A)\times
  [\nu(A)]} \exp(\eta t)t^{j} \myvector{u}_{(\eta,j)}^{T} \, ,
\end{align*}
where $\myvector{u}_{(\eta,j)}^{T}$ is the vector of coefficients of
$t^j\exp(\eta t)$ in $\myvector{b}^{T}\exp(At)$.

Note that if $\myvector{x}\in\mathcal{V}^{r}$ and $(\eta,j)\in
(\sigma(A)\setminus \Reals)\times \Naturals_{0}$, then
$\myvector{u}_{(\eta,j)}^{T}\myvector{x}=0$, as
$\myvector{u}_{(\eta,j)}^{T}\myvector{x}$ is the coefficient of $t^{j} \exp(\eta
t)$ in $\myvector{b}^{T}\exp(At) \myvector{x}$, and $\mathcal{V}^{r}$ is
invariant under $\exp(At)$. Moreover,
%
\begin{align*}
\mathit{ET}\cap\mathcal{V}^{r}=(\mathcal{B}\cap\mathcal{C}) \cup
\begin{cases}
\lbrace \myvector{0} \rbrace & \text{ if } \triangleright \text{ is } \geq \\
\emptyset & \text{ if } \triangleright \text{ is } >
\end{cases}
%\mathit{ET}\cap\mathcal{V}^{r}=(\mathcal{B}\cap\mathcal{C})\cup\lbrace \myvector{0}\mbox{ if }\triangleright\mbox{ is }\geq\mbox{ as opposed to }> \rbrace
\end{align*}
where
\begin{align*}
\mathcal{B}=&\bigcap\limits_{(\eta,j)\in (\sigma(A)\setminus \Reals)\times [\nu(A)]} \lbrace \myvector{x}\in\Reals^{d}: \myvector{u}_{(\eta,j)}^{T} \myvector{x}=0 \rbrace \\
\mathcal{C}=&\bigcup\limits_{(\eta,j)\in(\sigma(A)\cap \Reals)\times [\nu(A)]} \bigg[ \lbrace \myvector{x}\in\Reals^{d}: \myvector{u}_{(\eta,j)}^{T} \myvector{x}>0 \rbrace \cap \\
&\bigcap\limits_{(\rho,m)\succ (\eta,j)} \lbrace \myvector{x}\in\Reals^{d}: \myvector{u}_{(\rho,m)}^{T} \myvector{x}=0 \rbrace \bigg]
\end{align*}

The set $\mathit{ET}\cap\mathcal{V}^{r}$ can be seen to be convex from
the above characterisation. Alternatively, note that $\mathit{ET}$ can
be shown to be convex from its definition and that $\mathcal{V}^{r}$
is convex, therefore so must be their intersection. Thus
$\mathit{ET}\cap\mathcal{V}^{r}$ must be a convex polytope whose
definition possibly involves canonically-represented real algebraic
numbers, and the Polytope Escape Problem reduces to testing this
polytope for non-emptiness.
\end{proof}
