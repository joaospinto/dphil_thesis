\section{Introduction}

In ambient space $\Reals^{d}$, a \emph{continuous linear
  dynamical system} is a trajectory $\myvector{x}(t)$, where $t$
ranges over the non-negative reals, defined by a differential equation
$\dot{\myvector{x}}(t)=f(\myvector{x}(t))$ in which the function
$f$ is \emph{affine} or \emph{linear}. If the initial point
$\myvector{x}(0)$ is given, the differential equation uniquely
defines the entire trajectory. (Linear) dynamical systems have been
extensively studied in Mathematics, Physics, and Engineering, and more
recently have played an increasingly important role in Computer
Science, notably in the modelling and analysis of cyber-physical
systems; a recent and authoritative textbook on the matter
is~\cite{Alu15}.

In the study of dynamical systems, particularly from the perspective
of control theory, considerable attention has been given to the study
of \emph{invariant sets}, i.e., subsets of $\Reals^{d}$ from which
no trajectory can escape; see, e.g.,
\cite{CastelanH92,BlondelT00,BM07,SDI08}. Our focus in the present
chapter is on sets with the dual property that \emph{no trajectory
  remains trapped}. Such sets play a key role in analysing
\emph{liveness} properties in cyber-physical systems (see, for
instance,~\cite{Alu15}): discrete progress is ensured by
guaranteeing that all trajectories (i.e., from any initial starting
point) must eventually reach a point at which they `escape'
(temporarily or permanently) the set in question.

More precisely, given an affine function
$f:\Reals^{d}\rightarrow \Reals^{d}$ and a convex polytope
$\mathcal{P}\subseteq\Reals^{d}$, both specified using rational
coefficients encoded in binary, we consider the \emph{Polytope
  Escape Problem} which asks whether there is some point
$\myvector{x}_0$ in $\mathcal{P}$ for which the corresponding
trajectory of the solution to the differential equation
\begin{equation*}
\begin{displaystyle} \begin{cases}
\dot{\myvector{x}}(t)=f(\myvector{x}(t)) \\
\myvector{x}(0)=\myvector{x}_{0}
\end{cases} \end{displaystyle}
\end{equation*}
is entirely contained in $\mathcal{P}$. Our main result is to show
that this problem is decidable by reducing it in polynomial time to
the decision version of linear programming with real algebraic
coefficients, which itself reduces in polynomial time to deciding the
truth of a sentence in the first-order theory of the reals, a problem
whose complexity is known to lie between $\NP$ and
$\PSPACE$~\cite{Canny88}. Our algorithm makes use of spectral
techniques and relies among others on tools from Diophantine
approximation.

It is interesting to note that a seemingly closely related problem,
that of determining whether a given trajectory of a linear dynamical
system ever hits a given hyperplane (also known as the
\emph{continuous Skolem Problem}), is not known to be decidable; see,
in particular,~\cite{ContinuousSkolem,ContinuousSkolem3,COW16b:LICS16}. When the
target is instead taken to be a single point (rather than a
hyperplane), the corresponding reachability question (known as the
\emph{continuous Orbit Problem}) can be decided in polynomial
time~\cite{Hainry08}.
